\documentclass[10pt]{dokument-tiwo}


\begin{document}


\Tytul{Plan testowania białoskrzynkowego}
\Data{2012-12-22}
\Wersja{1.0}
\Autorzy{MO}
\MakeDokumentMeta


\section{Wprowadzenie}
    Testowaniu podlega system realizujący algorytm Forda Fulkersona, przystosowany do wyznaczania maksymalnego przepływu. Testowane elementy pochodzą 
z kodu źródłowego załączonego wraz z książką \emph{"Algorytmy. Almanach"}. Celem testowania jest zweryfikowanie poprawności wykonywania poszczegółnych 
funkcji oraz wartości przez nie zwracanych. Kryterium satysfakcji przeprowadzanych testów będzie wyznaczanie na podstawie współczynnika pokrycia kodu, gdzie 
jego minimalną dopuszczalną wartością jest 90\%. Część testów została dostarczona razem z systemem.  

\section{Podejście}
    Testowanie zostanie przeprowadzone techniką testowania strukturalnego. Głównym kryterium zaliczenia testów bedzie wartość pokrycia linii kodu, którego
 wartość musi przekraczać conajmniej 90\%.

    Narzędzia oraz oprogramowanie użyte w procesie testowania:
    \begin{itemize}
        \item NetBeans IDE 7.2.1
        \item Oracle JDK 1.7
        \item TestNG 6.8
        \item EMMA Code Coverege plugin 2.3
    \end{itemize}

\section{Testowane elementy}
    Lista poszczególnych klas poddawanych testowaniu oraz pokrycie kodu zapewnione przez testy zawarte wraz z kodem źródłowym:

    \begin{itemize}
        \item algs.list\\
            \begin{tabular}{@{} >{\bfseries}p{0.24\textwidth} @{\hspace{0.02\textwidth}} p{0.74\textwidth} @{}}
                \toprule
                Klasa & \bfseries{Aktualne pokrycie} \\
                \toprule
                DoubleLinkedList &
                19\%
                \midrule
                DoubleLinkedListIterator &
                59\%
                \midrule
                DoubleNode &
                27\%
                \midrule
                List &
                33\%
                \midrule
                ListIterator &
                88\%
                \midrule
                Node &
                100\%
                \bottomrule
            \end{tabular}
        \item algs.network\\
            \begin{tabular}{@{} >{\bfseries}p{0.24\textwidth} @{\hspace{0.02\textwidth}} p{0.74\textwidth} @{}}
                \toprule
                Klasa & \bfseries{Aktualne pokrycie} \\
                \toprule
                BFS_SearchList &
                100\%
                \midrule
                EdgeInfo &
                100\%
                \midrule
                FlowNetwork &
                100\%
                \midrule
                FlowNetworkAdjacencyList &
                82\%
                \midrule
                FordFulkerson &
                100\%
                \midrule
                Search &
                100\%
                \midrule
                VertexInfo &
                100\%
                \midrule
                VertexStructure &
                93\%
                \bottomrule	
            \end{tabular}
  \end{itemize} 

\section{Funkcje wyłączone z testowania}
    Lista nieużywanych metod dostarczanych przez testowane klasy:

    \begin{itemize}
        \item FlowNetworkAdjacencyList.getCost()
    \end{itemize}

\section{Zadania testowania}
    \begin{tabular}{@{} >{\bfseries}p{0.24\textwidth} @{\hspace{0.02\textwidth}} p{0.74\textwidth} @{}}
        \toprule
        Zadanie & \bfseries{Jednostka odpowidzialna} \\
        \toprule
        Projekt tsetów &
        MO
        \midrule
        Przygotowanie przypadków testowych &
        MO
        \midrule
        Implementacja testów &
        MO
        \midrule
        Uruchomienie testów i weryfikacja wyników &
        MO, TC
        \midrule
        Akceptacja wyników przebiegu testowania &
        TC
        \bottomrule
    \end{tabular}

\section{Harmonogram}
    Ustalony po otrzymaniu konkretnych deadline`ów od kierownika.

\newpage
\section*{Historia dokumentu}
\begin{versions}
    \version*{1.0}{2012-12-22}{MO}%
        Oczekuje na zatwierdzenie.
\end{versions}


\end{document}
