\subsection{Testy czarno\dywiz skrzynkowe}

\subsubsection{Wprowadzenie}
Obiekt realizowanych testów czarno\dywiz skrzynkowych nie odbiega w istotny sposób od założeń dla całego projektu. Testy czarno\dywiz skrzynkowe koncentrują się na analizie poprawności wyników działania algorytmu \textsc{Forda-Fulkersona} przy znanych danych wejściowych. %Napisać coś o zastosowanej wyroczni. Weryfikacja ręcznie nie brzmi najlepiej

\subsubsection{Testowane elementy}
Obiektem testów czarnoskrzynkowych jest klasa \texttt{FordFulkerson} z projektu \texttt{ford-fulkerson-test-suite}, zawierająca implementację algorytmu \textsc{Forda-Fulkersona}. W trakcie testów bezpośrednio wykorzystywane z projektu \texttt{ford-fulkerson-test-suite} są również klasy:

\subsubsection{Testowana funkcjonalność}
Testowana w ramach testów czarno\dywiz skrzynkowych funkcjonalność obejmuje:
\begin{itemize}[nosep]
    \item Działanie konstruktora klasy \texttt{FordFulkerson}
    \item Możliwość wykonania metody \texttt{compute()} klasy \texttt{FordFulkerson}
    \item Poprawność wyników działania algorytmu \textsc{Forda-Fulkersona}, tj porównanie wartości zwróconych przez metodę \texttt{toString()}, klasy \texttt{FordFulkerson} po wykonaniu metody \texttt{compute()} do wyroczni.
\end{itemize}

\subsubsection{Testowana funkcjonalność --- wyłączenia}
Wyłączenia z testowania funkcjonalności obejmują wyłączenia projektu, a dodatkowo:
\paragraph{Niewykorzystywane ze względu na wybór implementacji}
\begin{itemize}[nosep]
    \item Klasa \texttt{DFS\_SearchList}
    \item Klasa \texttt{FlowNetworkAdjacencyList}
    \item Klasa \texttt{ShortestPathArray}
\end{itemize}

\paragraph{Klasy abstrakcyjne, po których dziedziczą wykorzystywane przy testowaniu klasy}
\begin{itemize}[nosep]
    \item Klasa abstrakcyjna \texttt{FlowNetwork}
    \item Klasa abstrakcyjna\texttt{Search}
\end{itemize}

\paragraph{Szczegóły wykorzystywanej implementacji}
\begin{itemize}[nosep]
    \item Klasa \texttt{EdgeInfo}
    \item Klasa \texttt{DFS\_SearchArray} --- implementująca algorytm Deep\dywiz First\dywiz Search do wykorzystania na sieciach przepływowych z \texttt{FlowNetworkArray}, która jest wymagana jako drugi argument konstruktura klasy \texttt{FordFulkerson}
    \item Klasa \texttt{VertexInfo}
    \item Klasa \texttt{VertexStructure}
\end{itemize}

\paragraph{Nie wnoszące dodatkowej funkcjonalności}
\begin{itemize}[nosep]
    \item Klasa \texttt{Example} --- stanowi tylko przykładowy sposób wykorzystania algorytmu \textsc{Forda-Fulkersona}, nie jest elementem implementacji%Do Kierownika: Możesz pomyśleć nad dodaniem Example.java, do project-wide wyłączeń z testów
\end{itemize}

\subsubsection{Podejście}%%%TU PRACUJE <<USUNĄĆ PRZED COMMITEM
Podstawą do realizacji testów czarno\dywiz skrzynkowych było wyróżnienie klas abstrakcji, które pozwoliłyby możliwie solidnie przetestować funkcjonalność dostarczonej wraz z książką \emph{Algorytmy. Almanach} implementacji algorytmu \textsc{Forda-Fulkersona}, unikając jednocześnie testowania totalnego.

Zidentyfikowane zostały następujące klasy abstrakcji:

\paragraph{Ze względu na ilość węzłów}
\begin{itemize}[nosep]
    \item Ujemna ilość węzłów.
    \item Brak jakichkolwiek węzłów.
    \item Dokładnie jeden węzeł, źródłowy, pośredni lub ujście.
    \item Sieć przepływowa bez węzłów pośrednich.
    \item Sieć przepływowa z jednym węzłem pośrednim --- jako minimalna ilość węzłów pośrednich w sieci przepływowej z węzłami pośrednimi
    \item Sieć przepływowa z kilkoma węzłami pośrednimi --- jako niegraniczna ilość węzłów w sieci przepływowej z węzłami pośrednimi
    \item Sieć przepływowa z $1e9$ węzłami pośrednimi --- jako maksymalna ilość węzłów pośrednich w sieci przepływowej z węzłami pośrednimi
    \item Ilość węzłów mniejsza od zadeklarowanej.
    \item Ilość węzłów większa od zadeklarowanej.
\end{itemize}

\paragraph{Ze względu pętle}
\begin{itemize}[nosep]
    \item W źródle.
    \item W węźle pośrednim.
    \item W ujściu.
    \item Bez pętli.
\end{itemize}

\paragraph{Ze względu na krawędzie bezpośrednich między źródłem, a ujściem}
\begin{itemize}[nosep]
    \item Ze źródła do ujścia.
    \item Z ujścia do źródła.
    \item Bez krawędzi pomiędzy źródłem, a ujściem.
\end{itemize}

\paragraph{Ze względu na przepustowość krawędzi}
\begin{itemize}[nosep]
    \item Dodatnia.
    \item Zerowa.
    \item Ujemna.
\end{itemize}

\paragraph{Ze względu na poprawność krawędzi}
\begin{itemize}[nosep]
    \item Poprawne krawędzie.
    \item Krawędzie zaczynające się w nieistniejącym węźle.
    \item Krawędzie kończące się w nieistniejącym węźle.
\end{itemize}

\paragraph{Ze względu na istnienie krawędzi zwielokrotnionych}
\begin{itemize}[nosep]
    \item Bez krawędzi zwielokrotnionych.
    \item Z bezpośrednią zwielokrotnioną krawędzią pomiędzy źródłem, a ujściem.
    \item Z krawędziami zwielokrotnionymi o przeciwnych zwrotach.
\end{itemize}
