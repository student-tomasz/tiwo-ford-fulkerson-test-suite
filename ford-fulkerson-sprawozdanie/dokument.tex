\documentclass[10pt]{dokument-tiwo}


\begin{document}


\Tytul{Sprawozdanie z testowania algorytmu Forda-Fulkersona}
\Data{2012-12-21}
\Wersja{1.0}
\Autorzy{TC, MM, MO, RW}
\MakeDokumentMeta


\section{Plan testowania}
  %[RW]
  \subsection{Testy czarnoskrzynkowe}
    \subsubsection{Klasy abstrakcji}
    Na potrzeby realizacji testów czarnoskrzynkowych zostały zdefiniowane\\następujące klasy abstrakcji danych wejściowych:
    \begin{itemize}
    \item Sieć z samym źródłem, bez krawędzi.
    \item Sieć z samym źródłem i pętlą.
    \item Sieć z samym źródłem i pętlą o ujemnej przepustowości.


    \item Sieć bez węzłów pośrednich.
    \item Sieć bez węzłów pośrednich z wieloma krawędziami.
    \item Sieć bez węzłów pośrednich z pętlą.
    \item Sieć bez węzłów pośrednich z krawędzią o ujemnej przepustowości.
    \item Sieć bez węzłów pośrednich z krawędzią o zerowej przepustowości.
    \item Sieć bez węzłów pośrednich z krawędzią od ujścia do źródłą.
    \item Sieć bez węzłów pośrednich z pętlą o ujemnej przepustowości.
    \end{itemize}
  %[\RW]

\section{Projekt testów}


\section{Realizacja testów}


\section{Wykonanie testów}


\section{Ocena rezultatów testów}


\section{Wnioski}


\newpage
\section*{Historia dokumentu}
\begin{versions}
    \version*{1.0}{2012-12-21}{TC}%
        Zatwierdzono.
\end{versions}


\end{document}
