\documentclass[10pt]{dokument-tiwo}


\begin{document}


\Tytul{Sprawozdanie z testowania algorytmu \textsc{Forda-Fulkersona}}
\Data{2012-12-25}
\Wersja{0.2.2}
\Autorzy{TC, MM, MO, RW}
\MakeDokumentMeta


\section{Plan testowania}

\subsection{Wprowadzenie}
Testowaniu podlega system realizujący algorytm \textsc{Forda-Fulkersona} służący
do wyznaczania maksymalnego przepływu w sieciach. Testowane elementy pochodzą z
kodu załączonego wraz z książką \emph{Algorytmy. Almanach}. Celem testowania
jest zweryfikowanie poprawności już zaimplementowanej funkcjonalności.

\subsection{Testowane elementy}
Testowaniu podlega projekt \texttt{ford-fulkerson-test-suite} zawierający
implementację algorytmu \textsc{Forda-Fulkersona}. Testowana wersja projektu
znajduje się w repozytorium, w rewizji \texttt{e5ebef7} o nazwie \texttt
{testing-final}.

\subsection{Testowana funkcjonalność}
Uwaga procesu testowania będzie skupiona na konkretnych implementacjach systemu:
\begin{itemize}[nosep]
    \item dla klasy \texttt{FlowNetwork} --- implementacja
        \texttt{FlowNetworkArray},
    \item dla klasy \texttt{Search} --- implementacja \texttt{DFS\_SearchArray},
\end{itemize}
W chwili obecnej dostępne są też alternatywne implementacje operujące na
listach, odpowiednio \texttt{FlowNetworkAdjacencyList} oraz
\texttt{DFS\_SearchList}. Wyszczególnienie implementacji jest istotne dla
planowania testów biało\dywiz skrzynkowych. Testy czarno\dywiz skrzynkowe
nie mają wiedzy o wybranej implementacji, powinny działać niezależnie od wyboru.

\subsection{Testowana funkcjonalność --- wyłączenia}
Do zakresu testów nie wchodzą moduły pomocnicze:
\begin{itemize}[nosep]
    \item pakiet \texttt{algs.heap},
    \item pakiet \texttt{algs.list}.
\end{itemize}
Oraz moduły nie wykorzystywane przez testowany system algorytmu:
\begin{itemize}[nosep]
    \item moduł \texttt{algs.network.generator}.
\end{itemize}

\subsection{Podejście}
Środowisko testowe wymaga korzystani z:
\begin{itemize}[nosep]
    \item NetBeans IDE 7.2.1
    \item Oracle JDK 1.7\_u9
    \item TestNG 6.8
\end{itemize}

\noindent
Specyficzne metodyki, narzędzia i kryteria satysfakcji są opisane w sekcji
\ref{sec:szczegolowy_plan}.


\section{Szczegółowy plan testowania}
\label{sec:szczegolowy_plan}
\subsection{Testy czarno\dywiz skrzynkowe}

\subsubsection{Wprowadzenie}
Obiekt realizowanych testów czarno\dywiz skrzynkowych nie odbiega w istotny sposób od założeń dla całego projektu. Testy czarno\dywiz skrzynkowe koncentrują się na analizie poprawności wyników działania algorytmu \textsc{Forda-Fulkersona} przy znanych danych wejściowych. %Napisać coś o zastosowanej wyroczni. Weryfikacja ręcznie nie brzmi najlepiej 

\subsubsection{Testowane elementy}
Obiektem testów czarnoskrzynkowych jest klasa \texttt{FordFulkerson} z projektu \texttt{ford-fulkerson-test-suite}, zawierająca implementację algorytmu \textsc{Forda-Fulkersona}. W trakcie testów bezpośrednio wykorzystywane z projektu \texttt{ford-fulkerson-test-suite} są również klasy:
\begin{itemize}[nosep]
    
    
\end{itemize}

\subsubsection{Testowana funkcjonalność}
Testowana w ramach testów czarno\dywiz skrzynkowych funkcjonalność obejmuje:
\begin{itemize}[nosep]
    \item Działanie konstruktora klasy \texttt{FordFulkerson}
    \item Możliwość wykonania metody \textttt{compute()} klasy \texttt{FordFulkerson}
    \item Poprawność wyników działania algorytmu \textsc{Forda-Fulkersona}, tj porównanie wartości zwróconych przez metodę \textttt{toString()}, klasy \texttt{FordFulkerson} po wykonaniu metody \texttt{compute()} do wyroczni.
\end
\subsubsection{Testowana funkcjonalność --- wyłączenia}
Wyłączenia z testowania funkcjonalności obejmują wyłączenia projektu, a dodatkowo:
\paragraph{Niewykorzystywane ze względu na wybór implementacji}
\begin{itemize}[nosep]
    \item Klasa \texttt{DFS\_SearchList}
    \item Klasa \texttt{FlowNetworkAdjacencyList}
    \item Klasa \texttt{ShortestPathArray}
\end{itemize}

\paragraph{Klasy abstrakcyjne, po których dziedziczą wykorzystywane przy testowaniu klasy}
\begin{itemize}[nosep]
    \item Klasa abstrakcyjna \texttt{FlowNetwork}
    \item Klasa abstrakcyjna\texttt{Search}
\end{itemize}

\paragraph{Szczegóły wykorzystywanej implementacji}
\begin{itemize}[nosep]
    \item Klasa \texttt{EdgeInfo}
    \item Klasa \texttt{DFS\_SearchArray} --- implementująca algorytm Deep\dywiz First\dywiz Search do wykorzystania na sieciach przepływowych z \texttt{FlowNetworkArray}, która jest wymagana jako drugi argument konstruktura klasy \texttt{FordFulkerson}
    \item Klasa \texttt{VertexInfo}
    \item Klasa \texttt{VertexStructure}
\end{itemize}

\paragraph{Nie wnoszące dodatkowej funkcjonalności}
\begin{itemize}[nosep]
    \item Klasa \texttt{Example} --- stanowi tylko przykładowy sposób wykorzystania algorytmu \textsc{Forda-Fulkersona}, nie jest elementem implementacji
%Do Kierownika: Możesz pomyśleć nad dodaniem Example.java, do project-wide wyłączeń z testów
\end{itemize}

\subsubsection{Podejście}
Testy czarno\dywiz skrzynkowe zostaną przeprowadzone we

\subsubsection{Projekt testów}
Na potrzeby realizacji testów czarnoskrzynkowych zostały zdefiniowane
następujące klasy abstrakcji danych wejściowych opisane poniżej.

\paragraph{Sieć bez węzłów}
Sieć bez żadnych węzłów nie istnieje, nie da się dla niej wyznaczyć maksymalnego
przepływu.

\paragraph{Sieć z 1 węzłem}
Za szczególny przypadek sieci przepływu można uznać sieć składającą się z
dokładnie jednego węzła. Niezależnie od pozostałych parametrów, taka sieć
powinna zostać odrzucona jako nieprawidłowa. Wyróżnić można warianty złożone z:
\begin{itemize}[nosep]
    \item Samego źródła.
    \item Węzła pośredniego.
    \item Ujścia.
    \item Wspólnego źródła i ujścia.
    \item Wspólnego źródła i ujścia z pętlą.
\end{itemize}
Szczególną uwagę należy zwrócić na ostatni z wymienionych wariantów, który jako
jedyny posiadający krawędź, która można traktować jako krawędź ze źródła do
ujścia.

Można wyróżnić cztery schematy sieci z pojedynczym węzłem, każdy w czterech
wariantach: bez pętli, z pętlą o dodatniej/ujemnej/zerowej przepustowości, co
przekładałoby się na szesnaście przypadków testowych. W ramach planu ograniczono
się do pięciu najistotniejszych kombinacji.

\paragraph{Sieć bez węzłów pośrednich}
Sieć złożona ze źródła i ujścia, bez jakichkolwiek węzłów pośrednich. Wyróżnione
zostały następujące klasy abstrakcji dla topologii sieci z węzłami:
\begin{itemize}[nosep]
    \item Nie połączonymi żadną krawędzią.
    \item Połączonymi pojedynczą krawędzią skierowaną od źródła do ujścia o dodatniej przepustowości.
    \item Połączonymi pojedynczą krawędzią skierowaną od źródła do ujścia o dodatniej przepustowości z pętlą o dodatniej przepustowości w ujściu.
    \item Połączonymi pojedynczą krawędzią skierowaną od źródła do ujścia o dodatniej przepustowości z pętlą o ujemnej przepustowości w źródle.
    \item Połączonymi pojedynczą krawędzią skierowaną od źródła do ujścia o zerowej przepustowości.
    \item Połączonymi pojedynczą krawędzią skierowaną od źródła do ujścia o ujemnej przepustowości.
    \item Połączonymi pojedynczą krawędzią skierowaną od ujścia do źródła o dodatniej przepustowości.
    \item Połączonymi pojedynczą krawędzią skierowaną od ujścia do źródła o ujemnej przepustowości.
    \item Z wieloma krawędziami skierowanymi od żródła do ujścia.
    \item Z wieloma krawędziami skierowanymi od ujścia do źródła.
    \item Z wieloma krawędziami skierowanymi w różnych stronach.
\end{itemize}
Można bez problemu zdefiniować ponad sto wariantów topologii sieci przepływowej
bez węzłów pośrednich. Trzy ze względu na rodzaj przepustowości krawędzi
(dodatnia/ujemna/zerowa), siedem ze względu na istnienie i rodzaj pętli (brak/w
źródle z dodatnią przepustowościa/w źródle z ujemną przepustowością/w źródle z
zerową przepustowością/w ujściu z dodatnią przepustowościa/w ujściu z ujemną
przepustowościa/w ujściu z zerową przepustowościa), oraz sześć wariantów
połączeń pomiędzy krawędziami (bez krawędzi, pojedyncza od źródła do ujścia,
pojedyncza od ujścia do źródła, zwielokrotniona ze źródła do ujścia,
zwielokrotniona z ujścia do źródła, z pomieszanymi zwrotami). Do przetestowania
zostało wybranych 11 najbardziej reprezentatywnych topologii sieci bez węzłów
pośrednich.

\paragraph{Sieć z 1 węzłem pośrednim}
Prosta sieć przepływu zbudowana ze źródła, jednego węzła pośredniego i ujścia.
Wyróżnione zostały następujące klasy abstrakcji w zależności od topologii:
\begin{itemize}[nosep]
    \item Połączone pojedynczymi krawędziami skierowanymi ze źródła do węzłą
    pośredniego i z węzła pośredniego do ujścia.
    % Z -> P -> U
    \item Z pojedynczymi krawędziami skierowanymi z ujścia do węzła pośredniego
    i z węzła pośredniego do źródła.
    % Z <- P <- U
    \item Z pojedynczymi krawędziami skierowanymi z węzła pośredniego do ujścia
    i z węzła pośredniego do źródła.
    % Z <- P -> U
    \item Z pojedynczymi krawędziami skierowanymi ze źródła do węzła pośredniego
    i z ujścia do węzła pośredniego.
    % Z -> P <- U
    \item Połączone pojedynczymi krawędziami skierowanymi ze źródła do węzła
    pośredniego i z węzła pośredniego do źródła z dodatkową krawędzią o
    dodatniej przepustowości ze źródła do ujścia.
    % Z -> P -> U
    % |_________^
    \item Połączone pojedynczymi krawędziami skierowanymi ze źródła do węzła
    pośredniego i z węzła pośredniego do źródła z pętlą o dodatniej
    przepustowości w węźle pośrednim.
    % Z -> P -> U
    %     / ^
    %     \_|
    \item Połączone zwielokrotnionymi krawędziami ze źródła do węzła pośredniego
    i z węzła pośredniego do źródła, z mieszanymi zwrotami.
    % Z -> P -> U
    %  ^__/ ^__/
    \item Połączone pojedynczą krawędzią skierowaną ze źródła do węzła
    pośredniego, bez krawędzi do ujścia.
    % Z -> P    U
    \item Połączone pojedynczą krawędzia skierowaną ze źródła do węzła
    pośredniego, oraz krawędzią ze źródłą do ujścia.
    % Z -> P    U
    % |_________^
    \item Połączone pojedynczą krawędzia skierowaną z węzła pośredniego do
    ujścia, bez połączenia ze źródłem.
    % Z    P -> U
\end{itemize}
W przypadku sieci z jednym węzłem pośrednim można wyróżnić nawet ponad tysiąc
różnych rodzajów topologii sieci. Dla dwóch krawędzi istnieje już dziewięć
kombinacji rodzajów przepustowości (dodatnia/ujemna/zerowa), trzy możliwe
położenia pętli (w źródle/węźle pośrednim/ujściu), trzy rodzaje przepustowości w
pętli (dodatni/ujemny/zerowy), cztery możliwe kombinacje zwrotów krawędzi, oraz
cztery warianty zwielokrotnienia krawędzi w topologii sieci. Doliczenie się
takiej liczby możliwych rodzajów topologii sieci, nie wymagało nawet
uwzględnienia możliwości istnienia krawędzi bezpośredniej ze źródła do ujścia,
czy braku którejś krawędzi. Ostatecznie zdecydowano się zatem na dziesięć
różnych klas abstrakcji.

\paragraph{Sieć z 2 węzłami pośrednimi połączonymi równolegle}
W tej sekcji jako warte przetestowania wyróżnioniono następujące schematy sieci
z 2 węzłami pośrednimi połączonymi równolegle:
\begin{itemize}[nosep]
    \item Z krawędziami o dodatniej przepustowości ze źródła do obu węzłów
    pośrednich i z obu węzłów pośrednich do ujścia.
    \item Z dodatkową krawędzią o dodatniej przepustowości z pierwszego węzła
    pośredniego do drugiego węzła pośredniego.
    \item Z dodatkową krawędzią o dodatniej przepustowości z drugiego węzła
    pośredniego do pierwszego węzła pośredniego.
    \item Z dodatkowymi krawędziami o dodatniej przepustowości z pierwszego
    węzła pośredniego do drugiego węzła pośredniego i z drugiego węzła
    pośredniego do pierwszego węzła pośredniego.\footnote{Potencjalne zagrożenie
    stworzeniem złożonej pętli.}
    \item Z dodatkową krawędzią o dodatniej przepustowości ze źródła do ujścia.
    \item Z dodatkową krawędzia o dodatniej przepustowości z ujścia do źródła.
    \item Z dodatkową krawędzią o dodatniej przepustowości z ujścia do
    pierwszego węzła pośredniego.
    \item Z dodatkową krawędzią o dodatniej przepustowości z pierwszego węzła
    pośredniego do źródłą.
    \item Z dodatkowymi krawędziami o dodatniej przepustowości z ujścia do
    pierwszego węzła pośredniego i z pierwszego węzłą pośredniego do źródła.
\end{itemize}
Klasy abstrakcji dla sieci z 2 węzłami pośrednimi zostały pomyślane przede
wszystkim dla przetestowania zachowania oprogramowania dla różnych schematów
połączeń, stąd chociażby brak w tej sekcji klas abstrakcji z pętlami.

\paragraph{Sieć z 2 węzłami pośrednimi połączonymi szeregowo}
W tym wypadku jako istotne dla testów klasy abstrakcji zdefiniowano topologie
sieci z 2 węzłami pośrednimi połączonymi szeregowo:
\begin{itemize}[nosep]
    \item Z krawędziami o dodatniej przepustowości ze źródła do pierwszego węzła
    pośredniej, z pierwszego węzła pośredniego do drugiego węzła pośredniego i z
    drugiego węzła pośredniego do ujścia.
    \item Sieć z 2 węzłami pośrednimi połączonymi szeregowo z wieloma
    krawędziami pomiędzy węzłami pośrednimi.
    \item Sieć z 2 węzłami pośrednimi połączonymi szeregowo z krawędzią o
    zerowej przepustowości pomiędzy węzłami pośrednimi.
    \item Sieć z 2 węzłami pośrednimi połączonymi szeregowo z krawędzią o
    ujemnej przepustowości pomiędzy węzłami pośrednimi.
    \item Sieć z 2 węzłami pośrednimi połączonymi szeregowo z krawędzią z
    drugiego węzłą pośredniego do pierwszego węzła pośredniego.\footnote{Ujemna
    przepustowość pomiędzy pierwszym a drugim węzłem pośrednim nie jest tożsama
    z krawędzią o dodatniej przepustowości od drugiego do pierwszego węzła
    pośredniego. Ujemna przepustowość krawędzi jest niepoprawna z definicji,
    natomiast krawędź skierowana w przeciwną stronę jest jak najbardziej
    poprawna z punktu widzenia definicji, choć taka sieć wciąż nie miałaby
    żadnej przepustowości.}
    \item Sieć z 2 węzłami pośrednimi połączonymi szeregowo z krawędzią od
    pierwszego węzła pośredniego do ujścia.
    \item Sieć z 2 węzłami pośrednimi połączonymi szeregowo z krawędzią od
    źródła do drugiego węzła pośredniego.
\end{itemize}
Dodanie schematów z połączeniami szeregowymi zasadniczo wyczerpuje możliwości
budowy schematów połączeń, każdy bardziej złożony (prawidłowy) schemat połączeń
będzie bazował na pewnej kombinacji powyższych wariacji na temat połączeń
szeregowych i równoległych.

\paragraph{Sieć z nieistniejącymi węzłami}
Liczba węzłów określona w danych wejściowych jest większa niż liczba
zdefiniowanych węzłów.

\paragraph{Sieć z ujemną liczbą węzłów}
W danych wejściowych liczba węzłów w sieci jest określona za pomocą liczby
ujemnej.

\paragraph{Sieć z $1e9$ węzłów}
Test dla bardzo dużej (najlepiej granicznej) wartości liczby węzłów.

\subsection{Testy biało\dywiz skrzynkowe}

\subsubsection{\texttt algs.netvork.VertexStructure}
\begin{center}
\begin{tabular}{@{} >{\ttfamily}p{0.2\textwidth} @{\hspace{0.02\textwidth}} p{0.6\textwidth} @{}}
    \toprule
    \multicolumn{2}{@{}c@{}}{\bfseries{ToStringTest}} \\
    \midrule
    {\bfseries Funkcja} & \bfseries String \texttt{toString()} \\
    \hline
    {\bfseries Opis} & Wypisuje listę węzłów wychodzących i wchodzacych. \\
    \hline
    {\bfseries Dane wejściowe} & {\begin{enumerate} 
                                        \item Struktura bez zdefiniowanych krawędzi wejscia/wyjścia 
                                        \item Krawędź wyjścia (1, 2, 1); krawędź wejścia (2, 1, 1)
                                        \item Trzy krawędzie  wejścia (0, 3, 1), (1, 3, 1), (2, 3, 1); trzy krawędzie wyjścia (3, 4, 1), (3, 5, 1) , (3, 6, 1)
                                    \end{enumerate}} \\
    \hline
    {\bfseries Oczekiwany wynik} & {\begin{enumerate} 
                                        \item Struktura bez zdefiniowanych krawędzi wejscia/wyjścia 
                                        \item Krawędź wyjścia (1, 2, 1); krawędź wejścia (2, 1, 1)
                                        \item Trzy krawędzie  wejścia (0, 3, 1), (1, 3, 1), (2, 3, 1); trzy krawędzie wyjścia (3, 4, 1), (3, 5, 1) , (3, 6, 1)
                                    \end{enumerate}} \\
    \hline
    {\bfseries Wykonawca} & MO \\
    \bottomrule
\end{tabular}
\end{center}

\subsubsection{\texttt algs.network.FlowNetworkArray}
\begin{center}
\begin{tabular}{@{} >{\ttfamily}p{0.2\textwidth} @{\hspace{0.02\textwidth}} p{0.6\textwidth} @{}}
    \toprule
    \multicolumn{2}{@{}c@{}}{\bfseries{MinimalNetworkConstructorTest}} \\ 
    \midrule
    {\bfseries Id} & B.2.1 \\
    \hline
    {\bfseries Funkcja} & \texttt{FlowNetworkArray (\bfseries int sourceIndex, \bfseries int sinkIndex, \bfseries int numVertices)} \\
    \hline
    {\bfseries Opis} & Konstruktor minimalnej struktury sieci. Inicjalizuje tylko niezbedne zmienne. \\
    \hline
    {\bfseries Dane wejściowe} & {\begin{enumerate} 
                                        \item sourceIndex = 0; sinkIndex = 1; numVertices = 2
                                        \item sourceIndex = -1; sinkIndex = -2; numVertices = -2
                                        \item sourceIndex = 0; sinkIndex = 1; numVertices = 2
                                        \item sourceIndex = 0; sinkIndex = 7; numVertices = 2
                                    \end{enumerate}} \\
    \hline
    {\bfseries Oczekiwany wynik} & {\begin{enumerate} 
                                        \item sourceIndex = 0; sinkIndex = 1; numVertices = 2
                                        \item IllegalArgumentException
                                        \item IllegalArgumentException
                                        \item IllegalArgumentException
                                    \end{enumerate}} \\
    \hline
    {\bfseries Wykonawca} & MO \\
    \bottomrule
\end{tabular}
\end{center}

\begin{center}
\begin{tabular}{@{} >{\ttfamily}p{0.2\textwidth} @{\hspace{0.02\textwidth}} p{0.6\textwidth} @{}}
    \toprule
    \multicolumn{2}{@{}c@{}}{\bfseries{NetworkConstructorTest}} \\ 
    \midrule
    {\bfseries Id} & B.2.1 \\
    \hline
    {\bfseries Funkcja} & \texttt{FlowNetworkArray (\bfseries int numVertices, \bfseries int sourceIndex, 
                                                    \bfseries int sinkIndex,
                                                    \bfseries Iterator<Edges> edges} \\
    \hline
    {\bfseries Opis} & Konstruktor struktury reprezentującej graf przpływu. \\
    \hline
    {\bfseries Dane wejściowe} & {\begin{enumerate} 
                                        \item numVertices = 4; sourceIndex = 0; sinkIndex = 3;  edges =  (0, 1, 3);(1, 2, 2); (2, 3, 2); (0, 2, 3)
                                        \item numVertices = -4; sourceIndex = 0; sinkIndex = 3; edges =  (0, 1, 3);(1, 2, 2); (2, 3, 2); (0, 2, 3)
                                        \item numVertices = 4; sourceIndex = -1; sinkIndex = 3; edges =  (0, 1, 3);(1, 2, 2); (2, 3, 2); (0, 2, 3)
                                        \item numVertices = 4; sourceIndex = 0; sinkIndex = -3; edges =  (0, 1, 3);(1, 2, 2); (2, 3, 2); (0, 2, 3)
                                        \item numVertices = 2; sourceIndex = 0; sinkIndex = 3; edges =  (0, 1, 3);(1, 2, 2); (2, 3, 2); (0, 2, 3)
                                        \item numVertices = 4; sourceIndex = 3; sinkIndex = 0; edges =  (0, 1, 3);(1, 2, 2); (2, 3, 2); (0, 2, 3)
                                        \item numVertices = 0; sourceIndex = 1; sinkIndex = 2; edges =  pusty Iterator
                                    \end{enumerate}} \\
    \hline
    {\bfseries Oczekiwany wynik} & {\begin{enumerate} 
                                        \item numVertices = 4; sourceIndex = 0; sinkIndex = 3;  edges =  (0, 1, 3);(1, 2, 2); (2, 3, 2); (0, 2, 3)
                                        \item IllegalArgumentException
                                        \item IllegalArgumentException
                                        \item IllegalArgumentException
                                        \item IllegalArgumentException
                                        \item IllegalArgumentException
                                        \item numVertices = 4; sourceIndex = 0; sinkIndex = 3;  edges =  pusta tablica EdgeInfo[4][4]
                                    \end{enumerate}} \\
    \hline
    {\bfseries Wykonawca} & MO \\
    \bottomrule
\end{tabular}
\end{center}

\begin{center}
\begin{tabular}{@{} >{\ttfamily}p{0.2\textwidth} @{\hspace{0.02\textwidth}} p{0.6\textwidth} @{}}
    \toprule
    \multicolumn{2}{@{}c@{}}{\bfseries{IllegalStateExceptionTest}} \\
    \midrule
    {\bfseries Id} & B.2.2 \\
    \hline
    {\bfseries Funkcja} & \bfseries void \texttt{ validate()} \\
    \hline
    {\bfseries Opis} & Metoda weryfikuje czy informacje na temat sieci są akceptowalne. 
                       Zwracany jest wyjątek, IllegalStateException w dwóch przypadkach:
        \begin{enumerate}
            \item Przepływ krawędzi jest wiekszy niż przepustowaość przepustowaść
            \item Ilość krawędzi wchodzących jest różna od krawędzi wychodzących
        \end{enumerate}\\
    \hline
    {\bfseries Dane wejściowe} & {\begin{enumerate} 
                                        \item Krawędź (1, 2, 1), wymuszony flow = 2
                                        \item Krawędź (1, 2, 1), wymuszony flow = 1
                                    \end{enumerate}} \\
    \hline
    {\bfseries Oczekiwany wynik} & {\begin{enumerate} 
                                        \item IllegalStateException
                                        \item IllegalStateException
                                    \end{enumerate}} \\
    \hline
    {\bfseries Wykonawca} & MO \\
    \bottomrule
\end{tabular}
\end{center}



\section{Projekt testów}
\label{sec:projekt}
\subsection{Testy czarno\dywiz skrzynkowe}

\subsubsection{Wprowadzenie}
Obiekt realizowanych testów czarno\dywiz skrzynkowych nie odbiega w istotny sposób od założeń dla całego projektu. Testy czarno\dywiz skrzynkowe koncentrują się na analizie poprawności wyników działania algorytmu \textsc{Forda-Fulkersona} przy znanych danych wejściowych. %Napisać coś o zastosowanej wyroczni. Weryfikacja ręcznie nie brzmi najlepiej 

\subsubsection{Testowane elementy}
Obiektem testów czarnoskrzynkowych jest klasa \texttt{FordFulkerson} z projektu \texttt{ford-fulkerson-test-suite}, zawierająca implementację algorytmu \textsc{Forda-Fulkersona}. W trakcie testów bezpośrednio wykorzystywane z projektu \texttt{ford-fulkerson-test-suite} są również klasy:
\begin{itemize}[nosep]
    
    
\end{itemize}

\subsubsection{Testowana funkcjonalność}
Testowana w ramach testów czarno\dywiz skrzynkowych funkcjonalność obejmuje:
\begin{itemize}[nosep]
    \item Działanie konstruktora klasy \texttt{FordFulkerson}
    \item Możliwość wykonania metody \textttt{compute()} klasy \texttt{FordFulkerson}
    \item Poprawność wyników działania algorytmu \textsc{Forda-Fulkersona}, tj porównanie wartości zwróconych przez metodę \textttt{toString()}, klasy \texttt{FordFulkerson} po wykonaniu metody \texttt{compute()} do wyroczni.
\end
\subsubsection{Testowana funkcjonalność --- wyłączenia}
Wyłączenia z testowania funkcjonalności obejmują wyłączenia projektu, a dodatkowo:
\paragraph{Niewykorzystywane ze względu na wybór implementacji}
\begin{itemize}[nosep]
    \item Klasa \texttt{DFS\_SearchList}
    \item Klasa \texttt{FlowNetworkAdjacencyList}
    \item Klasa \texttt{ShortestPathArray}
\end{itemize}

\paragraph{Klasy abstrakcyjne, po których dziedziczą wykorzystywane przy testowaniu klasy}
\begin{itemize}[nosep]
    \item Klasa abstrakcyjna \texttt{FlowNetwork}
    \item Klasa abstrakcyjna\texttt{Search}
\end{itemize}

\paragraph{Szczegóły wykorzystywanej implementacji}
\begin{itemize}[nosep]
    \item Klasa \texttt{EdgeInfo}
    \item Klasa \texttt{DFS\_SearchArray} --- implementująca algorytm Deep\dywiz First\dywiz Search do wykorzystania na sieciach przepływowych z \texttt{FlowNetworkArray}, która jest wymagana jako drugi argument konstruktura klasy \texttt{FordFulkerson}
    \item Klasa \texttt{VertexInfo}
    \item Klasa \texttt{VertexStructure}
\end{itemize}

\paragraph{Nie wnoszące dodatkowej funkcjonalności}
\begin{itemize}[nosep]
    \item Klasa \texttt{Example} --- stanowi tylko przykładowy sposób wykorzystania algorytmu \textsc{Forda-Fulkersona}, nie jest elementem implementacji
%Do Kierownika: Możesz pomyśleć nad dodaniem Example.java, do project-wide wyłączeń z testów
\end{itemize}

\subsubsection{Podejście}
Testy czarno\dywiz skrzynkowe zostaną przeprowadzone we

\subsubsection{Projekt testów}
Na potrzeby realizacji testów czarnoskrzynkowych zostały zdefiniowane
następujące klasy abstrakcji danych wejściowych opisane poniżej.

\paragraph{Sieć bez węzłów}
Sieć bez żadnych węzłów nie istnieje, nie da się dla niej wyznaczyć maksymalnego
przepływu.

\paragraph{Sieć z 1 węzłem}
Za szczególny przypadek sieci przepływu można uznać sieć składającą się z
dokładnie jednego węzła. Niezależnie od pozostałych parametrów, taka sieć
powinna zostać odrzucona jako nieprawidłowa. Wyróżnić można warianty złożone z:
\begin{itemize}[nosep]
    \item Samego źródła.
    \item Węzła pośredniego.
    \item Ujścia.
    \item Wspólnego źródła i ujścia.
    \item Wspólnego źródła i ujścia z pętlą.
\end{itemize}
Szczególną uwagę należy zwrócić na ostatni z wymienionych wariantów, który jako
jedyny posiadający krawędź, która można traktować jako krawędź ze źródła do
ujścia.

Można wyróżnić cztery schematy sieci z pojedynczym węzłem, każdy w czterech
wariantach: bez pętli, z pętlą o dodatniej/ujemnej/zerowej przepustowości, co
przekładałoby się na szesnaście przypadków testowych. W ramach planu ograniczono
się do pięciu najistotniejszych kombinacji.

\paragraph{Sieć bez węzłów pośrednich}
Sieć złożona ze źródła i ujścia, bez jakichkolwiek węzłów pośrednich. Wyróżnione
zostały następujące klasy abstrakcji dla topologii sieci z węzłami:
\begin{itemize}[nosep]
    \item Nie połączonymi żadną krawędzią.
    \item Połączonymi pojedynczą krawędzią skierowaną od źródła do ujścia o dodatniej przepustowości.
    \item Połączonymi pojedynczą krawędzią skierowaną od źródła do ujścia o dodatniej przepustowości z pętlą o dodatniej przepustowości w ujściu.
    \item Połączonymi pojedynczą krawędzią skierowaną od źródła do ujścia o dodatniej przepustowości z pętlą o ujemnej przepustowości w źródle.
    \item Połączonymi pojedynczą krawędzią skierowaną od źródła do ujścia o zerowej przepustowości.
    \item Połączonymi pojedynczą krawędzią skierowaną od źródła do ujścia o ujemnej przepustowości.
    \item Połączonymi pojedynczą krawędzią skierowaną od ujścia do źródła o dodatniej przepustowości.
    \item Połączonymi pojedynczą krawędzią skierowaną od ujścia do źródła o ujemnej przepustowości.
    \item Z wieloma krawędziami skierowanymi od żródła do ujścia.
    \item Z wieloma krawędziami skierowanymi od ujścia do źródła.
    \item Z wieloma krawędziami skierowanymi w różnych stronach.
\end{itemize}
Można bez problemu zdefiniować ponad sto wariantów topologii sieci przepływowej
bez węzłów pośrednich. Trzy ze względu na rodzaj przepustowości krawędzi
(dodatnia/ujemna/zerowa), siedem ze względu na istnienie i rodzaj pętli (brak/w
źródle z dodatnią przepustowościa/w źródle z ujemną przepustowością/w źródle z
zerową przepustowością/w ujściu z dodatnią przepustowościa/w ujściu z ujemną
przepustowościa/w ujściu z zerową przepustowościa), oraz sześć wariantów
połączeń pomiędzy krawędziami (bez krawędzi, pojedyncza od źródła do ujścia,
pojedyncza od ujścia do źródła, zwielokrotniona ze źródła do ujścia,
zwielokrotniona z ujścia do źródła, z pomieszanymi zwrotami). Do przetestowania
zostało wybranych 11 najbardziej reprezentatywnych topologii sieci bez węzłów
pośrednich.

\paragraph{Sieć z 1 węzłem pośrednim}
Prosta sieć przepływu zbudowana ze źródła, jednego węzła pośredniego i ujścia.
Wyróżnione zostały następujące klasy abstrakcji w zależności od topologii:
\begin{itemize}[nosep]
    \item Połączone pojedynczymi krawędziami skierowanymi ze źródła do węzłą
    pośredniego i z węzła pośredniego do ujścia.
    % Z -> P -> U
    \item Z pojedynczymi krawędziami skierowanymi z ujścia do węzła pośredniego
    i z węzła pośredniego do źródła.
    % Z <- P <- U
    \item Z pojedynczymi krawędziami skierowanymi z węzła pośredniego do ujścia
    i z węzła pośredniego do źródła.
    % Z <- P -> U
    \item Z pojedynczymi krawędziami skierowanymi ze źródła do węzła pośredniego
    i z ujścia do węzła pośredniego.
    % Z -> P <- U
    \item Połączone pojedynczymi krawędziami skierowanymi ze źródła do węzła
    pośredniego i z węzła pośredniego do źródła z dodatkową krawędzią o
    dodatniej przepustowości ze źródła do ujścia.
    % Z -> P -> U
    % |_________^
    \item Połączone pojedynczymi krawędziami skierowanymi ze źródła do węzła
    pośredniego i z węzła pośredniego do źródła z pętlą o dodatniej
    przepustowości w węźle pośrednim.
    % Z -> P -> U
    %     / ^
    %     \_|
    \item Połączone zwielokrotnionymi krawędziami ze źródła do węzła pośredniego
    i z węzła pośredniego do źródła, z mieszanymi zwrotami.
    % Z -> P -> U
    %  ^__/ ^__/
    \item Połączone pojedynczą krawędzią skierowaną ze źródła do węzła
    pośredniego, bez krawędzi do ujścia.
    % Z -> P    U
    \item Połączone pojedynczą krawędzia skierowaną ze źródła do węzła
    pośredniego, oraz krawędzią ze źródłą do ujścia.
    % Z -> P    U
    % |_________^
    \item Połączone pojedynczą krawędzia skierowaną z węzła pośredniego do
    ujścia, bez połączenia ze źródłem.
    % Z    P -> U
\end{itemize}
W przypadku sieci z jednym węzłem pośrednim można wyróżnić nawet ponad tysiąc
różnych rodzajów topologii sieci. Dla dwóch krawędzi istnieje już dziewięć
kombinacji rodzajów przepustowości (dodatnia/ujemna/zerowa), trzy możliwe
położenia pętli (w źródle/węźle pośrednim/ujściu), trzy rodzaje przepustowości w
pętli (dodatni/ujemny/zerowy), cztery możliwe kombinacje zwrotów krawędzi, oraz
cztery warianty zwielokrotnienia krawędzi w topologii sieci. Doliczenie się
takiej liczby możliwych rodzajów topologii sieci, nie wymagało nawet
uwzględnienia możliwości istnienia krawędzi bezpośredniej ze źródła do ujścia,
czy braku którejś krawędzi. Ostatecznie zdecydowano się zatem na dziesięć
różnych klas abstrakcji.

\paragraph{Sieć z 2 węzłami pośrednimi połączonymi równolegle}
W tej sekcji jako warte przetestowania wyróżnioniono następujące schematy sieci
z 2 węzłami pośrednimi połączonymi równolegle:
\begin{itemize}[nosep]
    \item Z krawędziami o dodatniej przepustowości ze źródła do obu węzłów
    pośrednich i z obu węzłów pośrednich do ujścia.
    \item Z dodatkową krawędzią o dodatniej przepustowości z pierwszego węzła
    pośredniego do drugiego węzła pośredniego.
    \item Z dodatkową krawędzią o dodatniej przepustowości z drugiego węzła
    pośredniego do pierwszego węzła pośredniego.
    \item Z dodatkowymi krawędziami o dodatniej przepustowości z pierwszego
    węzła pośredniego do drugiego węzła pośredniego i z drugiego węzła
    pośredniego do pierwszego węzła pośredniego.\footnote{Potencjalne zagrożenie
    stworzeniem złożonej pętli.}
    \item Z dodatkową krawędzią o dodatniej przepustowości ze źródła do ujścia.
    \item Z dodatkową krawędzia o dodatniej przepustowości z ujścia do źródła.
    \item Z dodatkową krawędzią o dodatniej przepustowości z ujścia do
    pierwszego węzła pośredniego.
    \item Z dodatkową krawędzią o dodatniej przepustowości z pierwszego węzła
    pośredniego do źródłą.
    \item Z dodatkowymi krawędziami o dodatniej przepustowości z ujścia do
    pierwszego węzła pośredniego i z pierwszego węzłą pośredniego do źródła.
\end{itemize}
Klasy abstrakcji dla sieci z 2 węzłami pośrednimi zostały pomyślane przede
wszystkim dla przetestowania zachowania oprogramowania dla różnych schematów
połączeń, stąd chociażby brak w tej sekcji klas abstrakcji z pętlami.

\paragraph{Sieć z 2 węzłami pośrednimi połączonymi szeregowo}
W tym wypadku jako istotne dla testów klasy abstrakcji zdefiniowano topologie
sieci z 2 węzłami pośrednimi połączonymi szeregowo:
\begin{itemize}[nosep]
    \item Z krawędziami o dodatniej przepustowości ze źródła do pierwszego węzła
    pośredniej, z pierwszego węzła pośredniego do drugiego węzła pośredniego i z
    drugiego węzła pośredniego do ujścia.
    \item Sieć z 2 węzłami pośrednimi połączonymi szeregowo z wieloma
    krawędziami pomiędzy węzłami pośrednimi.
    \item Sieć z 2 węzłami pośrednimi połączonymi szeregowo z krawędzią o
    zerowej przepustowości pomiędzy węzłami pośrednimi.
    \item Sieć z 2 węzłami pośrednimi połączonymi szeregowo z krawędzią o
    ujemnej przepustowości pomiędzy węzłami pośrednimi.
    \item Sieć z 2 węzłami pośrednimi połączonymi szeregowo z krawędzią z
    drugiego węzłą pośredniego do pierwszego węzła pośredniego.\footnote{Ujemna
    przepustowość pomiędzy pierwszym a drugim węzłem pośrednim nie jest tożsama
    z krawędzią o dodatniej przepustowości od drugiego do pierwszego węzła
    pośredniego. Ujemna przepustowość krawędzi jest niepoprawna z definicji,
    natomiast krawędź skierowana w przeciwną stronę jest jak najbardziej
    poprawna z punktu widzenia definicji, choć taka sieć wciąż nie miałaby
    żadnej przepustowości.}
    \item Sieć z 2 węzłami pośrednimi połączonymi szeregowo z krawędzią od
    pierwszego węzła pośredniego do ujścia.
    \item Sieć z 2 węzłami pośrednimi połączonymi szeregowo z krawędzią od
    źródła do drugiego węzła pośredniego.
\end{itemize}
Dodanie schematów z połączeniami szeregowymi zasadniczo wyczerpuje możliwości
budowy schematów połączeń, każdy bardziej złożony (prawidłowy) schemat połączeń
będzie bazował na pewnej kombinacji powyższych wariacji na temat połączeń
szeregowych i równoległych.

\paragraph{Sieć z nieistniejącymi węzłami}
Liczba węzłów określona w danych wejściowych jest większa niż liczba
zdefiniowanych węzłów.

\paragraph{Sieć z ujemną liczbą węzłów}
W danych wejściowych liczba węzłów w sieci jest określona za pomocą liczby
ujemnej.

\paragraph{Sieć z $1e9$ węzłów}
Test dla bardzo dużej (najlepiej granicznej) wartości liczby węzłów.



\section{Realizacja testów}


\section{Wykonanie testów}


\section{Ocena rezultatów testów}


\section{Wnioski}


\newpage
\section*{Historia dokumentu}
\begin{versions}
    \version*{0.1.0}{2012-12-21}{TC}%
        Szablon dokumentu.
    \version{0.2.0}{2012-12-22}{RW}
        Dodanie planu testowania biało-skrzynkowego.
    \version{0.2.1}{2012-12-22}{MO}
        Dodanie planu testowania czarno-skrzynkowego.
    \version{0.2.2}{2012-12-25}{TC}
        Połączenie planów. Dodanie planu ogólnego. Rozróżnienie na plany
        szczegółowe.
\end{versions}


\end{document}
