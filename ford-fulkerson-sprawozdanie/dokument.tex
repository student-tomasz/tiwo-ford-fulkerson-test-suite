\documentclass[10pt]{dokument-tiwo}


\begin{document}


\Tytul{Sprawozdanie z testowania algorytmu Forda-Fulkersona}
\Data{2012-12-21}
\Wersja{1.0}
\Autorzy{TC, MM, MO, RW}
\MakeDokumentMeta


\section{Plan testowania}
\subsection{Testy czarnoskrzynkowe}
\subsubsection{Klasy abstrakcji}
Na potrzeby realizacji testów czarnoskrzynkowych zostały zdefiniowane następujące klasy abstrakcji danych wejściowych:
\subsubsubsection{Sieć bez węzłów}
Sieć bez żadnych węzłów nie istnieje, nie da się dla niej wyznaczyć maksymalnego przepływu.
\subsubsubsection{Sieć z 1 węzłem}
Za szczególny przypadek sieci przepływu można uznać sieć składającą się z dokładnie jednego węzła. Niezależnie od pozostałych parametrów, taka sieć powinna zostać odrzucona jako nieprawidłowa. Wyróżnić można warianty złożone z:

\begin{itemize}
    \item Samego źródła.
    \item Węzła pośredniego.
    \item Ujścia.
    \item Wspólnego źródła i ujścia.
    \item Wspólnego źródła i ujścia z pętlą.
\end{itemize}
Szczególną uwagę należy zwrócić na ostatni z wymienionych wariantów, jako jedyny posiadający krawędź, która można traktować jako krawędź ze źródła do ujścia.

Można wyróżnić cztery schematy sieci z pojedynczym węzłem, każdy w czterech wariantach: bez pętli, z pętlą o dodatniej/ujemnej/zerowej przepustowości, co przekładałoby się na szesnaście przypadków testowych. W ramach planu ograniczono się do pięciu najistotniejszych kombinacji.

\subsubsubsection{Sieć bez węzłów pośrednich}
Sieć złożona ze źródłą i ujścia, bez jakichkolwiek węzłów pośrednich. Wyróżnione zostały następujące klasy abstrakcji:
\begin{itemize}
    \item Nie połączone żadną krawędzią.
    \item Połączone pojedynczą krawędzią skierowaną od źródła do ujścia o dodatniej przepustowości.
    \item Połączone pojedynczą krawędzia skierowaną od źródła do ujścia o dodatniej przepustowości z pętlą w ujściu.
    \item Połączone pojedynczą krawędzia skierowaną od źródła do ujścia o dodatniej przepustowości z pętlą o ujemnej przepustowości w źródle.
    \item Połączone pojedynczą krawędzią skierowaną od źródła do ujścia o zerowej przepustowości.
    \item Połączone pojedynczą krawędzią skierowaną od źródła do ujścia o ujemnej przepustowości.
    \item Połączone pojedynczą krawędzią skierowaną od ujścia do źródła o dodatniej przepustowości.
    \item Połączone pojedynczą krawędzią skierowaną od ujścia do źródła o ujemnej przepustowości.
    \item Z wieloma krawędziami skierowanymi od żródła do ujścia.
    \item Z wieloma krawędziami skierowanymi od ujścia do źródła.
    \item Z wieloma krawędziami skierowanymi w różnych stronach.
\end{itemize}
Można bez problemu zdefiniować ponad 100 wariantów topologii sieci przepływowej bez węzłów pośrednich. Trzy ze względu na rodzaj przepustowości krawędzi (dodatnia/ujemna/zerowa), siedem ze względu na istnienie i rodzaj pętli (brak/w źródle z dodatnią przepustowościa/w źródle z ujemną przepustowością/w źródle z zerową przepustowością/w ujściu z dodatnią przepustowościa/w ujściu z ujemną przepustowościa/w ujściu z zerową przepustowościa), oraz sześć wariantów połączeń pomiędzy krawędziami (bez krawędzi, pojedyncza od źródła do ujścia, pojedyncza od ujścia do źródła, zwielokrotniona ze źródła do ujścia, zwielokrotniona z ujścia do źródła, z pomieszanymi zwrotami). Do przetestowania zostało wybranych 11 najbardziej reprezentatywnych.

\subsubsubsection{Sieć z 1 węzłem pośrednim}
Prosta sieć przepływu zbudowana ze źródła, jednego węzła pośredniego i ujścia. Wyróżnione zostały następujące klasy abstrakcji:
\begin{itemize}
    \item Połączone pojedynczymi krawędziami skierowanymi ze źródła do węzłą pośredniego i z węzła pośredniego do źródła.
% Z -> P -> U
    \item Z pojedynczymi krawędziami skierowanymi z ujścia do węzła pośredniego i z węzła pośredniego do źródła.
% Z <- P <- U
    \item Z pojedynczymi krawędziami skierowanymi z węzła pośredniego do ujścia i z węzła pośredniego do źródła.
% Z <- P -> U
    \item Z pojedynczymi krawędziami skierowanymi ze źródła do węzła pośredniego i z ujścia do węzła pośredniego.
% Z -> P <- U
    \item Połączone pojedynczymi krawędziami skierowanymi ze źródła do węzłą pośredniego i z węzła pośredniego do źródła z dodatkową krawędzią o dodatniej przepustowości ze źródła do ujścia.
% Z -> P -> U
% |_________^
    \item Połączone pojedynczymi krawędziami skierowanymi ze źródła do węzłą pośredniego i z węzła pośredniego do źródła z pętlą o dodatniej przepustowości w węźle pośrednim.
% Z -> P -> U
%     / ^
%     \_|
    \item Połączone zwielokrotnionymi krawędziami ze źródła do węzła pośredniego i z węzła pośredniego do źródła, z mieszanymi zwrotami.
% Z -> P -> U
%  ^__/ ^__/
    \item Połączone pojedynczą krawędzią skierowaną ze źródła do węzła pośredniego, bez krawędzi do ujścia.
% Z -> P    U
    \item Połączone pojedynczą krawędzia skierowaną ze źródła do węzła pośredniego, oraz krawędzią ze źródłą do ujścia.
% Z -> P    U
% |_________^
    \item Połączone pojedynczą krawędzia skierowaną z węzła pośredniego do ujścia, bez połączenia ze źródłem.
% Z    P -> U

\end{itemize}
W przypadku sieci z jednym węzłem pośrednim można wyróżnić nawet ponad tysiąc różnych rodzajów topologii sieci. Dla dwóch krawędzi istnieje już dziewięć kombinacji rodzajów przepustowości (dodatnia/ujemna/zerowa), trzy możliwe położenia pętli (w źródle/węźle pośrednim/ujściu), trzy rodzaje przepustowości w pętli (dodatni/ujemny/zerowy), cztery możliwe kombinacje zwrotów krawędzi, oraz cztery warianty zwielokrotnienia krawędzi w topologii sieci. Nawet nie uwzględniając możliwości istnienia krawędzi bezpośredniej ze źródła do ujścia, czy braku którejś krawędzi. Ostatecznie zdecydowano się na dziesięć różnych klas abstrakcji.

\subsubsubsection{Sieć z 2 węzłami pośrednimi połączonymi równolegle}
\begin{itemize}
    \item Sieć z 2 węzłami pośrednimi połączonymi równolegle.
    \item Sieć z 2 węzłąmi pośrednimi połączonymi równolegle z krawędzią z pierwszego węzła pośredniego do drugiego węzłą pośredniego.%poprawna
    \item Sieć z 2 węzłąmi pośrednimi połączonymi równolegle z krawędzią z drugiego węzła pośredniego do pierwszego węzłą pośredniego.%poprawna
    \item Sieć z 2 węzłami pośrednimi połączonymi równolegle z krawędzią z pierwszego węzła pośredniego do drugiego węzła pośredniego i odwrotnie.%niby pętla
    \item Sieć z 2 węzłami pośrednimi połączonymi równolegle z krawędzią ze źródła do ujścia.
    \item Sieć z 2 węzłąmi pośrednimi połączonymi równolegle z krawędzia z ujścia do źródłą.
    \item Sieć z 2 węzłami pośrednimi połączonymi równolegle z krawędzią z ujścia do pierwszego węzła pośredniego.
    \item Sieć z 2 węzłami pośrednimi połączonymi równolegle z krawędzią z pierwszego węzłą pośredniego do źródłą.
    \item Sieć z 2 węzłami pośrednimi połączonymi równolegle z krawędzią z ujścia do pierwszego węzła pośredniego i z pierwszego węzłą pośredniego do źródła.
\end{itemize}
\subsubsubsection{Sieć z 2 węzłami pośrednimi połączonymi szeregowo}
\begin{itemize}
    \item Sieć z 2 węzłami pośrednimi połączonymi szeregowo.
    \item Sieć z 2 węzłami pośrednimi połączonymi szeregowo z wieloma krawędziami pomiędzy węzłami pośrednimi.
    \item Sieć z 2 węzłami pośrednimi połączonymi szeregowo z krawędzią o zerowym przepływie pomiędzy węzłami pośrednimi.
    \item Sieć z 2 węzłami pośrednimi połączonymi szeregowo z krawędzią o ujemnym przepływie pomiędzy węzłami pośrednimi.
    \item Sieć z 2 węzłami pośrednimi połączonymi szeregowo z krawędzią z drugiego węzłą pośredniego do pierwszego węzła pośredniego.
%Ujemny przepływ pomiędzy 1. a 2. węzłem, nie jest równoznaczny z odwróconą krawędzią, ujemny przepływ jest niepoprawny z definicji sieci, ale krawędź skierowana w złą stronę już nie, choć sieć jest dalej niepoprawna.
    \item Sieć z 2 węzłami pośrednimi połączonymi szeregowo z pętlą w pierwszym węźle pośrednim.
    \item Sieć z 2 węzłami pośrednimi połączonymi szeregowo z pętlą w drugim węźle pośrednim.
    \item Sieć z 2 węzłami pośrednimi połączonymi szeregowo z krawędzią od pierwszego węzła pośredniego do ujścia.
    \item Sieć z 2 węzłami pośrednimi połączonymi szeregowo z krawędzią od źródła do drugiego węzła pośredniego.
\end{itemize}

\subsubsubsection{Sieć z nieistniejącymi węzłami}
Liczba węzłów określona w danych wejściowych jest większa niż liczba zdefiniowanych węzłów.

\subsubsubsection{Sieć z ujemną liczbą węzłów}
W danych wejściowych liczba węzłów w sieci jest określona za pomocą liczby ujemnej.

\subsubsubsection{Sieć z 1000000000 węzłów}
Test dla granicznej wartości liczby węzłów.


\section{Projekt testów}


\section{Realizacja testów}


\section{Wykonanie testów}


\section{Ocena rezultatów testów}


\section{Wnioski}


\newpage
\section*{Historia dokumentu}
\begin{versions}
    \version*{1.0}{2012-12-21}{TC}%
        Zatwierdzono.
\end{versions}


\end{document}
