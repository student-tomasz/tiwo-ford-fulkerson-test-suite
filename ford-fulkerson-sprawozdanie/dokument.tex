\documentclass[10pt]{dokument-tiwo}


\begin{document}


\Tytul{Sprawozdanie z testowania algorytmu Forda-Fulkersona}
\Data{2012-12-21}
\Wersja{1.0}
\Autorzy{TC, MM, MO, RW}
\MakeDokumentMeta


\section{Plan testowania}
\subsection{Testy czarnoskrzynkowe}
\subsubsection{Klasy abstrakcji}
Na potrzeby realizacji testów czarnoskrzynkowych zostały zdefiniowane następujące klasy abstrakcji danych wejściowych:
\subsubsubsection{Sieć bez węzłów}
Sieć bez żadnych węzłów nie istnieje, nie da się dla niej wyznaczyć maksymalnego przepływu.
\subsubsubsection{Sieć z 1 węzłem}
Za szczególny przypadek sieci przepływu można uznać sieć składającą się z dokładnie jednego węzła. Niezależnie od pozostałych parametrów, taka sieć powinna zostać odrzucona jako nieprawidłowa. Wyróżnić można warianty złożone z:
\begin{itemize}
%bez niczego
%z pętlą
%z ujemną pętlą
    \item Samego źródła.
    \item Węzła pośredniego.
    \item Ujścia.
    \item Wspólnego źródła i ujścia.
\end{itemize}
Szczególną uwagę należy zwrócić na ostatni z wymienionych wariantów.

\subsubsubsection{Sieć bez węzłów pośrednich}
Sieć złożona ze źródłą i ujścia, bez jakichkolwiek węzłów pośrednich. Wyróżnione zostały następujące klasy abstrakcji:
\begin{itemize}
    \item Połączone pojedynczą krawędzią skierowaną od źródła do ujścia.
%o ujemnej przepustowości
%o dodatniej przepustowości
%o zerowej przepustowości
    \item Połączone pojedynczą krawędzią skierowaną od ujścia do źródła.
%o ujemnej przepustowości
%o dodatniej przepustowości
%o zerowej przepustowości
    \item Z wieloma krawędziami skierowanymi od żródła do ujścia.
    \item Z wieloma krawędziami skierowanymi od ujścia do źródła.
    
    \item Sieć bez węzłów pośrednich z pętlą.
    \item Sieć bez węzłów pośrednich z krawędzią o ujemnej przepustowości.
    \item Sieć bez węzłów pośrednich z krawędzią o zerowej przepustowości.
    \item Sieć bez węzłów pośrednich z krawędzią od ujścia do źródłą.
    \item Sieć bez węzłów pośrednich z pętlą o ujemnej przepustowości.
    \item Sieć bez węzłów pośrednich z wieloma krawędziami i pętlą.
    \item Sieć bez węzłów pośrednich z wieloma krawędziami i krawędzią o ujemnej przepustowości.
    \item Sieć bez węzłów pośrednich z wieloma krawędziami, pętlą i krawędzia o ujemnej przepustowości.
    \item Sieć bez węzłów pośrednich z wieloma krawędziami i pętlą o ujemnej przepustowości.   
\end{itemize}
\subsubsubsection{Sieć z 1 węzłem pośrednim}
\begin{itemize}
    \item Sieć z 1 węzłem pośrednim.
    \item Sieć z 1 węzłem pośrednim z krawędzią ze źródłą do węzła pośredniego i z ujścia do węzła pośredniego o dodatniej przepustowości.
    \item Sieć z 1 węzłem pośrednim z wieloma krawędziami od źródła do węzła pośredniego.
    \item Sieć z 1 węzłem pośrednim z wieloma krawędziami od węzła pośredniego do ujścia.
    \item Sieć z 1 węzłem pośrednim z krawędzią ze źródła do węzła pośredniego o dodatnim przepływie i z węzła pośredniego do ujścia o zerowej przepustowości.
    \item Sieć z 1 węzłem pośrednim z krawędzią ze źródła do węzła pośredniego o zerowym przepływie i z węzła pośredniego do ujścia o dodatniej przepustowości.
    \item Sieć z 1 węzłem pośrednim i pętlą w tym węźle.
    \item Sieć z 1 węzłem pośrednim i pętlą o ujemnej przepustowści w tym węźle.
    \item Sieć z 1 węzłem pośrednim i krawędzią ze źródła do ujścia.
\end{itemize}
\subsubsubsection{Sieć z 2 węzłami pośrednimi połączonymi równolegle}
\begin{itemize}
    \item Sieć z 2 węzłami pośrednimi połączonymi równolegle.
    \item Sieć z 2 węzłąmi pośrednimi połączonymi równolegle z krawędzią z pierwszego węzła pośredniego do drugiego węzłą pośredniego.%poprawna
    \item Sieć z 2 węzłąmi pośrednimi połączonymi równolegle z krawędzią z drugiego węzła pośredniego do pierwszego węzłą pośredniego.%poprawna
    \item Sieć z 2 węzłami pośrednimi połączonymi równolegle z krawędzią z pierwszego węzła pośredniego do drugiego węzła pośredniego i odwrotnie.%niby pętla
    \item Sieć z 2 węzłami pośrednimi połączonymi równolegle z krawędzią ze źródła do ujścia.
    \item Sieć z 2 węzłąmi pośrednimi połączonymi równolegle z krawędzia z ujścia do źródłą.
    \item Sieć z 2 węzłami pośrednimi połączonymi równolegle z krawędzią z ujścia do pierwszego węzła pośredniego.
    \item Sieć z 2 węzłami pośrednimi połączonymi równolegle z krawędzią z pierwszego węzłą pośredniego do źródłą.
    \item Sieć z 2 węzłami pośrednimi połączonymi równolegle z krawędzią z ujścia do pierwszego węzła pośredniego i z pierwszego węzłą pośredniego do źródła.
\end{itemize}
\subsubsubsection{Sieć z 2 węzłami pośrednimi połączonymi szeregowo}
\begin{itemize}
    \item Sieć z 2 węzłami pośrednimi połączonymi szeregowo.
    \item Sieć z 2 węzłami pośrednimi połączonymi szeregowo z wieloma krawędziami pomiędzy węzłami pośrednimi.
    \item Sieć z 2 węzłami pośrednimi połączonymi szeregowo z krawędzią o zerowym przepływie pomiędzy węzłami pośrednimi.
    \item Sieć z 2 węzłami pośrednimi połączonymi szeregowo z krawędzią o ujemnym przepływie pomiędzy węzłami pośrednimi.
    \item Sieć z 2 węzłami pośrednimi połączonymi szeregowo z krawędzią z drugiego węzłą pośredniego do pierwszego węzła pośredniego.
%Ujemny przepływ pomiędzy 1. a 2. węzłem, nie jest równoznaczny z odwróconą krawędzią, ujemny przepływ jest niepoprawny z definicji sieci, ale krawędź skierowana w złą stronę już nie, choć sieć jest dalej niepoprawna.
    \item Sieć z 2 węzłami pośrednimi połączonymi szeregowo z pętlą w pierwszym węźle pośrednim.
    \item Sieć z 2 węzłami pośrednimi połączonymi szeregowo z pętlą w drugim węźle pośrednim.
    \item Sieć z 2 węzłami pośrednimi połączonymi szeregowo z krawędzią od pierwszego węzła pośredniego do ujścia.
    \item Sieć z 2 węzłami pośrednimi połączonymi szeregowo z krawędzią od źródła do drugiego węzła pośredniego.
\end{itemize}

\section{Projekt testów}


\section{Realizacja testów}


\section{Wykonanie testów}


\section{Ocena rezultatów testów}


\section{Wnioski}


\newpage
\section*{Historia dokumentu}
\begin{versions}
    \version*{1.0}{2012-12-21}{TC}%
        Zatwierdzono.
\end{versions}


\end{document}
