\documentclass[10pt]{minutes}

%Code snippets
\usepackage{listings}
\usepackage{color}
\usepackage{textcomp}
\definecolor{listinggray}{gray}{0.9}
\definecolor{lbcolor}{rgb}{0.9,0.9,0.9}
\lstset{
	backgroundcolor=\color{lbcolor},
	tabsize=4,
	rulecolor=,
	language=Java,
        basicstyle=\scriptsize,
        upquote=true,
        aboveskip={1.5\baselineskip},
        columns=fixed,
        showstringspaces=false,
        extendedchars=true,
        breaklines=true,
        prebreak = \raisebox{0ex}[0ex][0ex]{\ensuremath{\hookleftarrow}},
        frame=single,
        showtabs=false,
        showspaces=false,
        showstringspaces=false,
        identifierstyle=\ttfamily,
        keywordstyle=\color[rgb]{0,0,1},
        commentstyle=\color[rgb]{0.133,0.545,0.133},
        stringstyle=\color[rgb]{0.627,0.126,0.941},
}



\begin{document}

\Tytul{\emph{Ford-Fulkerson Przepływ maksymalny}}
\Data{2012-12-12 1:00}
\Miejsce{Przyczółek}
\Obecni{MO}
\MakeMinutesMeta


\section{Wstęp} % (fold)
  \subsection{Opis lagorytmu}

\section{Instrukcja obsługi}
  \subsection{Dane wejściowe}
    Algorytm Forda-Fulkersona opisany powyżej przyjmuje dwa argumenty:
    \begin{itemize}
      \item \emph{FlowNetwork} - graf przepływu sieci
      \item \emph{Search} - metoda wykorzystywana do wyszukiwania ścieżek powiększających.
    \end{itemize}

  \subsubsection{Graf przepływu}
    Graf przepływu realizowany jest przy pomocy algorytmu z użyciem list powiązanych, \emph{FlowNetworkAdjacencyArray}. Do każdego wierzchołka grafu przypisywane są dwie listy,
 krawędzi przednich oraz tylnich. Rozwiązanie to jest nieodpowiednie dla zastosowań z użyciem dużych grafów, ponieważ zajmuje spore ilości pamięci.\\
Argumenty wejściowe:
    \begin{itemize}
      \item \emph{int} - ilość wszystkich węzłów w grafie
      \item \emph{int} - indeks węzła początkowego
      \item \emph{int} - indeks węzła docelowego
      \item \emph{Iterator<EdgesInfo>} - lista krawedzi wraz z ich przepustowością
    \end{itemize}

  \subsubsection{Krawędzie}
    Informacje o krawędziach grafu przechowywane są w kolekcji obiektów \emph{EgdeInfo}, które zawierają informację o wierzchołkach,
pomiędzy którymi dana krawędź się znajduje oraz o jakie posiada możliwości przepustowe.
    \begin{itemize}
      \item \emph{int} - indeks węzła startowego
      \item \emph{int} - indeks węzła końcowego
      \item \emph{int} - wartość przepustowa krawędzi
    \end{itemize}

  \subsection{Dane wyjściowe}
%wyjaśnione z rysunkami; sposób przechowywania danych wyjściowych.

\section{Przykład użycia}
%fragmenty kodu i rysunki
W pierwszej kolejności ustalane są główne parametry grafu.
\begin{itemize}
\item Ilość wezłów grafu
\item Indeks węzła źródłowego, \emph{source}
\item Indeks węzła docelowego, \emph{sink}
\end{itemize}
\begin{lstlisting}        
  int numVertices = 6; 
  int srcIndex = 0; 
  int sinkIndex = 5;
\end{lstlisting}

Następnie definiowane są poszczególne krawędzie grafu wraz z ich przepustowością.
\begin{itemize}
\item Indeks węzła początkowego krawedzi
\item Indeks węzła końcowego krawędzi
\item Maksymalna prapustowość krawędzi
\end{itemize}
\begin{lstlisting}
  ArrayList preIterator = new ArrayList();
  EdgeInfo edge1 = new EdgeInfo(0, 1, 3);
  EdgeInfo edge2 = new EdgeInfo(1, 3, 2);
  EdgeInfo edge3 = new EdgeInfo(3, 5, 3);
  EdgeInfo edge4 = new EdgeInfo(1, 4, 2);
  EdgeInfo edge5 = new EdgeInfo(0, 2, 2);
  EdgeInfo edge6 = new EdgeInfo(2, 4, 3);
  EdgeInfo edge7 = new EdgeInfo(4, 5, 2);
  EdgeInfo edge8 = new EdgeInfo(2, 3, 2);
  preIterator.add(edge1);
  preIterator.add(edge2);
  preIterator.add(edge3);
  preIterator.add(edge4);
  preIterator.add(edge5);
  preIterator.add(edge6);
  preIterator.add(edge7);
  preIterator.add(edge8);
                               
  Iterator<EdgeInfo> edges = preIterator.iterator();  
\end{lstlisting}
 
Tworzony jest obiekt reprazentujący cały graf przepływu.
\begin{lstlisting}      
  FlowNetworkAdjacencyList network = new FlowNetworkAdjacencyList(numVertices, srcIndex, sinkIndex, edges);  
        
  System.out.println(network.toString());
\end{lstlisting}

Wybierana jest funkcja wyszukująca i wykonywany jest algorytm Forda-Fulkersona.
\begin{lstlisting}
  //algorytm przeszukiwania wszerz
  BFS_SearchList search = new BFS_SearchList(network);

  //algorytm FORD-FULKERSON
  FordFulkerson fordFulkerson = new FordFulkerson(network, search);

  //wykonaj algorytm
  fordFulkerson.compute();
\end{lstlisting}

Wyświetlenie wyznaczonego maksymlnego przepływu.
\begin{lstlisting}        
  System.out.println("Wynik:");
  System.out.println(network.toString());
\end{lstlisting}
       

\section{Słownik pojęć i definicje}
% może w postaci tabelki ładnie zrobić?


\end{document}
