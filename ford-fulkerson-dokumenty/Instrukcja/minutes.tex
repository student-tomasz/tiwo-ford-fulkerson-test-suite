\documentclass[10pt]{minutes}

%Code snippets
\usepackage{listings}
\usepackage{color}
\usepackage{textcomp}
\definecolor{listinggray}{gray}{0.9}
\definecolor{lbcolor}{rgb}{0.9,0.9,0.9}
\lstset{
	backgroundcolor=\color{lbcolor},
	tabsize=4,
	rulecolor=,
	language=matlab,
        basicstyle=\scriptsize,
        upquote=true,
        aboveskip={1.5\baselineskip},
        columns=fixed,
        showstringspaces=false,
        extendedchars=true,
        breaklines=true,
        prebreak = \raisebox{0ex}[0ex][0ex]{\ensuremath{\hookleftarrow}},
        frame=single,
        showtabs=false,
        showspaces=false,
        showstringspaces=false,
        identifierstyle=\ttfamily,
        keywordstyle=\color[rgb]{0,0,1},
        commentstyle=\color[rgb]{0.133,0.545,0.133},
        stringstyle=\color[rgb]{0.627,0.126,0.941},
}



\begin{document}

\Tytul{\emph{Ford-Fulkerson Przepływ maksymalny}}
\Data{2012-12-12 1:00}
\Miejsce{Przyczółek}
\Obecni{MO}
\MakeMinutesMeta


\section{Wstęp} % (fold)
  \subsection{Opis lagorytmu}

\section{Instrukcja obsługi}
  \subsection{Dane wejściowe}
    Algorytm Forda-Fulkersona opisany powyżej przyjmuje dwa argumenty:
    \begin{itemize}
      \item \emph{FlowNetwork} - graf przepływu sieci
      \item \emph{Search} - metoda wykorzystywana do wyszukiwania ścieżek powiększających.
    \end{itemize}

  \subsubsection{Graf przepływu}
    Graf przepływu realizowany jest przy pomocy algorytmu z użyciem list powiązanych, \emph{FlowNetworkAdjacencyArray}. Do każdego wierzchołka grafu przypisywane są dwie listy,
 krawędzi przednich oraz tylnich. Rozwiązanie to jest nieodpowiednie dla zastosowań z użyciem dużych grafów, ponieważ zajmuje spore ilości pamięci.\\
Argumenty wejściowe:
    \begin{itemize}
      \item \emph{int} - ilość wszystkich węzłów w grafie
      \item \emph{int} - indeks węzła początkowego
      \item \emph{int} - indeks węzła docelowego
      \item \emph{Iterator<EdgesInfo>} - lista krawedzi wraz z ich przepustowością
    \end{itemize}

  \subsubsection{Krawędzie}
    Informacje o krawędziach grafu przechowywane są w kolekcji obiektów \emph{EgdeInfo}, które zawierają informację o wierzchołkach,
pomiędzy którymi dana krawędź się znajduje oraz o jakie posiada możliwości przepustowe.
    \begin{itemize}
      \item \emph{int} - indeks węzła startowego
      \item \emph{int} - indeks węzła końcowego
      \item \emph{int} - wartość przepustowa krawędzi
    \end{itemize}

  \subsection{Dane wyjściowe}
%wyjaśnione z rysunkami; sposób przechowywania danych wyjściowych.

\section{Przykłady użycia}
%fragmenty kodu i rysunki
\begin{lstlisting}
if draw
    print([outputpath, 'mygraph.eps'],'-depsc')
end
\end{lstlisting}
\section{Słownik pojęć i definicje}
% może w postaci tabelki ładnie zrobić?


\end{document}
