\subsection{Wprowadzenie}
Testowaniu podlega system realizujący algorytm \textsc{Forda-Fulkersona} służący
do wyznaczania maksymalnego przepływu w sieciach. Testowane elementy pochodzą z
kodu załączonego wraz z książką \emph{Algorytmy. Almanach}. Celem testowania
jest zweryfikowanie poprawności już zaimplementowanej funkcjonalności.

\subsection{Testowane elementy}
Testowaniu podlega projekt \texttt{ford-fulkerson-test-suite} zawierający
implementację algorytmu \textsc{Forda-Fulkersona}. Testowana wersja projektu
znajduje się w repozytorium, w rewizji \texttt{e5ebef7} o nazwie \texttt
{testing-final}.

\subsection{Testowana funkcjonalność}
Uwaga procesu testowania będzie skupiona na konkretnych implementacjach systemu:
\begin{itemize}
    \item dla klasy \texttt{FlowNetwork} --- implementacja
        \texttt{FlowNetworkArray},
    \item dla klasy \texttt{Search} --- implementacja \texttt{DFS\_SearchArray}.
\end{itemize}
W chwili obecnej dostępne są też alternatywne implementacje operujące na
listach, odpowiednio \texttt{FlowNetworkAdjacencyList} oraz
\texttt{DFS\_SearchList}. Wyszczególnienie implementacji jest istotne dla
planowania testów biało\dywiz skrzynkowych. Testy czarno\dywiz skrzynkowe
nie mają wiedzy o wybranej implementacji, powinny działać niezależnie od wyboru.

\subsection{Testowana funkcjonalność --- wyłączenia}
Do zakresu testów nie wchodzą moduły pomocnicze:
\begin{itemize}
    \item pakiet \texttt{algs.heap},
    \item pakiet \texttt{algs.list}.
\end{itemize}
Oraz moduły nie wykorzystywane przez testowany system algorytmu:
\begin{itemize}
    \item moduł \texttt{algs.network.generator}.
\end{itemize}

\subsection{Podejście}
Środowisko testowe zakłada korzystanie z:
\begin{itemize}
    \item NetBeans IDE 7.2.1
    \item Oracle JDK 1.7\_u9
    \item TestNG 6.8
\end{itemize}

\noindent
Specyficzne metodyki, narzędzia i kryteria satysfakcji są opisane w sekcji
\ref{sec:szczegolowy_plan}.
