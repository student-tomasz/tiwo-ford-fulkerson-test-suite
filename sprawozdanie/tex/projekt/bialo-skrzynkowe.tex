\subsection{Testowanie biało\dywiz skrzynkowe}

\subsubsection{algs.network.VertexStructure}
\begin{center}
\begin{tabular}{@{} >{\ttfamily}p{0.2\textwidth} @{\hspace{0.02\textwidth}} p{0.6\textwidth} @{}}
    \toprule
    {\bfseries Id} & B.1.1 \\
    \hline
    {\bfseries Funkcja} & \bfseries String \texttt{toString()} \\
    \hline
    {\bfseries Opis} & Wypisuje listę węzłów wychodzących i wchodzacych. \\
    \hline
    {\bfseries Przypadki testowe} & {\begin{enumerate}
                                        \item Struktura bez zdefiniowanych krawędzi wejscia/wyjścia
                                        \item Struktura z pojedynczą krawędzią wejścia/wyjścia
                                        \item Struktura z wieloma krawędziami wejścia wyjścia
                                    \end{enumerate}} \\
    \hline
    {\bfseries Wykonawca} & MO \\
    \bottomrule
\end{tabular}
\end{center}

\subsubsection{algs.network.FlowNetworkArray}
\begin{center}
\begin{tabular}{@{} >{\ttfamily}p{0.2\textwidth} @{\hspace{0.02\textwidth}} p{0.6\textwidth} @{}}
    \toprule
    {\bfseries Id} & B.2.1 \\
    \hline
    {\bfseries Funkcja} & \texttt{FlowNetworkArray (\bfseries int sourceIndex, \bfseries int sinkIndex, \bfseries int numVertices)} \\
    \hline
    {\bfseries Opis} & Konstruktor minimalnej struktury sieci. Inicjalizuje tylko niezbedne zmienne. \\
    \hline
    {\bfseries Przypadki testowe} & {\begin{enumerate}
                                        \item Wartości oczekiwane
                                        \item Ujemne wartości
                                        \item sinkIndex < sourceIndex
                                        \item sinkIndex - sourceIndex > numVertices
                                    \end{enumerate}} \\
    \hline
    {\bfseries Wykonawca} & MO \\
    \bottomrule
\end{tabular}
\end{center}

\begin{center}
\begin{tabular}{@{} >{\ttfamily}p{0.2\textwidth} @{\hspace{0.02\textwidth}} p{0.6\textwidth} @{}}
    \toprule
    {\bfseries Id} & B.2.1 \\
    \hline
    {\bfseries Funkcja} & \texttt{FlowNetworkArray (\bfseries int sourceIndex,
                                                    \bfseries int sinkIndex, \bfseries int numVertices),
                                                    \bfseries Iterator<Edges> edges} \\
    \hline
    {\bfseries Opis} & Konstruktor struktury reprezentującej graf przpływu. \\
    \hline
    {\bfseries Przypadki testowe} & {\begin{enumerate}
                                        \item Wartości oczekiwane
                                        \item Ujemne wartości
                                        \item sinkIndex - sourceIndex > numVertices
                                        \item sinkIndex < sourceIndex
                                        \item Pusta kolekcja krawędzi
                                    \end{enumerate}} \\
    \hline
    {\bfseries Wykonawca} & MO \\
    \bottomrule
\end{tabular}
\end{center}

\begin{center}
\begin{tabular}{@{} >{\ttfamily}p{0.2\textwidth} @{\hspace{0.02\textwidth}} p{0.6\textwidth} @{}}
    \toprule
    {\bfseries Id} & B.2.2 \\
    \hline
    {\bfseries Funkcja} & \bfseries void \texttt{ validate()} \\
    \hline
    {\bfseries Opis} & Metoda weryfikuje czy informacje na temat sieci są akceptowalne.
                       Zwracany jest wyjątek, IllegalStateException w dwóch przypadkach:
        \begin{enumerate}
            \item Przepływ krawędzi jest wiekszy niż przepustowaość przepustowaść
            \item Ilość krawędzi wchodzących jest różna od krawędzi wychodzących
        \end{enumerate}\\
    \hline
    {\bfseries Przypadki testowe} & Tak jak wyżej. \\
    \hline
    {\bfseries Wykonawca} & MO \\
    \bottomrule
\end{tabular}
\end{center}
