\subsection{Testowanie z wykorzystaniem obiektów pozornych}

\subsubsection{Wprowadzenie}
Celem testów integracyjnych jest sprawdzenie poprawności danych przekazywanych między modułami.
Wykorzystanie obiektów pozornych znacznie upraszcza ten proces, dając możliwość testowania "jednocześnie" tylko jednego modułu.
\subsubsection{Testowane elementy}
Punktem wejściowym testów jest klasa \texttt{FordFulkerson} zawierająca implementację algorytmu \texttt{Forda-Fulkersona}.

\subsubsection{Podejście}
Testy zostaną przeprowadzone przy podejściu z góry do dołu, oraz integracji w szerz. Jako punkt wejścia obrana została klasa FordFulkerson.


\subsubsection{Zadania testowania}
\begin{center}
\begin{tabular}{@{} p{0.44\textwidth} @{\hspace{0.02\textwidth}} p{0.22\textwidth} @{}}
    \toprule
    {\bfseries Zadanie} & {\bfseries Osoba odpowiedzialna} \\
    \toprule
    Projekt testów & MM \\
    Przygotowanie przypadków testowych & MM \\
    Implementacja testów & MM \\
    Uruchomienie testów i weryfikacja wyników & MM, TC \\
    Akceptacja wyników przebiegu testowania & TC \\
    \bottomrule
\end{tabular}
\end{center}
