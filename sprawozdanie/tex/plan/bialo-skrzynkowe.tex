\subsection{Testowanie biało\dywiz skrzynkowe}

\subsubsection{Wprowadzenie}
Kryterium satysfakcji przeprowadzanych testów wyznaczane jest na podstawie
współczynnika pokrycia kodu. Minimalną dopuszczalną wartością jest 90\%. Część
testów została dostarczona razem z systemem.

\subsubsection{Testowane elementy}
Lista poszczególnych klas poddawanych testowaniu oraz pokrycie kodu zapewnione
przez testy zawarte wraz z kodem źródłowym.

\begin{center}
\begin{tabular}{@{} >{\ttfamily}p{0.33\textwidth} @{\hspace{0.02\textwidth}} c @{}}
    \toprule
    \multicolumn{2}{@{}c@{}}{\bfseries Pakiet \texttt{algs.network}} \\
    \midrule
    {\normalfont\bfseries Klasa} & {\bfseries Aktualne pokrycie} \\
    \toprule
    FordFulkerson & 100\% \\
    FlowNetwork & 100\% \\
    FlowNetworkAdjacencyList & 75\% \\
    FlowNetworkArray & 94\% \\
    Search & 100\% \\
    DFS\_SearchList & 100\% \\
    DFS\_SearchArray & 100\% \\
    EdgeInfo & 100\% \\
    VertexInfo & 100\% \\
    VertexStructure & 93\% \\
    \bottomrule
\end{tabular}
\end{center}
Załacznik: \href{./raport/index.html}{Raport obecnego stanu pokrycia.}

\subsubsection{Funkcje wyłączone z testowania}
Lista metod dostarczanych przez testowane klasy, które nie wchodzą w zakres
testowania biało\dywiz skrzynkowego:
\begin{itemize}[nosep]
    \item \verb|FlowNetworkAdjacencyList.getCost()|
\end{itemize}

\subsubsection{Podejście}
Testowanie zostanie przeprowadzone techniką testowania strukturalnego. Głównym
kryterium zaliczenia testów będzie współczynnik pokrycia linii kodu, którego wartość
musi przekraczać 90\%.\\

\noindent
Dodatkowo do wytycznych środowiska testowego ustalonych globalnie użyte zostaną:
\begin{itemize}[nosep]
    \item Cobertura plugin 1.9.4.1
\end{itemize}
