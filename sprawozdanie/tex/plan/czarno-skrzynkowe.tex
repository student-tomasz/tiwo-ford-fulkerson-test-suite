\subsection{Testowanie czarno\dywiz skrzynkowe}

\subsubsection{Wprowadzenie}
Cel testów czarno\dywiz skrzynkowych nie różni się od wcześniejszych założeń.
Testy czarno\dywiz skrzynkowe koncentrują się na weryfikacji wyników działania
algorytmu \textsc{Forda\dywiz Fulkersona} dla określonych danych wejściowych.

\subsubsection{Testowane elementy}
Głównym obiektem testów czarno\dywiz skrzynkowych jest klasa
\texttt{FordFulkerson} z projektu \texttt{ford-fulkerson-test-suite},
zawierająca implementację algorytmu \textsc{Forda-Fulkersona}. W trakcie testów
bezpośrednio wykorzystywane są również inne klasy wchodzące do projektu
\texttt{ford-fulkerson-test-suite}.

\subsubsection{Testowana funkcjonalność}
Zakres testów czarno\dywiz skrzynkowych obejmuje:
\begin{enumerate}
    \item Działanie konstruktora klasy \texttt{FordFulkerson}.
    \item Możliwość wykonania metody \texttt{compute()} klasy \texttt{FordFulkerson}.
    \item Poprawność wyników działania implementacja algorytmu \textsc{Forda\dywiz
        Fulkersona} w klasie \texttt{FordFulkerson}. Weryfikacja polega na
        porównaniu wartości zwracanych przez \texttt{toString()} po wykonaniu
        metody \texttt{compute()}, do wartości zwracanych przez wyrocznię.
\end{enumerate}

\subsubsection{Testowana funkcjonalność --- wyłączenia}
Z testowania czarno\dywiz skrzynkowego wyłączone zostaną funkcjonalności:
\begin{itemize}
    \item Z klas niewykorzystywanych ze względu na wybór implementacji:
        \begin{itemize}
            \item \texttt{DFS\_SearchList},
            \item \texttt{FlowNetworkAdjacencyList},
            \item \texttt{ShortestPathArray}.
        \end{itemize}
    \item Z klas abstrakcyjnych, po których dziedziczą wykorzystywane klasy:
        \begin{itemize}
            \item \texttt{FlowNetwork}
            \item \texttt{Search}
        \end{itemize}
    \item Z klas reprezentujących szczegóły implementacji:
        \begin{itemize}
            \item \texttt{EdgeInfo},
            \item \texttt{DFS\_SearchArray} -- implementująca algorytm
                przeszukiwania \textsc{Depth\dywiz first search} na
                implementacji sieci przepływu z klasy \texttt{FlowNetworkArray},
                której instancja jest wymagana jako drugi argument konstruktura
                klasy \texttt{FordFulkerson};
            \item \texttt{VertexInfo},
            \item \texttt{VertexStructure}.
        \end{itemize}
\end{itemize}

\subsubsection{Podejście}
Podstawą do projektu testów czarno\dywiz skrzynkowych jest wyróżnienie klas
abstrakcji. Odpowiedni dobór klas pozwoli przetestować implementację
algorytmu \textsc{Forda\dywiz Fulkersona} w rozsądnym zakresie, jednocześnie
unikając testowania totalnego. Zidentyfikowane zostały następujące klasy
abstrakcji:
\begin{enumerate}
    \item Ze względu na ilość węzłów:
        \begin{enumerate}
            \item Ujemna ilość węzłów.
            \item Brak jakichkolwiek węzłów.
            \item Dokładnie jeden węzeł, źródłowy, pośredni lub ujście.
            \item Sieć przepływowa bez węzłów pośrednich.
            \item Sieć przepływowa z jednym węzłem pośrednim -- jako minimalna
                ilość węzłów pośrednich w sieci przepływowej z węzłami
                pośrednimi.
            \item Sieć przepływowa z kilkoma węzłami pośrednimi -- jako
                niegraniczna ilość węzłów w sieci przepływowej z węzłami
                pośrednimi.
            \item Sieć przepływowa z $1e9$ węzłami pośrednimi -- jako maksymalna
                ilość węzłów pośrednich w sieci przepływowej z węzłami
                pośrednimi.
            \item Ilość węzłów mniejsza od zadeklarowanej.
            \item Ilość węzłów większa od zadeklarowanej.
        \end{enumerate}
    \item Ze względu na pętle:
        \begin{enumerate}
            \item W źródle.
            \item W węźle pośrednim.
            \item W ujściu.
            \item Bez pętli.
        \end{enumerate}
    \item Ze względu na krawędzie bezpośrednie pomiędzy źródłem, a ujściem:
        \begin{enumerate}
            \item Ze źródła do ujścia.
            \item Z ujścia do źródła.
            \item Bez krawędzi pomiędzy źródłem, a ujściem.
        \end{enumerate}
    \item Ze względu na przepustowość krawędzi:
        \begin{enumerate}
            \item Dodatnia.
            \item Zerowa.
            \item Ujemna.
        \end{enumerate}
    \item Ze względu na poprawność krawędzi:
        \begin{enumerate}
            \item Poprawne krawędzie.
            \item Krawędzie zaczynające się w nieistniejącym węźle.
            \item Krawędzie kończące się w nieistniejącym węźle.
        \end{enumerate}
    \item Ze względu na istnienie krawędzi zwielokrotnionych:
        \begin{enumerate}
            \item Bez krawędzi zwielokrotnionych.
            \item Z bezpośrednią zwielokrotnioną krawędzią pomiędzy źródłem, a ujściem.
            \item Z krawędziami zwielokrotnionymi o przeciwnych zwrotach.
        \end{enumerate}
\end{enumerate}
