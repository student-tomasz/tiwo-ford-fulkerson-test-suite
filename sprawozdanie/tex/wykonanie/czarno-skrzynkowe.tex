\subsection{Testowanie czarno\dywiz skrzynkowe}

Testy czarno\dywiz skrzynkowe są załączone w zestawie
\texttt{BlackBoxTests.xml} z pakietu \texttt{rw.blackbox}.

\subsubsection{Sieć bez węzłów}
Wszystkie testy na sieci bez węzłów zrealizowano w klasie
\texttt{EmptyNetworkTest}.

\begin{center}
\begin{tabular}{@{} >{\bfseries}p{0.2\textwidth} @{\hspace{0.02\textwidth}} p{0.6\textwidth} @{}}
    \toprule
    TestCase & \texttt{rw.blackbox.EmptyNetworkTest.testCase1()} \\
    \midrule
    ID & C.1.1.1 \\
    \midrule
    Oczekiwany wynik &
    \begin{minipage}[h]{0.6\textwidth}
        \texttt{0}
    \end{minipage} \\
    \midrule
    Uzyskany wynik &
    \begin{minipage}[h]{0.6\textwidth}
        \texttt{0}
    \end{minipage} \\
    \midrule
    Rezultat & \textsc{Test zaliczony} \\
    \bottomrule
\end{tabular}
\end{center}

\begin{center}
\begin{tabular}{@{} >{\bfseries}p{0.2\textwidth} @{\hspace{0.02\textwidth}} p{0.6\textwidth} @{}}
    \toprule
    TestCase & \texttt{rw.blackbox.EmptyNetworkTest.testCase1()} \\
    \midrule
    ID & C.1.1.2 \\
    \midrule
    Oczekiwany wynik &
    \begin{minipage}[h]{0.6\textwidth}
        \texttt{Exception}
    \end{minipage} \\
    \midrule
    Uzyskany wynik &
    \begin{minipage}[h]{0.6\textwidth}
        \texttt{Exception}
    \end{minipage} \\
    \midrule
    Rezultat & \textsc{Test zaliczony} \\
    \bottomrule
\end{tabular}
\end{center}

\subsubsection{Sieć z 1 węzłem}
Wszystkie testy zrealizowane w klasie \texttt{JustOneElementTest}.

\begin{center}
\begin{tabular}{@{} >{\bfseries}p{0.2\textwidth} @{\hspace{0.02\textwidth}} p{0.6\textwidth} @{}}
    \toprule
    TestCase & \texttt{rw.blackbox.JustOneElementTest.testCase1()} \\
    \midrule
    ID & C.1.2.1 \\
    \midrule
    Opis & Samego źródła. \\
    \midrule
    Oczekiwany wynik &
    \begin{minipage}[h]{0.6\textwidth}
        \texttt{0}
    \end{minipage} \\
    \midrule
    Uzyskany wynik &
    \begin{minipage}[h]{0.6\textwidth}
        \texttt{0}
    \end{minipage} \\
    \midrule
    Rezultat & \textsc{Test zaliczony} \\
    \bottomrule
\end{tabular}
\end{center}

\begin{center}
\begin{tabular}{@{} >{\bfseries}p{0.2\textwidth} @{\hspace{0.02\textwidth}} p{0.6\textwidth} @{}}
    \toprule
    TestCase & \texttt{rw.blackbox.JustOneElementTest.testCase2()} \\
    \midrule
    ID & C.1.2.2 \\
    \midrule
    Opis & Węzła wspólnego. \\
    \midrule
    Oczekiwany wynik &
    \begin{minipage}[h]{0.6\textwidth}
        \texttt{Exception}
    \end{minipage} \\
    \midrule
    Uzyskany wynik &
    \begin{minipage}[h]{0.6\textwidth}
        \texttt{Exception}
    \end{minipage} \\
    \midrule
    Rezultat & \textsc{Test zaliczony} \\
    \bottomrule
\end{tabular}
\end{center}

\begin{itemize}[nosep]
    \item Ujścia.
    \texttt{testCase3}
    Oczekiwany wynik: Exception
    Uzyskany wynik: Exception
    \textsc{Test zaliczony}

    \item Wspólnego źródła i ujścia.
    \texttt{testCase4}
    Oczekiwany wynik: 0
    Uzyskany wynik: 0
    \textsc{Test zaliczony}

    \item Wspólnego źródła i ujścia z pętlą.
    \texttt{testCase5}
    Oczekiwany wynik: 0
    Uzyskany wynik: 0
    \textsc{Test zaliczony}
\end{itemize}


\subsubsection{Sieć bez węzłów pośrednich}
Testy na sieciach bez węzłów pośrednich zrealizowano w klasie JustSourceAndSinkTest.
\begin{itemize}[nosep]
    \item Nie połączonymi żadną krawędzią.
    \texttt{testCase1}
    Oczekiwany wynik: 0
    Uzyskany wynik: 0
    \textsc{Test zaliczony}

    \item Połączonymi pojedynczą krawędzią skierowaną od źródła do ujścia o dodatniej przepustowości.
    \texttt{testCase2}
    Oczekiwany wynik: 9
    Uzyskany wynik: 9
    \textsc{Test zaliczony}

    \item Połączonymi pojedynczą krawędzią skierowaną od źródła do ujścia o dodatniej przepustowości z pętlą o dodatniej przepustowości w ujściu.
    \texttt{testCase3a}
    Oczekiwany wynik: 57956756
    Uzyskany wynik: 57956756
    \textsc{Test zaliczony}

    \texttt{testCase3b}
    Oczekiwany wynik: Exception
    Uzyskany wynik: brak wyjątku
    \textsc{Test nie zaliczony}

    \item Połączonymi pojedynczą krawędzią skierowaną od źródła do ujścia o dodatniej przepustowości z pętlą o ujemnej przepustowości w źródle.
    \texttt{testCase4a}
    Oczekiwany wynik: 346723
    Uzyskany wynik: 346723
    \textsc{Test zaliczony}

    \texttt{testCase4b}
    Oczekiwany wynik: Exception
    Uzyskany wynik: brak wyjątku
    \textsc{Test nie zaliczony}

    \item Połączonymi pojedynczą krawędzią skierowaną od źródła do ujścia o zerowej przepustowości.
    \texttt{testCase5}
    Oczekiwany wynik: 0
    Uzyskany wynik: 0
    \textsc{Test zaliczony}

    \item Połączonymi pojedynczą krawędzią skierowaną od źródła do ujścia o ujemnej przepustowości.
    \texttt{testCase6a}
    Oczekiwany wynik: 0
    Uzyskany wynik: 0
    \textsc{Test zaliczony}

    \texttt{testCase6b}
    Oczekiwany wynik: Exception
    Uzyskany wynik: Exception
    \textsc{Test zaliczony}

    \item Połączonymi pojedynczą krawędzią skierowaną od ujścia do źródła o dodatniej przepustowości.
    \texttt{testCase7}
    Oczekiwany wynik: 0
    Uzyskany wynik: 0
    \textsc{Test zaliczony}

    \item Połączonymi pojedynczą krawędzią skierowaną od ujścia do źródła o ujemnej przepustowości.
    \texttt{testCase8a}
    Oczekiwany wynik: 0
    Uzyskany wynik: 0
    \textsc{Test zaliczony}

    \texttt{testCase8b}
    Oczekiwany wynik: Exception
    Uzyskany wynik: brak wyjątku
    \textsc{Test nie zaliczony}

    \item Z wieloma krawędziami skierowanymi od żródła do ujścia.
    \texttt{testCase9a}
    Oczekiwany wynik: 18
    Uzyskany wynik: 4
    \textsc{Test nie zaliczony}

    \texttt{testCase9b}
    Oczekiwany wynik: Exception
    Uzyskany wynik: brak wyjątku
    \textsc{Test nie zaliczony}

    \item Z wieloma krawędziami skierowanymi od ujścia do źródła.
    \texttt{testCase10a}
    Oczekiwany wynik: 0
    Uzyskany wynik: 0
    \textsc{Test zaliczony}

    \texttt{testCase10b}
    Oczekiwany wynik: Exception
    Uzyskany wynik: brak wyjątku


    \item Z wieloma krawędziami skierowanymi w różnych stronach.
    \texttt{testCase11a}
    Oczekiwany wynik: 4
    Uzyskany wynik: 4
    \textsc{Test zaliczony}

    \texttt{testCase11b}
    Oczekiwany wynik: 13
    Uzyskany wynik: 7
    \textsc{Test nie zaliczony}

    \texttt{testCase11c}
    Oczekiwany wynik: Exception
    Uzyskany wynik: brak wyjątku
    \textsc{Test nie zaliczony}


\end{itemize}


\subsubsection{Sieć z 1 węzłem pośrednim}
Testy na sieciach z jednym węzłem pośrednim zostały zgrupowane
w klasie SingleVertexTest.
\begin{itemize}[nosep]
    \item Połączone pojedynczymi krawędziami skierowanymi ze źródła do węzłą
    pośredniego i z węzła pośredniego do ujścia.
    \texttt{testCase1a}
    Oczekiwany wynik: 126
    Uzyskany wynik: 126
    \textsc{Test zaliczony}

    \texttt{testCase1b}
    Oczekiwany wynik: 75269
    Uzyskany wynik: 75269
    \textsc{Test zaliczony}

    \item Z pojedynczymi krawędziami skierowanymi z ujścia do węzła pośredniego
    i z węzła pośredniego do źródła.
    \texttt{testCase2}
    Oczekiwany wynik: 0
    Uzyskany wynik: 0
    \textsc{Test zaliczony}

    \item Z pojedynczymi krawędziami skierowanymi z węzła pośredniego do ujścia
    i z węzła pośredniego do źródła.
    \texttt{testCase3}
    Oczekiwany wynik: 0
    Uzyskany wynik: 0
    \textsc{Test zaliczony}

    \item Z pojedynczymi krawędziami skierowanymi ze źródła do węzła pośredniego
    i z ujścia do węzła pośredniego.
    \texttt{testCase4}
    Oczekiwany wynik: 0
    Uzyskany wynik: 0
    \textsc{Test zaliczony}

    \item Połączone pojedynczymi krawędziami skierowanymi ze źródła do węzła
    pośredniego i z węzła pośredniego do źródła z dodatkową krawędzią o
    dodatniej przepustowości ze źródła do ujścia.
    \texttt{testCase5}
    Oczekiwany wynik: 83570
    Uzyskany wynik: 83570
    \textsc{Test zaliczony}

    \item Połączone pojedynczymi krawędziami skierowanymi ze źródła do węzła
    pośredniego i z węzła pośredniego do źródła z pętlą o dodatniej
    przepustowości w węźle pośrednim.
    \texttt{testCase6a}
    Oczekiwany wynik: 17
    Uzyskany wynik: 17
    \textsc{Test zaliczony}

    \texttt{testCase6b}
    Oczekiwany wynik: Exception
    Uzyskany wynik: brak wyjątku

    \item Połączone zwielokrotnionymi krawędziami ze źródła do węzła pośredniego
    i z węzła pośredniego do źródła, z mieszanymi zwrotami.
    \texttt{testCase7a}
    Oczekiwany wynik: 5
    Uzyskany wynik: 5
    \textsc{Test zaliczony}

    \texttt{testCase7b}
    Oczekiwany wynik: Exception
    Uzyskany wynik: brak wyjątku

    \texttt{testCase7c}
    Oczekiwany wynik: 16
    Uzyskany wynik: 13
    \textsc{Test nie zaliczony}

    \item Połączone pojedynczą krawędzią skierowaną ze źródła do węzła
    pośredniego, bez krawędzi do ujścia.

    \texttt{testCase8}
    Oczekiwany wynik: 0
    Uzyskany wynik: 0
    \textsc{Test zaliczony}

    \item Połączone pojedynczą krawędzia skierowaną ze źródła do węzła
    pośredniego, oraz krawędzią ze źródłą do ujścia.
    \texttt{testCase9}
    Oczekiwany wynik: 5
    Uzyskany wynik: 5
    \textsc{Test zaliczony}

    \item Połączone pojedynczą krawędzia skierowaną z węzła pośredniego do
    ujścia, bez połączenia ze źródłem.
    \texttt{testCase10}
    Oczekiwany wynik: 0
    Uzyskany wynik: 0
    \textsc{Test zaliczony}
\end{itemize}


\subsubsection{Sieć z 2 węzłami pośrednimi połączonymi równolegle}
Testy operujące na sieciach z 2 węzłami pośrednimi połączonymi równolegle zostały
zebrane w klasie TwoParallelVerticesTest.
\begin{itemize}[nosep]
    \item Z krawędziami o dodatniej przepustowości ze źródła do obu węzłów
    pośrednich i z obu węzłów pośrednich do ujścia.
    \texttt{testCase1}
    Oczekiwany wynik: 4
    Uzyskany wynik: 4
    \textsc{Test zaliczony}


    \item Z dodatkową krawędzią o dodatniej przepustowości z pierwszego węzła
    pośredniego do drugiego węzła pośredniego.
    \texttt{testCase2}
    Oczekiwany wynik: 8
    Uzyskany wynik: 8
    \textsc{Test zaliczony}

    \item Z dodatkową krawędzią o dodatniej przepustowości z drugiego węzła
    pośredniego do pierwszego węzła pośredniego.
    \texttt{testCase3}
    Oczekiwany wynik: 9
    Uzyskany wynik: 9
    \textsc{Test zaliczony}

    \item Z dodatkowymi krawędziami o dodatniej przepustowości z pierwszego
    węzła pośredniego do drugiego węzła pośredniego i z drugiego węzła
    pośredniego do pierwszego węzła pośredniego.
    \texttt{testCase4}
    Oczekiwany wynik: 8
    Uzyskany wynik: 8
    \textsc{Test zaliczony}

    \item Z dodatkową krawędzią o dodatniej przepustowości ze źródła do ujścia.
    \texttt{testCase5}
    Oczekiwany wynik: 10
    Uzyskany wynik: 10
    \textsc{Test zaliczony}

    \item Z dodatkową krawędzia o dodatniej przepustowości z ujścia do źródła.
    \texttt{testCase6}
    Oczekiwany wynik: 3
    Uzyskany wynik: 3
    \textsc{Test zaliczony}

    \item Z dodatkową krawędzią o dodatniej przepustowości z ujścia do
    pierwszego węzła pośredniego.
    \texttt{testCase7}
    Oczekiwany wynik: 5
    Uzyskany wynik: 5
    \textsc{Test zaliczony}

    \item Z dodatkową krawędzią o dodatniej przepustowości z pierwszego węzła
    pośredniego do źródłą.
    \texttt{testCase8}
    Oczekiwany wynik: 8
    Uzyskany wynik: 8
    \textsc{Test zaliczony}

    \item Z dodatkowymi krawędziami o dodatniej przepustowości z ujścia do
    pierwszego węzła pośredniego i z pierwszego węzłą pośredniego do źródła.
    \texttt{testCase9}
    Oczekiwany wynik: 5
    Uzyskany wynik: 5
    \textsc{Test zaliczony}
\end{itemize}


\subsubsection{Sieć z 2 węzłami pośrednimi połączonymi szeregowo}
Testy wykorzystujące sieci z 2 węzłami pośrednimi znajdują się z kolei w klasie
TwoSerialVerticesTest.
\begin{itemize}[nosep]
    \item Z krawędziami o dodatniej przepustowości ze źródła do pierwszego węzła
    pośredniego, z pierwszego węzła pośredniego do drugiego węzła pośredniego i z
    drugiego węzła pośredniego do ujścia.
    \texttt{testCase1}
    Oczekiwany wynik: 3
    Uzyskany wynik: 3
    \textsc{Test zaliczony}

    \item Sieć z 2 węzłami pośrednimi połączonymi szeregowo z wieloma
    krawędziami pomiędzy węzłami pośrednimi.
    \texttt{testCase2a}
    Oczekiwany wynik: 15
    Uzyskany wynik: 3
    \textsc{Test nie zaliczony}

    \texttt{testCase2b}
    Oczekiwany wynik: Exception
    Uzyskany wynik: brak wyjątku
    \textsc{Test nie zaliczony}

    \item Sieć z 2 węzłami pośrednimi połączonymi szeregowo z krawędzią o
    zerowej przepustowości pomiędzy węzłami pośrednimi.
    \texttt{testCase3}
    Oczekiwany wynik: 0
    Uzyskany wynik: 0
    \textsc{Test zaliczony}

    \item Sieć z 2 węzłami pośrednimi połączonymi szeregowo z krawędzią o
    ujemnej przepustowości pomiędzy węzłami pośrednimi.
    \texttt{testCase4}
    Oczekiwany wynik: 0
    Uzyskany wynik: 0
    \textsc{Test zaliczony}

    \item Sieć z 2 węzłami pośrednimi połączonymi szeregowo z krawędzią z
    drugiego węzłą pośredniego do pierwszego węzła pośredniego.
    \texttt{testCase5}
    Oczekiwany wynik: 4
    Uzyskany wynik: 4
    \textsc{Test zaliczony}

    \item Sieć z 2 węzłami pośrednimi połączonymi szeregowo z krawędzią od
    pierwszego węzła pośredniego do ujścia.
    \texttt{testCase6}
    Oczekiwany wynik: 7
    Uzyskany wynik: 7
    \textsc{Test zaliczony}


    \item Sieć z 2 węzłami pośrednimi połączonymi szeregowo z krawędzią od
    źródła do drugiego węzła pośredniego.
    \texttt{testCase7}
    Oczekiwany wynik: 9
    Uzyskany wynik: 9
    \textsc{Test zaliczony}


\end{itemize}


\subsubsection{Sieć z nieistniejącymi węzłami}


\subsubsection{Sieć z ujemną liczbą węzłów}
Częściowo zrealizowano w klasie NegativeSizeNetworkTest Nie jest możliwe
stworzenie sieci z ujemną ilością węzłów.

\subsubsection{Sieć z $1e9$ węzłów}
Ograniczono wielkość testowanej sieci do $1e3$ węzłów, zrealizowano w klasie
HugeFlowNetworkTest Ręczne stworzenie sieci przepływu tej wielkości jest
nierealne. Metoda generateEdges z klasy FlowNetworkGenerator ma złożoność
rzędu $O(n^3)$, co znacząco ogranicza możliwości w tym zakresie.
%%n^3 poprawić notację
