\subsection{Testy czarno\dywiz skrzynkowe}



\subsubsection{Sieć bez węzłów}
Wszystkie testy na sieci bez węzłów zrealizowano w klasie EmptyNetworkTest.

\texttt{testCase1}
Liczba węzłów w sieci:0
Indeks źródła: 2
Indeks ujścia: 5
Krawędzie: NULL
Oczekiwany wynik: 0

\texttt{testCase2}
Liczba węzłów w sieci: 0
Indeks źródła: 0
Indeks ujścia: 0
Krawędzie: NULL
Oczekiwany wynik: Exception

\emph{Algorytm poprawnie nie znajduje niezerowego przepływu maksymalnego w sieci
z samym źródłem.}

\subsubsection{Sieć z 1 węzłem}
Wszystkie testy zrealizowane w klasie JustOneElementTest.
\begin{itemize}[nosep]
    \item Samego źródła.
    \texttt{testCase1}
    Liczba węzłów w sieci: 1
    Indeks źródła: 0
    Indeks ujścia: 1
    Krawędzie: NULL
    Oczekiwany wynik: 0

    \emph{Algorytm poprawnie nie znajduje niezerowego przepływu maksymalnego w sieci
    z samym źródłem.}

    \item Węzła pośredniego.
    \texttt{testCase2}
    Liczba węzłów w sieci: 1
    Indeks źródła: 2
    Indeks ujścia: 1
    Krawędzie: NULL
    Oczekiwany wynik: Exception

    \emph{Algorytm poprawnie zwraca wyjątek przy próbie odwołania się
    do nieistniejącego węzła.}

    \item Ujścia.
    \texttt{testCase3}
    Liczba węzłów w sieci: 1
    Indeks źródła: 1
    Indeks ujścia: 0
    Krawędzie: NULL
    Oczekiwany wynik: Exception

    \emph{Algorytm poprawnie zwraca wyjątek przy próbie odwołania się
    do nieistniejącego węzła.}

    \item Wspólnego źródła i ujścia.
    \texttt{testCase4}
    Liczba węzłów w sieci: 1
    Indeks źródła: 0
    Indeks ujścia: 0
    Krawędzie: NULL
    Oczekiwany wynik: 0

    \emph{Algorytm poprawnie nie znajduje niezerowego
    przepływu maksymalnego w sieci ze źródłem i ujściem w tym samym punkcie.}

    \item Wspólnego źródła i ujścia z pętlą.
    \texttt{testCase5}
    Liczba węzłów w sieci: 1
    Indeks źródła: 0
    Indeks ujścia: 1
    Krawędzie:
    \begin{enumerate}[nosep]
        \item krawędź:
        początek: 0
        koniec: 0
        przepustowość: 5
    \end{enumerate}
    Oczekiwany wynik: 0

    \emph{Algorytm poprawnie nie znajduje niezerowego
    przepływu maksymalnego w sieci ze źródłem i ujściem w tym samym punkcie z pętlą.}
\end{itemize}


\subsubsection{Sieć bez węzłów pośrednich}
Testy na sieciach bez węzłów pośrednich zrealizowano w klasie JustSourceAndSinkTest.
\begin{itemize}[nosep]
    \item Nie połączonymi żadną krawędzią.
    \texttt{testCase1}
    Liczba węzłów w sieci: 2
    Indeks źródła: 0
    Indeks ujścia: 1
    Krawędzie: NULL
    Oczekiwany wynik: 0

    \emph{Algorytm poprawnie nie znajduje niezerowego
    przepływu maksymalnego w sieci bez ścieżki od źródła do ujścia.}

    \item Połączonymi pojedynczą krawędzią skierowaną od źródła do ujścia o dodatniej przepustowości.
    \texttt{testCase2}
    Liczba węzłów w sieci: 2
    Indeks źródła: 0
    Indeks ujścia: 1
    Krawędzie:
    \begin{enumerate}[nosep]
        \item krawędź:
        początek: 0
        koniec: 1
        przepustowość: 9
    \end{enumerate}
    Oczekiwany wynik: 9
    \emph{Algorytm poprawnie znajduje maksymalny przepływ w sieci.}

    \item Połączonymi pojedynczą krawędzią skierowaną od źródła do ujścia o dodatniej przepustowości z pętlą o dodatniej przepustowości w ujściu.
    \texttt{testCase3a}
    Liczba węzłów w sieci: 2
    Indeks źródła: 0
    Indeks ujścia: 1
    Krawędzie:
    \begin{enumerate}[nosep]
        \item krawędź:
        początek: 0
        koniec: 1
        przepustowość: 57956756
        \item krawędź:
        początek: 1
        koniec: 1
        przepustowość: 6363
    \end{enumerate}
    Oczekiwany wynik: 57956756

    \texttt{testCase3b}
    Liczba węzłów w sieci: 2
    Indeks źródła: 0
    Indeks ujścia: 1
    Krawędzie:
    \begin{enumerate}[nosep]
        \item krawędź:
        początek: 0
        koniec: 1
        przepustowość: 12
        \item krawędź:
        początek: 1
        koniec: 1
        przepustowość: 2
    \end{enumerate}
    Oczekiwany wynik: Exception

    \emph{Algorytm poprawnie znajduje maksymalny przepływ
w sieci. Algorytm nie informuje o istnieniu pętli w sieci.}

    \item Połączonymi pojedynczą krawędzią skierowaną od źródła do ujścia o dodatniej przepustowości z pętlą o ujemnej przepustowości w źródle.
    \texttt{testCase4a}
    Liczba węzłów w sieci: 2
    Indeks źródła: 0
    Indeks ujścia: 1
    Krawędzie:
    \begin{enumerate}[nosep]
        \item krawędź:
        początek: 0
        koniec: 1
        przepustowość: 346723
        \item krawędź:
        początek: 0
        koniec: 0
        przepustowość: -1623474
    \end{enumerate}
    Oczekiwany wynik: 346723

    \texttt{testCase4b}
    Liczba węzłów w sieci: 2
    Indeks źródła: 0
    Indeks ujścia: 1
    Krawędzie:
    \begin{enumerate}[nosep]
        \item krawędź:
        początek: 0
        koniec: 1
        przepustowość: 4
        \item krawędź:
        początek: 1
        koniec: 1
        przepustowość: -2
    \end{enumerate}
    Oczekiwany wynik: Exception

    \emph{Algorytm poprawnie znajduje maksymalny przepływ
    w sieci. Algorytm nie podnosi wyjątku przy pętli o ujemnej przepustowości.}

    \item Połączonymi pojedynczą krawędzią skierowaną od źródła do ujścia o zerowej przepustowości.
    \texttt{testCase5}
    Liczba węzłów w sieci: 2
    Indeks źródła: 0
    Indeks ujścia: 1
    Krawędzie:
    \begin{enumerate}[nosep]
        \item krawędź:
        początek: 0
        koniec: 1
        przepustowość: 0
    \end{enumerate}
    Oczekiwany wynik: 0

    \emph{Algorytm poprawnie nie znajduje niezerowego przepływu
    maksymalnego w sieci bez ścieżki od źródła do ujścia.}

    \item Połączonymi pojedynczą krawędzią skierowaną od źródła do ujścia o ujemnej przepustowości.
    \texttt{testCase6a}
    Liczba węzłów w sieci: 2
    Indeks źródła: 0
    Indeks ujścia: 1
    Krawędzie:
    \begin{enumerate}[nosep]
        \item krawędź:
        początek: 0
        koniec: 1
        przepustowość: -5
    \end{enumerate}
    Oczekiwany wynik: 0

    \texttt{testCase6b}
    Liczba węzłów w sieci: 2
    Indeks źródła: 0
    Indeks ujścia: 1
    Krawędzie:
    \begin{enumerate}[nosep]
        \item krawędź:
        początek: 0
        koniec: 1
        przepustowość: -173848
    \end{enumerate}
    Oczekiwany wynik: Exception

    \emph{Algorytm poprawnie nie znajduje
    dodatniego przepływu maksymalnego w sieci bez ścieżki od źródła do ujścia. Poprawnie
    podnoszony jest wyjątek dla krawędzi o ujemnej przepustowości.}

    \item Połączonymi pojedynczą krawędzią skierowaną od ujścia do źródła o dodatniej przepustowości.
    \texttt{testCase7}
    Liczba węzłów w sieci: 2
    Indeks źródła: 0
    Indeks ujścia: 1
    Krawędzie:
    \begin{enumerate}[nosep]
        \item krawędź:
        początek: 1
        koniec: 0
        przepustowość: 5
    \end{enumerate}
    Oczekiwany wynik: 0

    \emph{Algorytm poprawnie nie znajduje niezerowego
    przepływu maksymalnego w sieci bez ścieżki od źródła do ujścia.}

    \item Połączonymi pojedynczą krawędzią skierowaną od ujścia do źródła o ujemnej przepustowości.
    \texttt{testCase8a}
    Liczba węzłów w sieci: 2
    Indeks źródła: 0
    Indeks ujścia: 1
    Krawędzie:
    \begin{enumerate}[nosep]
        \item krawędź:
        początek: 1
        koniec: 0
        przepustowość: -4
    \end{enumerate}
    Oczekiwany wynik: 0

    \texttt{testCase8b}
    Liczba węzłów w sieci: 2
    Indeks źródła: 0
    Indeks ujścia: 1
    Krawędzie:
    \begin{enumerate}[nosep]
        \item krawędź:
        początek: 1
        koniec: 0
        przepustowość: -1639273
    \end{enumerate}
    Oczekiwany wynik: Exception

    \emph{Algorytm poprawnie nie znajduje
    niezerowego przepływu maksymalnego w sieci bez ścieżki o dodatniej
    przepustowości od źródła do ujścia. Nie jest podnoszony wyjątek
    w związku z występowaniem krawędzi o ujemnej przepustowości.}

    \item Z wieloma krawędziami skierowanymi od żródła do ujścia.
    \texttt{testCase9a}
    Liczba węzłów w sieci: 2
    Indeks źródła: 0
    Indeks ujścia: 1
    Krawędzie:
    \begin{enumerate}[nosep]
        \item krawędź:
        początek: 0
        koniec: 1
        przepustowość: 12
        \item krawędź:
        początek: 0
        koniec: 1
        przepustowość: 2
        \item krawędź:
        początek: 0
        koniec: 1
        przepustowość: 4
    \end{enumerate}
    Oczekiwany wynik: 18

    \texttt{testCase9b}
    Liczba węzłów w sieci: 2
    Indeks źródła: 0
    Indeks ujścia: 1
    Krawędzie:
    \begin{enumerate}[nosep]
        \item krawędź:
        początek: 1
        koniec: 0
        przepustowość: 9
        \item krawędź:
        początek: 1
        koniec: 0
        przepustowość: 3
        \item krawędź:
        początek: 1
        koniec: 0
        przepustowość: 15
    \end{enumerate}
    Oczekiwany wynik: Exception

    \emph{Algorytm nie odczytuje
    poprawnie przepustowości ze zwielokrotnionych krawędzi. Nie jest
    podnoszony wyjątek w związku z występowaniem zwielokrotnionych krawędzi.}

    \item Z wieloma krawędziami skierowanymi od ujścia do źródła.
    \texttt{testCase10a}
    Liczba węzłów w sieci: 2
    Indeks źródła: 0
    Indeks ujścia: 1
    Krawędzie:
    \begin{enumerate}[nosep]
        \item krawędź:
        początek: 0
        koniec: 1
        przepustowość: -16
        \item krawędź:
        początek: 0
        koniec: 1
        przepustowość: -5
        \item krawędź:
        początek: 0
        koniec: 1
        przepustowość: -3
    \end{enumerate}
    Oczekiwany wynik: 0

    \texttt{testCase10b}
    Liczba węzłów w sieci: 2
    Indeks źródła: 0
    Indeks ujścia: 1
    Krawędzie:
    \begin{enumerate}[nosep]
        \item krawędź:
        początek: 1
        koniec: 0
        przepustowość: -7
        \item krawędź:
        początek: 1
        koniec: 0
        przepustowość: -1
    \end{enumerate}
    Oczekiwany wynik: Exception

    \emph{Algorytm poprawnie
    nie znajduje niezerowego przepływu maksymalnego w sieci bez ścieżki
    o dodatniej przepustowości od źródła do ujścia. Nie jest podnoszony
    wyjątek mimo występowania krawędzi o ujemnej przepustowości i krawędzi
    zwielokrotnionych.}

    \item Z wieloma krawędziami skierowanymi w różnych stronach.
    \texttt{testCase11a}
    Liczba węzłów w sieci: 2
    Indeks źródła: 0
    Indeks ujścia: 1
    Krawędzie:
    \begin{enumerate}[nosep]
        \item krawędź:
        początek: 0
        koniec: 1
        przepustowość: 4
        \item krawędź:
        początek: 1
        koniec: 0
        przepustowość: 3
    \end{enumerate}
    Oczekiwany wynik: 4

    \texttt{testCase11b}
    Liczba węzłów w sieci: 2
    Indeks źródła: 0
    Indeks ujścia: 1
    Krawędzie:
    \begin{enumerate}[nosep]
        \item krawędź:
        początek: 0
        koniec: 1
        przepustowość: 6
        \item krawędź:
        początek: 1
        koniec: 0
        przepustowość: 11
        \item krawędź:
        początek: 0
        koniec: 1
        przepustowość: 7
        \item krawędź:
        początek: 1
        koniec: 0
        przepustowość: 2
    \end{enumerate}
    Oczekiwany wynik: 13

    \texttt{testCase11c}
    Liczba węzłów w sieci: 2
    Indeks źródła: 0
    Indeks ujścia: 1
    Krawędzie:
    \begin{enumerate}[nosep]
        \item krawędź:
        początek: 0
        koniec: 1
        przepustowość: 15
        \item krawędź:
        początek: 1
        koniec: 0
        przepustowość: 9
        \item krawędź:
        początek: 0
        koniec: 1
        przepustowość: 7
        \item krawędź:
        początek: 1
        koniec: 0
        przepustowość: 4
    \end{enumerate}
    Oczekiwany wynik: Exception

    \emph{Algorytm poprawnie znajduje przepływ maksymalny w sieci z dodatkową krawędzią
    o tym samym kierunku, a przeciwnym zwrocie. W przypadku zwielokrotnienia,
    którejkolwiek z tych krawędzi, algorytm zwraca nieprawidłową wartość
    przepływu, nie podnosząc wyjątku w związku z występowaniem krawędzi
    o ujemnej przepustowości.}

\end{itemize}


\subsubsection{Sieć z 1 węzłem pośrednim}
Testy na sieciach z jednym węzłem pośrednim zostały zgrupowane
w klasie SingleVertexTest.
\begin{itemize}[nosep]
    \item Połączone pojedynczymi krawędziami skierowanymi ze źródła do węzłą
    pośredniego i z węzła pośredniego do ujścia.
    \texttt{testCase1a}
    Liczba węzłów w sieci: 3
    Indeks źródła: 0
    Indeks ujścia: 2
    Krawędzie:
    \begin{enumerate}[nosep]
        \item krawędź:
        początek: 0
        koniec: 1
        przepustowość: 512
        \item krawędź:
        początek: 1
        koniec: 2
        przepustowość: 126
    \end{enumerate}
    Oczekiwany wynik: 126

    \texttt{testCase1b}
    Liczba węzłów w sieci: 3
    Indeks źródła: 0
    Indeks ujścia: 2
    Krawędzie:
    \begin{enumerate}[nosep]
        \item krawędź:
        początek: 0
        koniec: 1
        przepustowość: 104526
        \item krawędź:
        początek: 1
        koniec: 2
        przepustowość: 75269
        \item krawędź:
        początek: 1
        koniec: 0
        przepustowość: 1523
    \end{enumerate}
    Oczekiwany wynik: 75269

    \emph{Algorytm poprawnie znajduje maksymalny przepływ w sieci.}
    % Z -> P -> U

    \item Z pojedynczymi krawędziami skierowanymi z ujścia do węzła pośredniego
    i z węzła pośredniego do źródła.
    \texttt{testCase2}
    Liczba węzłów w sieci: 3
    Indeks źródła: 0
    Indeks ujścia: 2
    Krawędzie:
    \begin{enumerate}[nosep]
        \item krawędź:
        początek: 1
        koniec: 0
        przepustowość: 43
        \item krawędź:
        początek: 2
        koniec: 1
        przepustowość: 76
    \end{enumerate}
    Oczekiwany wynik: 0
    \emph{Algorytm poprawnie nie znajduje niezerowego przepływu maksymalnego w sieci bez ścieżki od źródła do ujścia.}
    % Z <- P <- U

    \item Z pojedynczymi krawędziami skierowanymi z węzła pośredniego do ujścia
    i z węzła pośredniego do źródła.
    \texttt{testCase3}
    Liczba węzłów w sieci: 3
    Indeks źródła: 0
    Indeks ujścia: 2
    Krawędzie:
    \begin{enumerate}[nosep]
        \item krawędź:
        początek: 1
        koniec: 0
        przepustowość: 325
        \item krawędź:
        początek: 1
        koniec: 2
        przepustowość: 12
    \end{enumerate}
    Oczekiwany wynik: 0
    \emph{Algorytm poprawnie nie znajduje niezerowego przepływu maksymalnego w sieci bez ścieżki od źródła do ujścia.}
    % Z <- P -> U

    \item Z pojedynczymi krawędziami skierowanymi ze źródła do węzła pośredniego
    i z ujścia do węzła pośredniego.
    \texttt{testCase4}
    Liczba węzłów w sieci: 3
    Indeks źródła: 0
    Indeks ujścia: 2
    Krawędzie:
    \begin{enumerate}[nosep]
        \item krawędź:
        początek: 0
        koniec: 1
        przepustowość: 9
        \item krawędź:
        początek: 2
        koniec: 1
        przepustowość: 13
    \end{enumerate}
    Oczekiwany wynik: 0
    \emph{Algorytm poprawnie nie znajduje niezerowego przepływu maksymalnego w sieci bez ścieżki od źródła do ujścia.}
    % Z -> P <- U

    \item Połączone pojedynczymi krawędziami skierowanymi ze źródła do węzła
    pośredniego i z węzła pośredniego do źródła z dodatkową krawędzią o
    dodatniej przepustowości ze źródła do ujścia.
    \texttt{testCase5}
    Liczba węzłów w sieci: 3
    Indeks źródła: 0
    Indeks ujścia: 2
    Krawędzie:
    \begin{enumerate}[nosep]
        \item krawędź:
        początek: 0
        koniec: 1
        przepustowość: 107209
        \item krawędź:
        początek: 1
        koniec: 2
        przepustowość: 75269
        \item krawędź:
        początek: 0
        koniec: 2
        przepustowość: 8301
    \end{enumerate}
    Oczekiwany wynik: 83570

    \emph{Algorytm poprawnie znajduje maksymalny przepływ w sieci.}
    % Z -> P -> U
    % |_________^

    \item Połączone pojedynczymi krawędziami skierowanymi ze źródła do węzła
    pośredniego i z węzła pośredniego do źródła z pętlą o dodatniej
    przepustowości w węźle pośrednim.
    \texttt{testCase6a}
    Liczba węzłów w sieci: 3
    Indeks źródła: 0
    Indeks ujścia: 2
    Krawędzie:
    \begin{enumerate}[nosep]
        \item krawędź:
        początek: 0
        koniec: 1
        przepustowość: 73
        \item krawędź:
        początek: 1
        koniec: 2
        przepustowość: 17
        \item krawędź:
        początek: 1
        koniec: 1
        przepustowość: 345
    \end{enumerate}
    Oczekiwany wynik: 17

    \texttt{testCase6b}
    Liczba węzłów w sieci: 3
    Indeks źródła: 0
    Indeks ujścia: 2
    Krawędzie:
    \begin{enumerate}[nosep]
        \item krawędź:
        początek: 0
        koniec: 1
        przepustowość: 73
        \item krawędź:
        początek: 1
        koniec: 2
        przepustowość: 17
        \item krawędź:
        początek: 1
        koniec: 1
        przepustowość: 345
    \end{enumerate}
    Oczekiwany wynik: Exception

    \emph{Algorytm poprawnie znajduje maksymalny przepływ w sieci. Algorytm nie informuje o istnieniu pętli w sieci.}
    % Z -> P -> U
    %     / ^
    %     \_|

    \item Połączone zwielokrotnionymi krawędziami ze źródła do węzła pośredniego
    i z węzła pośredniego do źródła, z mieszanymi zwrotami.
    \texttt{testCase7a}
    Liczba węzłów w sieci: 3
    Indeks źródła: 0
    Indeks ujścia: 2
    Krawędzie:
    \begin{enumerate}[nosep]
        \item krawędź:
        początek: 0
        koniec: 1
        przepustowość: 5
        \item krawędź:
        początek: 1
        koniec: 2
        przepustowość: 7
        \item krawędź:
        początek: 1
        koniec: 0
        przepustowość: 12
        \item krawędź:
        początek: 2
        koniec: 1
        przepustowość: 3
    \end{enumerate}
    Oczekiwany wynik: 5

    \texttt{testCase7b}
    Liczba węzłów w sieci: 3
    Indeks źródła: 0
    Indeks ujścia: 2
    Krawędzie:
    \begin{enumerate}[nosep]
        \item krawędź:
        początek: 0
        koniec: 1
        przepustowość: 16
        \item krawędź:
        początek: 1
        koniec: 2
        przepustowość: 8
        \item krawędź:
        początek: 2
        koniec: 1
        przepustowość: 2
        \item krawędź:
        początek: 1
        koniec: 2
        przepustowość: 13
    \end{enumerate}
    Oczekiwany wynik: Exception

    \texttt{testCase7c}
    Liczba węzłów w sieci: 3
    Indeks źródła: 0
    Indeks ujścia: 2
    Krawędzie:
    \begin{enumerate}[nosep]
        \item krawędź:
        początek: 0
        koniec: 1
        przepustowość: 16
        \item krawędź:
        początek: 1
        koniec: 2
        przepustowość: 8
        \item krawędź:
        początek: 1
        koniec:0
        przepustowość: 5
        \item krawędź:
        początek: 2
        koniec: 1
        przepustowość: 2
        \item krawędź:
        początek: 1
        koniec: 2
        przepustowość: 13
    \end{enumerate}
    Oczekiwany wynik: 16

    \emph{Algorytm poprawnie znajduje maksymalny przepływ w sieci o pojedynczych krawędziach. Konstruktor ani funkcja validate() z klasy FlowNetworkArray nie podnoszą wyjątków. W przypadku zwielokrotnienia krawędzi algorytm nie jest w stanie wyznaczyć prawidłowego maksymalnego przepływu w sieci.}
    % Z -> P -> U
    %  ^__/ ^__/

    \item Połączone pojedynczą krawędzią skierowaną ze źródła do węzła
    pośredniego, bez krawędzi do ujścia.

    \texttt{testCase8}
    Liczba węzłów w sieci: 3
    Indeks źródła: 0
    Indeks ujścia: 2
    Krawędzie:
    \begin{enumerate}[nosep]
        \item krawędź:
        początek: 0
        koniec: 1
        przepustowość: 3
    \end{enumerate}
    Oczekiwany wynik: 0

    \emph{Algorytm poprawnie nie znajduje niezerowego przepływu maksymalnego w sieci bez ścieżki od źródła do ujścia.}
    % Z -> P    U

    \item Połączone pojedynczą krawędzia skierowaną ze źródła do węzła
    pośredniego, oraz krawędzią ze źródłą do ujścia.
    \texttt{testCase9}
    Liczba węzłów w sieci: 3
    Indeks źródła: 0
    Indeks ujścia: 2
    Krawędzie:
    \begin{enumerate}[nosep]
        \item krawędź:
        początek: 0
        koniec: 1
        przepustowość: 13
        \item krawędź:
        początek: 0
        koniec: 2
        przepustowość: 5
    \end{enumerate}
    Oczekiwany wynik: 5

    \emph{Algorytm poprawnie znajduje maksymalny przepływ w sieci.}
    % Z -> P    U
    % |_________^

    \item Połączone pojedynczą krawędzia skierowaną z węzła pośredniego do
    ujścia, bez połączenia ze źródłem.
    \texttt{testCase10}
    Liczba węzłów w sieci: 3
    Indeks źródła: 0
    Indeks ujścia: 2
    Krawędzie:
    \begin{enumerate}[nosep]
        \item krawędź:
        początek: 1
        koniec: 2
        przepustowość: 7
    \end{enumerate}
    Oczekiwany wynik: 0

    \emph{Algorytm poprawnie nie znajduje niezerowego przepływu maksymalnego w sieci bez ścieżki od źródła do ujścia.}
    % Z    P -> U
\end{itemize}


\subsubsection{Sieć z 2 węzłami pośrednimi połączonymi równolegle}
Testy operujące na sieciach z 2 węzłami pośrednimi połączonymi równolegle zostały
zebrane w klasie TwoParallelVerticesTest.
\begin{itemize}[nosep]
    \item Z krawędziami o dodatniej przepustowości ze źródła do obu węzłów
    pośrednich i z obu węzłów pośrednich do ujścia.
    \texttt{testCase1}
    Liczba węzłów w sieci: 4
    Indeks źródła: 0
    Indeks ujścia: 3
    Krawędzie:
    \begin{enumerate}[nosep]
        \item krawędź:
        początek: 0
        koniec: 1
        przepustowość: 1
        \item krawędź:
        początek: 0
        koniec: 2
        przepustowość: 4
        \item krawędź:
        początek: 1
        koniec: 3
        przepustowość: 2
        \item krawędź:
        początek: 2
        koniec: 3
        przepustowość: 3
    \end{enumerate}
    Oczekiwany wynik: 4

    \emph{Algorytm poprawnie znajduje maksymalny przepływ.}

    \item Z dodatkową krawędzią o dodatniej przepustowości z pierwszego węzła
    pośredniego do drugiego węzła pośredniego.
    \texttt{testCase2}
    Liczba węzłów w sieci: 4
    Indeks źródła: 0
    Indeks ujścia: 3
    Krawędzie:
    \begin{enumerate}[nosep]
        \item krawędź:
        początek: 0
        koniec: 1
        przepustowość: 7
        \item krawędź:
        początek: 0
        koniec: 2
        przepustowość: 1
        \item krawędź:
        początek: 1
        koniec: 3
        przepustowość: 5
        \item krawędź:
        początek: 2
        koniec: 3
        przepustowość: 3
        \item krawędź:
        początek: 1
        koniec: 2
        przepustowość: 9
    \end{enumerate}
    Oczekiwany wynik: 8

    \emph{Algorytm poprawnie znajduje maksymalny przepływ.}

    \item Z dodatkową krawędzią o dodatniej przepustowości z drugiego węzła
    pośredniego do pierwszego węzła pośredniego.
    \texttt{testCase3}
    Liczba węzłów w sieci: 4
    Indeks źródła: 0
    Indeks ujścia: 3
    Krawędzie:
    \begin{enumerate}[nosep]
        \item krawędź:
        początek: 0
        koniec: 1
        przepustowość: 3
        \item krawędź:
        początek: 0
        koniec: 2
        przepustowość: 6
        \item krawędź:
        początek: 1
        koniec: 3
        przepustowość: 10
        \item krawędź:
        początek: 2
        koniec: 3
        przepustowość: 4
        \item krawędź:
        początek: 2
        koniec: 1
        przepustowość: 2
    \end{enumerate}
    Oczekiwany wynik: 9

    \emph{Algorytm poprawnie znajduje maksymalny przepływ.}

    \item Z dodatkowymi krawędziami o dodatniej przepustowości z pierwszego
    węzła pośredniego do drugiego węzła pośredniego i z drugiego węzła
    pośredniego do pierwszego węzła pośredniego.
    \texttt{testCase4}
    Liczba węzłów w sieci: 4
    Indeks źródła: 0
    Indeks ujścia: 3
    Krawędzie:
    \begin{enumerate}[nosep]
        \item krawędź:
        początek: 0
        koniec: 1
        przepustowość: 17
        \item krawędź:
        początek: 0
        koniec: 2
        przepustowość: 11
        \item krawędź:
        początek: 1
        koniec: 3
        przepustowość: 5
        \item krawędź:
        początek: 2
        koniec: 3
        przepustowość: 3
        \item krawędź:
        początek: 2
        koniec: 1
        przepustowość: 9
        \item krawędź:
        początek: 1
        koniec: 2
        przepustowość: 6
    \end{enumerate}
    Oczekiwany wynik: 8

    \emph{Algorytm poprawnie znajduje maksymalny przepływ.}

    \item Z dodatkową krawędzią o dodatniej przepustowości ze źródła do ujścia.
    \texttt{testCase5}
    Liczba węzłów w sieci: 4
    Indeks źródła: 0
    Indeks ujścia: 3
    Krawędzie:
    \begin{enumerate}[nosep]
        \item krawędź:
        początek: 0
        koniec: 1
        przepustowość: 5
        \item krawędź:
        początek: 0
        koniec: 2
        przepustowość: 3
        \item krawędź:
        początek: 1
        koniec: 3
        przepustowość: 4
        \item krawędź:
        początek: 2
        koniec: 3
        przepustowość: 9
        \item krawędź:
        początek: 0
        koniec: 3
        przepustowość: 3
    \end{enumerate}
    Oczekiwany wynik: 10

    \emph{Algorytm poprawnie znajduje maksymalny przepływ.}

    \item Z dodatkową krawędzia o dodatniej przepustowości z ujścia do źródła.
    \texttt{testCase6}
    Liczba węzłów w sieci: 4
    Indeks źródła: 0
    Indeks ujścia: 3
    Krawędzie:
    \begin{enumerate}[nosep]
        \item krawędź:
        początek: 0
        koniec: 1
        przepustowość: 2
        \item krawędź:
        początek: 0
        koniec: 2
        przepustowość: 5
        \item krawędź:
        początek: 1
        koniec: 3
        przepustowość: 8
        \item krawędź:
        początek: 2
        koniec: 3
        przepustowość: 1
        \item krawędź:
        początek: 3
        koniec: 0
        przepustowość: 4
    \end{enumerate}
    Oczekiwany wynik: 3

    \emph{Algorytm poprawnie znajduje maksymalny przepływ.}

    \item Z dodatkową krawędzią o dodatniej przepustowości z ujścia do
    pierwszego węzła pośredniego.
    \texttt{testCase7}
    Liczba węzłów w sieci: 4
    Indeks źródła: 0
    Indeks ujścia: 3
    Krawędzie:
    \begin{enumerate}[nosep]
        \item krawędź:
        początek: 0
        koniec: 1
        przepustowość: 1
        \item krawędź:
        początek: 0
        koniec: 2
        przepustowość: 9
        \item krawędź:
        początek: 1
        koniec: 3
        przepustowość: 12
        \item krawędź:
        początek: 2
        koniec: 3
        przepustowość: 4
        \item krawędź:
        początek: 3
        koniec: 1
        przepustowość: 3
    \end{enumerate}
    Oczekiwany wynik: 5

    \emph{Algorytm poprawnie znajduje maksymalny przepływ.}

    \item Z dodatkową krawędzią o dodatniej przepustowości z pierwszego węzła
    pośredniego do źródłą.
    \texttt{testCase8}
    Liczba węzłów w sieci: 4
    Indeks źródła: 0
    Indeks ujścia: 3
    Krawędzie:
    \begin{enumerate}[nosep]
        \item krawędź:
        początek: 0
        koniec: 1
        przepustowość: 8
        \item krawędź:
        początek: 0
        koniec: 2
        przepustowość: 3
        \item krawędź:
        początek: 1
        koniec: 3
        przepustowość: 5
        \item krawędź:
        początek: 2
        koniec: 3
        przepustowość: 7
        \item krawędź:
        początek: 2
        koniec: 1
        przepustowość: 6
    \end{enumerate}
    Oczekiwany wynik: 8

    \emph{Algorytm poprawnie znajduje maksymalny przepływ.}

    \item Z dodatkowymi krawędziami o dodatniej przepustowości z ujścia do
    pierwszego węzła pośredniego i z pierwszego węzłą pośredniego do źródła.
    \texttt{testCase9}
    Liczba węzłów w sieci: 4
    Indeks źródła: 0
    Indeks ujścia: 3
    Krawędzie:
    \begin{enumerate}[nosep]
        \item krawędź:
        początek: 0
        koniec: 1
        przepustowość: 7
        \item krawędź:
        początek: 0
        koniec: 2
        przepustowość: 8
        \item krawędź:
        początek: 1
        koniec: 3
        przepustowość: 2
        \item krawędź:
        początek: 2
        koniec: 3
        przepustowość: 3
        \item krawędź:
        początek: 3
        koniec: 1
        przepustowość: 11
        \item krawędź:
        początek: 1
        koniec: 0
        przepustowość: 8
    \end{enumerate}
    Oczekiwany wynik: 5
    \emph{Algorytm poprawnie znajduje maksymalny przepływ.}
\end{itemize}


\subsubsection{Sieć z 2 węzłami pośrednimi połączonymi szeregowo}
Testy wykorzystujące sieci z 2 węzłami pośrednimi znajdują się z kolei w klasie
TwoSerialVerticesTest.
\begin{itemize}[nosep]
    \item Z krawędziami o dodatniej przepustowości ze źródła do pierwszego węzła
    pośredniego, z pierwszego węzła pośredniego do drugiego węzła pośredniego i z
    drugiego węzła pośredniego do ujścia.
    \texttt{testCase1}
    Liczba węzłów w sieci: 4
    Indeks źródła: 0
    Indeks ujścia: 3
    Krawędzie:
    \begin{enumerate}[nosep]
        \item krawędź:
        początek: 0
        koniec: 1
        przepustowość: 5
        \item krawędź:
        początek: 1
        koniec: 2
        przepustowość: 3
        \item krawędź:
        początek: 2
        koniec: 3
        przepustowość: 7
    \end{enumerate}
    Oczekiwany wynik: 15

    \emph{Algorytm poprawnie znajduje maksymalny przepływ w sieci.}

    \item Sieć z 2 węzłami pośrednimi połączonymi szeregowo z wieloma
    krawędziami pomiędzy węzłami pośrednimi.
    \texttt{testCase2a}
    Liczba węzłów w sieci: 4
    Indeks źródła: 0
    Indeks ujścia: 3
    Krawędzie:
    \begin{enumerate}[nosep]
        \item krawędź:
        początek: 0
        koniec: 1
        przepustowość: 15
        \item krawędź:
        początek: 1
        koniec: 2
        przepustowość: 1
        \item krawędź:
        początek: 2
        koniec: 3
        przepustowość: 21
        \item krawędź:
        początek: 1
        koniec: 2
        przepustowość: 5
        \item krawędź:
        początek: 1
        koniec: 2
        przepustowość: 7
        \item krawędź:
        początek: 1
        koniec: 2
        przepustowość: 3
    \end{enumerate}
    Oczekiwany wynik: 15

    \texttt{testCase2a}
    Liczba węzłów w sieci: 4
    Indeks źródła: 0
    Indeks ujścia: 3
    Krawędzie:
    \begin{enumerate}[nosep]
        \item krawędź:
        początek: 0
        koniec: 1
        przepustowość: 15
        \item krawędź:
        początek: 1
        koniec: 2
        przepustowość: 1
        \item krawędź:
        początek: 2
        koniec: 3
        przepustowość: 21
        \item krawędź:
        początek: 1
        koniec: 2
        przepustowość: 5
        \item krawędź:
        początek: 1
        koniec: 2
        przepustowość: 7
        \item krawędź:
        początek: 1
        koniec: 2
        przepustowość: 3
    \end{enumerate}
    Oczekiwany wynik: 15

    \texttt{testCase2a}
    Liczba węzłów w sieci: 4
    Indeks źródła: 0
    Indeks ujścia: 3
    Krawędzie:
    \begin{enumerate}[nosep]
        \item krawędź:
        początek: 0
        koniec: 1
        przepustowość: 15
        \item krawędź:
        początek: 1
        koniec: 2
        przepustowość: 1
        \item krawędź:
        początek: 2
        koniec: 3
        przepustowość: 21
        \item krawędź:
        początek: 1
        koniec: 2
        przepustowość: 5
        \item krawędź:
        początek: 1
        koniec: 2
        przepustowość: 7
        \item krawędź:
        początek: 1
        koniec: 2
        przepustowość: 3
    \end{enumerate}
    Oczekiwany wynik: Exception

    \emph{Zwraca niepoprawną wartość. Konstruktor ani funkcja validate() z klasy FlowNetworkArray nie podnoszą wyjątków.}

    \item Sieć z 2 węzłami pośrednimi połączonymi szeregowo z krawędzią o
    zerowej przepustowości pomiędzy węzłami pośrednimi.
    \texttt{testCase3}
    Liczba węzłów w sieci: 4
    Indeks źródła: 0
    Indeks ujścia: 3
    Krawędzie:
    \begin{enumerate}[nosep]
        \item krawędź:
        początek: 0
        koniec: 1
        przepustowość: 3
        \item krawędź:
        początek: 1
        koniec: 2
        przepustowość: 0
        \item krawędź:
        początek: 2
        koniec: 3
        przepustowość: 4
    \end{enumerate}
    Oczekiwany wynik: 0

    \emph{Poprawnie nie znajduje niezerowego maksymalnego przepływu w sieci o zerowym przepływie maksymalnym.}

    \item Sieć z 2 węzłami pośrednimi połączonymi szeregowo z krawędzią o
    ujemnej przepustowości pomiędzy węzłami pośrednimi.
    \texttt{testCase4}
    Liczba węzłów w sieci: 4
    Indeks źródła: 0
    Indeks ujścia: 3
    Krawędzie:
    \begin{enumerate}[nosep]
        \item krawędź:
        początek: 0
        koniec: 1
        przepustowość: 7
        \item krawędź:
        początek: 1
        koniec: 2
        przepustowość: -1
        \item krawędź:
        początek: 2
        koniec: 3
        przepustowość: 8
    \end{enumerate}
    Oczekiwany wynik: 0

    \emph{Poprawnie nie znajduje niezerowego maksymalnego przepływu w sieci z ujemną przepustowością krawędzi w wąskim gardle.}

    \item Sieć z 2 węzłami pośrednimi połączonymi szeregowo z krawędzią z
    drugiego węzłą pośredniego do pierwszego węzła pośredniego.
    \texttt{testCase5}
    Liczba węzłów w sieci: 4
    Indeks źródła: 0
    Indeks ujścia: 3
    Krawędzie:
    \begin{enumerate}[nosep]
        \item krawędź:
        początek: 0
        koniec: 1
        przepustowość: 12
        \item krawędź:
        początek: 1
        koniec: 2
        przepustowość: 4
        \item krawędź:
        początek: 2
        koniec: 3
        przepustowość: 15
        \item krawędź:
        początek: 2
        koniec: 1
        przepustowość: 6
    \end{enumerate}
    Oczekiwany wynik: 4

    \emph{Algorytm poprawnie znajduje maksymalny przepływ w sieci.}

    \item Sieć z 2 węzłami pośrednimi połączonymi szeregowo z krawędzią od
    pierwszego węzła pośredniego do ujścia.
    \texttt{testCase6}
    Liczba węzłów w sieci: 4
    Indeks źródła: 0
    Indeks ujścia: 3
    Krawędzie:
    \begin{enumerate}[nosep]
        \item krawędź:
        początek: 0
        koniec: 1
        przepustowość: 9
        \item krawędź:
        początek: 1
        koniec: 2
        przepustowość: 7
        \item krawędź:
        początek: 2
        koniec: 3
        przepustowość: 2
        \item krawędź:
        początek: 2
        koniec: 1
        przepustowość: 5
    \end{enumerate}
    Oczekiwany wynik: 7

    \emph{Algorytm poprawnie znajduje maksymalny przepływ w sieci.}

    \item Sieć z 2 węzłami pośrednimi połączonymi szeregowo z krawędzią od
    źródła do drugiego węzła pośredniego.
    \texttt{testCase7}
    Liczba węzłów w sieci: 4
    Indeks źródła: 0
    Indeks ujścia: 3
    Krawędzie:
    \begin{enumerate}[nosep]
        \item krawędź:
        początek: 0
        koniec: 1
        przepustowość: 11
        \item krawędź:
        początek: 1
        koniec: 2
        przepustowość: 3
        \item krawędź:
        początek: 2
        koniec: 3
        przepustowość: 10
        \item krawędź:
        początek: 0
        koniec: 2
        przepustowość: 6
    \end{enumerate}
    Oczekiwany wynik: 9

    \emph{Algorytm poprawnie znajduje maksymalny przepływ w sieci.}

\end{itemize}


\subsubsection{Sieć z nieistniejącymi węzłami}


\subsubsection{Sieć z ujemną liczbą węzłów}
Częściowo zrealizowano w klasie NegativeSizeNetworkTest Nie jest możliwe
stworzenie sieci z ujemną ilością węzłów.

\subsubsection{Sieć z $1e9$ węzłów}
Ograniczono wielkość testowanej sieci do $1e3$ węzłów, zrealizowano w klasie
HugeFlowNetworkTest Ręczne stworzenie sieci przepływu tej wielkości jest
nierealne. Metoda generateEdges z klasy FlowNetworkGenerator ma złożoność
rzędu O(n^3), co znacząco ogranicza możliwości w tym zakresie.
%%n^3 poprawić notację

