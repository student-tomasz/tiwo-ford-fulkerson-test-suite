\subsection{Testowanie biało\dywiz skrzynkowe}

\subsubsection{Klasa testująca \texttt{algs.network.ToStringTest}}
% Krótki opis jakie testy ta klasa zawiera.
Klasa \texttt{algs.network.ToStringTest} zawiera testy dla metody
\texttt{algs.network.VertexStructure.toString()}.

\begin{center}
\begin{tabular}{@{} >{\bfseries}p{0.2\textwidth} @{\hspace{0.02\textwidth}} p{0.6\textwidth} @{}}
    \toprule
    TestCase & \texttt{algs.network.ToStringTest.testEmpty()} \\
    \midrule
    ID & B.1.1.1 \\
    \midrule
    Opis & Wypisuje listę węzłów wychodzących i wchodzacych. \\
    \midrule
    Dane wejściowe & Struktura bez zdefiniowanych krawędzi wejscia/wyjścia. \\
    \midrule
    Oczekiwany wynik & Struktura bez zdefiniowanych krawędzi wejscia/wyjścia. \\
    \midrule
    Wykonawca & MO \\
    \bottomrule
\end{tabular}
\end{center}

\begin{center}
\begin{tabular}{@{} >{\bfseries}p{0.2\textwidth} @{\hspace{0.02\textwidth}} p{0.6\textwidth} @{}}
    \toprule
    TestCase & \texttt{algs.network.ToStringTest.testOneEdge()} \\
    \midrule
    ID & B.1.1.2 \\
    \midrule
    Opis &  Wypisuje listę węzłów wychodzących i wchodzacych. \\
    \midrule
    Dane wejściowe & Krawędź wyjścia (1, 2, 1); krawędź wejścia (2, 1, 1). \\
    \midrule
    Oczekiwany wynik & Krawędź wyjścia (1, 2, 1); krawędź wejścia (2, 1, 1). \\
    \midrule
    Wykonawca & MO \\
    \bottomrule
\end{tabular}
\end{center}

\begin{center}
\begin{tabular}{@{} >{\bfseries}p{0.2\textwidth} @{\hspace{0.02\textwidth}} p{0.6\textwidth} @{}}
    \toprule
    TestCase & \texttt{algs.network.ToStringTest.testThreeEdges()} \\
    \midrule
    ID & B.1.1.3 \\
    \midrule
    Opis & Wypisuje listę węzłów wychodzących i wchodzacych. \\
    \midrule
    Dane wejściowe & Trzy krawędzie  wejścia (0, 3, 1), (1, 3, 1), (2, 3, 1); trzy krawędzie wyjścia (3, 4, 1), (3, 5, 1) , (3, 6, 1). \\
    \midrule
    Oczekiwany wynik & Trzy krawędzie wejścia (0, 3, 1), (1, 3, 1), (2, 3, 1); trzy krawędzie wyjścia (3, 4, 1), (3, 5, 1) , (3, 6, 1). \\
    \midrule
    Wykonawca & MO \\
    \bottomrule
\end{tabular}
\end{center}


\subsubsection{Klasa testująca \texttt{algs.network.MinimalNetworkConstructorTest}}
Klasa \texttt{algs.network.MinimalNetworkConstructorTest} zawiera testy dla
minimalnego konstruktora \texttt{algs.network.FlowNetworkArray}.

\textt{FlowNetworkArray(int sourceIndex, int sinkIndex, int numVertices)}
\begin{center}
\begin{tabular}{@{} >{\bfseries}p{0.2\textwidth} @{\hspace{0.02\textwidth}} p{0.6\textwidth} @{}}
    \toprule
    TestCase & \texttt{algs.network.MinimalNetworkConstructorTest.validArgumentTest()} \\
    \midrule
    ID & B.1.2.1 \\
    \midrule
    Opis & Konstruktor minimalnej struktury sieci. Inicjalizuje tylko niezbedne zmienne. \\
    \midrule
    Dane wejściowe & sourceIndex = 0; sinkIndex = 1; numVertices = 2
    \midrule
    Oczekiwany wynik & sourceIndex = 0; sinkIndex = 1; numVertices = 2
    \midrule
    Wykonawca & MO \\
    \bottomrule
\end{tabular}
\end{center}

\begin{center}
\begin{tabular}{@{} >{\bfseries}p{0.2\textwidth} @{\hspace{0.02\textwidth}} p{0.6\textwidth} @{}}
    \toprule
    TestCase & \texttt{algs.network.MinimalNetworkConstructorTest.invalidArgumentsTest()} \\
    \midrule
    ID & B.1.2.2 \\
    \midrule
    Opis & Konstruktor minimalnej struktury sieci. Inicjalizuje tylko niezbedne zmienne. \\
    \midrule
    Dane wejściowe & sourceIndex = -1; sinkIndex = -2; numVertices = -2
    \midrule
    Oczekiwany wynik & IllegalArgumentException
    \midrule
    Wykonawca & MO \\
    \bottomrule
\end{tabular}
\end{center}

\begin{center}
\begin{tabular}{@{} >{\bfseries}p{0.2\textwidth} @{\hspace{0.02\textwidth}} p{0.6\textwidth} @{}}
    \toprule
    TestCase & \texttt{algs.network.MinimalNetworkConstructorTest.sinkBeforeSourceTest()} \\
    \midrule
    ID & B.1.2.3 \\
    \midrule
    Opis & Konstruktor minimalnej struktury sieci. Inicjalizuje tylko niezbedne zmienne. \\
    \midrule
    Dane wejściowe & sourceIndex = 1; sinkIndex = 0; numVertices = 2
    \midrule
    Oczekiwany wynik & IllegalArgumentException
    \midrule
    Wykonawca & MO \\
    \bottomrule
\end{tabular}
\end{center}

\begin{center}
\begin{tabular}{@{} >{\bfseries}p{0.2\textwidth} @{\hspace{0.02\textwidth}} p{0.6\textwidth} @{}}
    \toprule
    TestCase & \texttt{algs.network.MinimalNetworkConstructorTest.tooFewVerticesTest()} \\
    \midrule
    ID & B.1.2.4 \\
    \midrule
    Opis & Konstruktor minimalnej struktury sieci. Inicjalizuje tylko niezbedne zmienne. \\
    \midrule
    Dane wejściowe & sourceIndex = 0; sinkIndex = 7; numVertices = 2
    \midrule
    Oczekiwany wynik & IllegalArgumentException
    \midrule
    Wykonawca & MO \\
    \bottomrule
\end{tabular}
\end{center}


\subsubsection{Klasa testująca \texttt{algs.network.NetworkConstructorTest}}
Klasa \texttt{algs.network.MinimalNetworkConstructorTest} zawiera testy dla
 konstruktora \texttt{algs.network.FlowNetworkArray}.

\textt{FlowNetworkArray(int numVertices, int sourceIndex, int sinkIndex, Iterator<EdgeInfo> edges)}
\begin{center}
\begin{tabular}{@{} >{\bfseries}p{0.2\textwidth} @{\hspace{0.02\textwidth}} p{0.6\textwidth} @{}}
    \toprule
    TestCase & \texttt{algs.network.NetworkConstructorTest.validArgumentsTest()} \\
    \midrule
    ID & B.1.3.1 \\
    \midrule
    Opis & Konstruktor struktury reprezentującej graf przpływu. \\
    \midrule
    Dane wejściowe & numVertices = 4; sourceIndex = 0; sinkIndex = 3;  edges =  (0, 1, 3);(1, 2, 2); (2, 3, 2); (0, 2, 3)
    \midrule
    Oczekiwany wynik & numVertices = 4; sourceIndex = 0; sinkIndex = 3;  edges =  (0, 1, 3);(1, 2, 2); (2, 3, 2); (0, 2, 3)
    \midrule
    Wykonawca & MO \\
    \bottomrule
\end{tabular}
\end{center}

\begin{center}
\begin{tabular}{@{} >{\bfseries}p{0.2\textwidth} @{\hspace{0.02\textwidth}} p{0.6\textwidth} @{}}
    \toprule
    TestCase & \texttt{algs.network.NetworkConstructorTest.invalidNumVerticesTest()} \\
    \midrule
    ID & B.1.3.2 \\
    \midrule
    Opis & Konstruktor struktury reprezentującej graf przpływu. \\
    \midrule
    Dane wejściowe & numVertices = -4; sourceIndex = 0; sinkIndex = 3; edges =  (0, 1, 3);(1, 2, 2); (2, 3, 2); (0, 2, 3)
    \midrule
    Oczekiwany wynik & IllegalArgumentException
    \midrule
    Wykonawca & MO \\
    \bottomrule
\end{tabular}
\end{center}

\begin{center}
\begin{tabular}{@{} >{\bfseries}p{0.2\textwidth} @{\hspace{0.02\textwidth}} p{0.6\textwidth} @{}}
    \toprule
    TestCase & \texttt{algs.network.NetworkConstructorTest.invalidSourceIndexTest()} \\
    \midrule
    ID & B.1.3.3 \\
    \midrule
    Opis & Konstruktor struktury reprezentującej graf przpływu. \\
    \midrule
    Dane wejściowe & numVertices = 4; sourceIndex = -1; sinkIndex = 3; edges =  (0, 1, 3);(1, 2, 2); (2, 3, 2); (0, 2, 3)
    \midrule
    Oczekiwany wynik & IllegalArgumentException
    \midrule
    Wykonawca & MO \\
    \bottomrule
\end{tabular}
\end{center}

\begin{center}
\begin{tabular}{@{} >{\bfseries}p{0.2\textwidth} @{\hspace{0.02\textwidth}} p{0.6\textwidth} @{}}
    \toprule
    TestCase & \texttt{algs.network.NetworkConstructorTest.invalidSinkIndexTest()} \\
    \midrule
    ID & B.1.3.4 \\
    \midrule
    Opis & Konstruktor struktury reprezentującej graf przpływu. \\
    \midrule
    Dane wejściowe & numVertices = 4; sourceIndex = 0; sinkIndex = -3; edges =  (0, 1, 3);(1, 2, 2); (2, 3, 2); (0, 2, 3)
    \midrule
    Oczekiwany wynik & IllegalArgumentException
    \midrule
    Wykonawca & MO \\
    \bottomrule
\end{tabular}
\end{center}

\begin{center}
\begin{tabular}{@{} >{\bfseries}p{0.2\textwidth} @{\hspace{0.02\textwidth}} p{0.6\textwidth} @{}}
    \toprule
    TestCase & \texttt{algs.network.NetworkConstructorTest.tooFewVerticesTest()} \\
    \midrule
    ID & B.1.3.5 \\
    \midrule
    Opis & Konstruktor struktury reprezentującej graf przpływu. \\
    \midrule
    Dane wejściowe & numVertices = 2; sourceIndex = 0; sinkIndex = 3; edges =  (0, 1, 3);(1, 2, 2); (2, 3, 2); (0, 2, 3)
    \midrule
    Oczekiwany wynik & IllegalArgumentException
    \midrule
    Wykonawca & MO \\
    \bottomrule
\end{tabular}
\end{center}

\begin{center}
\begin{tabular}{@{} >{\bfseries}p{0.2\textwidth} @{\hspace{0.02\textwidth}} p{0.6\textwidth} @{}}
    \toprule
    TestCase & \texttt{algs.network.NetworkConstructorTest.sinkBeforeSourceTest()} \\
    \midrule
    ID & B.1.3.6 \\
    \midrule
    Opis & Konstruktor struktury reprezentującej graf przpływu. \\
    \midrule
    Dane wejściowe & numVertices = 4; sourceIndex = 3; sinkIndex = 0; edges =  (0, 1, 3);(1, 2, 2); (2, 3, 2); (0, 2, 3)
    \midrule
    Oczekiwany wynik & IllegalArgumentException
    \midrule
    Wykonawca & MO \\
    \bottomrule
\end{tabular}
\end{center}

\begin{center}
\begin{tabular}{@{} >{\bfseries}p{0.2\textwidth} @{\hspace{0.02\textwidth}} p{0.6\textwidth} @{}}
    \toprule
    TestCase & \texttt{algs.network.NetworkConstructorTest.sinkBeforeSourceTest()} \\
    \midrule
    ID & B.1.3.7 \\
    \midrule
    Opis & Konstruktor struktury reprezentującej graf przpływu. \\
    \midrule
    Dane wejściowe & numVertices = 0; sourceIndex = 1; sinkIndex = 2; edges =  pusty Iterator
    \midrule
    Oczekiwany wynik & numVertices = 4; sourceIndex = 0; sinkIndex = 3;  edges =  pusta tablica EdgeInfo[4][4]
    \midrule
    Wykonawca & MO \\
    \bottomrule
\end{tabular}
\end{center}


\subsubsection{Klasa testująca \texttt{algs.network.IllegalStateExceptionTest}}
Klasa \texttt{algs.network.IllegalStateExceptionTest} zawiera testy weryfikujące
 poprawne wywoływanie wyjątków przez metodę \texttt{algs.network.FlowNetworkArray.validate()}.

\begin{center}
\begin{tabular}{@{} >{\bfseries}p{0.2\textwidth} @{\hspace{0.02\textwidth}} p{0.6\textwidth} @{}}
    \toprule
    TestCase & \texttt{algs.network.IllegalStateExceptionTest.moreFlowThanCapacityTest()} \\
    \midrule
    ID & B.1.4.1 \\
    \midrule
    Opis & Metoda weryfikuje czy informacje na temat sieci są akceptowalne.
           Zwracany jest wyjątek, IllegalStateException w dwóch przypadkach:
        \begin{enumerate}
            \item Przepływ krawędzi jest wiekszy niż przepustowaość przepustowaść
            \item Ilość krawędzi wchodzących jest różna od krawędzi wychodzących
        \end{enumerate}\\
    \midrule
    Dane wejściowe & Krawędź (1, 2, 1), wymuszony flow = 2
    \midrule
    Oczekiwany wynik & IllegalStateException
    \midrule
    Wykonawca & MO \\
    \bottomrule
\end{tabular}
\end{center}

\begin{center}
\begin{tabular}{@{} >{\bfseries}p{0.2\textwidth} @{\hspace{0.02\textwidth}} p{0.6\textwidth} @{}}
    \toprule
    TestCase & \texttt{algs.network.IllegalStateExceptionTest.flowConservationTest()} \\
    \midrule
    ID & B.1.4.2 \\
    \midrule
    Opis & Metoda weryfikuje czy informacje na temat sieci są akceptowalne.
           Zwracany jest wyjątek, IllegalStateException w dwóch przypadkach:
        \begin{enumerate}
            \item Przepływ krawędzi jest wiekszy niż przepustowaość przepustowaść
            \item Ilość krawędzi wchodzących jest różna od krawędzi wychodzących
        \end{enumerate}\\
    \midrule
    Dane wejściowe & Krawędź (1, 2, 1), wymuszony flow = 2
    \midrule
    Oczekiwany wynik & Krawędź (1, 2, 1), wymuszony flow = 1
    \midrule
    Wykonawca & MO \\
    \bottomrule
\end{tabular}
\end{center}