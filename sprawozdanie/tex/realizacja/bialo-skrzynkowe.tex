\subsection{Testowanie biało\dywiz skrzynkowe}

\subsubsection{Klasa testująca \texttt{algs.network.ToStringTest}}
% Krótki opis jakie testy ta klasa zawiera.
Klasa \texttt{algs.network.ToStringTest} zawiera testy dla metody
\texttt{algs.network.VertexStructure.toString()}.

\begin{center}
\begin{tabular}{@{} >{\bfseries}p{0.2\textwidth} @{\hspace{0.02\textwidth}} p{0.6\textwidth} @{}}
    \toprule
    TestCase & \texttt{algs.network.ToStringTest.testEmpty()} \\
    \midrule
    ID & B.1.1.1 \\
    \midrule
    Opis & Wypisuje listę węzłów wychodzących i wchodzacych. \\
    \midrule
    Dane wejściowe & Struktura bez zdefiniowanych krawędzi wejscia/wyjścia. \\
    \midrule
    Oczekiwany wynik & Struktura bez zdefiniowanych krawędzi wejscia/wyjścia. \\
    \midrule
    Wykonawca & MO \\
    \bottomrule
\end{tabular}
\end{center}

\begin{center}
\begin{tabular}{@{} >{\bfseries}p{0.2\textwidth} @{\hspace{0.02\textwidth}} p{0.6\textwidth} @{}}
    \toprule
    TestCase & \texttt{algs.network.ToStringTest.testOneEdge()} \\
    \midrule
    ID & B.1.1.2 \\
    \midrule
    Opis &
    Wypisuje listę węzłów wychodzących i wchodzacych. \\
    \midrule
    Dane wejściowe & Krawędź wyjścia (1, 2, 1); krawędź wejścia (2, 1, 1). \\
    \midrule
    Oczekiwany wynik & Krawędź wyjścia (1, 2, 1); krawędź wejścia (2, 1, 1). \\
    \midrule
    Wykonawca & MO \\
    \bottomrule
\end{tabular}
\end{center}

\begin{center}
\begin{tabular}{@{} >{\bfseries}p{0.2\textwidth} @{\hspace{0.02\textwidth}} p{0.6\textwidth} @{}}
    \toprule
    TestCase & \texttt{algs.network.ToStringTest.testThreeEdges()} \\
    \midrule
    ID & B.1.1.3 \\
    \midrule
    Opis & Wypisuje listę węzłów wychodzących i wchodzacych. \\
    \midrule
    Dane wejściowe & Trzy krawędzie  wejścia (0, 3, 1), (1, 3, 1), (2, 3, 1); trzy krawędzie wyjścia (3, 4, 1), (3, 5, 1) , (3, 6, 1). \\
    \midrule
    Oczekiwany wynik & Trzy krawędzie wejścia (0, 3, 1), (1, 3, 1), (2, 3, 1); trzy krawędzie wyjścia (3, 4, 1), (3, 5, 1) , (3, 6, 1). \\
    \midrule
    Wykonawca & MO \\
    \bottomrule
\end{tabular}
\end{center}

\subsubsection{Klasa \texttt{algs.network.VertexStructure}}
\begin{center}
\begin{tabular}{@{} >{\ttfamily}p{0.2\textwidth} @{\hspace{0.02\textwidth}} p{0.6\textwidth} @{}}
    \toprule
    \multicolumn{2}{@{}c@{}}{\bfseries{ToStringTest}} \\
    \midrule
    {\bfseries Funkcja} & \bfseries String \texttt{toString()} \\
    \hline
    {\bfseries Opis} & Wypisuje listę węzłów wychodzących i wchodzacych. \\
    \hline
    {\bfseries Dane wejściowe} & {\begin{enumerate}
                                        \item Struktura bez zdefiniowanych krawędzi wejscia/wyjścia
                                        \item Krawędź wyjścia (1, 2, 1); krawędź wejścia (2, 1, 1)
                                        \item Trzy krawędzie  wejścia (0, 3, 1), (1, 3, 1), (2, 3, 1); trzy krawędzie wyjścia (3, 4, 1), (3, 5, 1) , (3, 6, 1)
                                    \end{enumerate}} \\
    \hline
    {\bfseries Oczekiwany wynik} & {\begin{enumerate}
                                        \item Struktura bez zdefiniowanych krawędzi wejscia/wyjścia
                                        \item Krawędź wyjścia (1, 2, 1); krawędź wejścia (2, 1, 1)
                                        \item Trzy krawędzie  wejścia (0, 3, 1), (1, 3, 1), (2, 3, 1); trzy krawędzie wyjścia (3, 4, 1), (3, 5, 1) , (3, 6, 1)
                                    \end{enumerate}} \\
    \hline
    {\bfseries Wykonawca} & MO \\
    \bottomrule
\end{tabular}
\end{center}

\subsubsection{\texttt{algs.network.FlowNetworkArray}}
\begin{center}
\begin{tabular}{@{} >{\ttfamily}p{0.2\textwidth} @{\hspace{0.02\textwidth}} p{0.6\textwidth} @{}}
    \toprule
    \multicolumn{2}{@{}c@{}}{\bfseries{MinimalNetworkConstructorTest}} \\
    \midrule
    {\bfseries Id} & B.2.1 \\
    \hline
    {\bfseries Funkcja} & \texttt{FlowNetworkArray (\bfseries int sourceIndex, \bfseries int sinkIndex, \bfseries int numVertices)} \\
    \hline
    {\bfseries Opis} & Konstruktor minimalnej struktury sieci. Inicjalizuje tylko niezbedne zmienne. \\
    \hline
    {\bfseries Dane wejściowe} & {\begin{enumerate}
                                        \item sourceIndex = 0; sinkIndex = 1; numVertices = 2
                                        \item sourceIndex = -1; sinkIndex = -2; numVertices = -2
                                        \item sourceIndex = 0; sinkIndex = 1; numVertices = 2
                                        \item sourceIndex = 0; sinkIndex = 7; numVertices = 2
                                    \end{enumerate}} \\
    \hline
    {\bfseries Oczekiwany wynik} & {\begin{enumerate}
                                        \item sourceIndex = 0; sinkIndex = 1; numVertices = 2
                                        \item IllegalArgumentException
                                        \item IllegalArgumentException
                                        \item IllegalArgumentException
                                    \end{enumerate}} \\
    \hline
    {\bfseries Wykonawca} & MO \\
    \bottomrule
\end{tabular}
\end{center}

\begin{center}
\begin{tabular}{@{} >{\ttfamily}p{0.2\textwidth} @{\hspace{0.02\textwidth}} p{0.6\textwidth} @{}}
    \toprule
    \multicolumn{2}{@{}c@{}}{\bfseries{NetworkConstructorTest}} \\
    \midrule
    {\bfseries Id} & B.2.1 \\
    \hline
    {\bfseries Funkcja} & \texttt{FlowNetworkArray (\bfseries int numVertices, \bfseries int sourceIndex,
                                                    \bfseries int sinkIndex,
                                                    \bfseries Iterator<Edges> edges} \\
    \hline
    {\bfseries Opis} & Konstruktor struktury reprezentującej graf przpływu. \\
    \hline
    {\bfseries Dane wejściowe} & {\begin{enumerate}
                                        \item numVertices = 4; sourceIndex = 0; sinkIndex = 3;  edges =  (0, 1, 3);(1, 2, 2); (2, 3, 2); (0, 2, 3)
                                        \item numVertices = -4; sourceIndex = 0; sinkIndex = 3; edges =  (0, 1, 3);(1, 2, 2); (2, 3, 2); (0, 2, 3)
                                        \item numVertices = 4; sourceIndex = -1; sinkIndex = 3; edges =  (0, 1, 3);(1, 2, 2); (2, 3, 2); (0, 2, 3)
                                        \item numVertices = 4; sourceIndex = 0; sinkIndex = -3; edges =  (0, 1, 3);(1, 2, 2); (2, 3, 2); (0, 2, 3)
                                        \item numVertices = 2; sourceIndex = 0; sinkIndex = 3; edges =  (0, 1, 3);(1, 2, 2); (2, 3, 2); (0, 2, 3)
                                        \item numVertices = 4; sourceIndex = 3; sinkIndex = 0; edges =  (0, 1, 3);(1, 2, 2); (2, 3, 2); (0, 2, 3)
                                        \item numVertices = 0; sourceIndex = 1; sinkIndex = 2; edges =  pusty Iterator
                                    \end{enumerate}} \\
    \hline
    {\bfseries Oczekiwany wynik} & {\begin{enumerate}
                                        \item numVertices = 4; sourceIndex = 0; sinkIndex = 3;  edges =  (0, 1, 3);(1, 2, 2); (2, 3, 2); (0, 2, 3)
                                        \item IllegalArgumentException
                                        \item IllegalArgumentException
                                        \item IllegalArgumentException
                                        \item IllegalArgumentException
                                        \item IllegalArgumentException
                                        \item numVertices = 4; sourceIndex = 0; sinkIndex = 3;  edges =  pusta tablica EdgeInfo[4][4]
                                    \end{enumerate}} \\
    \hline
    {\bfseries Wykonawca} & MO \\
    \bottomrule
\end{tabular}
\end{center}

\begin{center}
\begin{tabular}{@{} >{\ttfamily}p{0.2\textwidth} @{\hspace{0.02\textwidth}} p{0.6\textwidth} @{}}
    \toprule
    \multicolumn{2}{@{}c@{}}{\bfseries{IllegalStateExceptionTest}} \\
    \midrule
    {\bfseries Id} & B.2.2 \\
    \hline
    {\bfseries Funkcja} & \bfseries void \texttt{ validate()} \\
    \hline
    {\bfseries Opis} & Metoda weryfikuje czy informacje na temat sieci są akceptowalne.
                       Zwracany jest wyjątek, IllegalStateException w dwóch przypadkach:
        \begin{enumerate}
            \item Przepływ krawędzi jest wiekszy niż przepustowaość przepustowaść
            \item Ilość krawędzi wchodzących jest różna od krawędzi wychodzących
        \end{enumerate}\\
    \hline
    {\bfseries Dane wejściowe} & {\begin{enumerate}
                                        \item Krawędź (1, 2, 1), wymuszony flow = 2
                                        \item Krawędź (1, 2, 1), wymuszony flow = 1
                                    \end{enumerate}} \\
    \hline
    {\bfseries Oczekiwany wynik} & {\begin{enumerate}
                                        \item IllegalStateException
                                        \item IllegalStateException
                                    \end{enumerate}} \\
    \hline
    {\bfseries Wykonawca} & MO \\
    \bottomrule
\end{tabular}
\end{center}
