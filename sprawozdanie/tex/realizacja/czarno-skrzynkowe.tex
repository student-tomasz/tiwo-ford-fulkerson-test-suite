\subsection{Testy czarno\dywiz skrzynkowe}

\subsubsection{Sieć bez węzłów}


\subsubsection{Sieć z 1 węzłem}

\begin{itemize}[nosep]
    \item Samego źródła.
    \item Węzła pośredniego.
    \item Ujścia.
    \item Wspólnego źródła i ujścia.
    \item Wspólnego źródła i ujścia z pętlą.
\end{itemize}


\subsubsection{Sieć bez węzłów pośrednich}

\begin{itemize}[nosep]
    \item Nie połączonymi żadną krawędzią.
    \item Połączonymi pojedynczą krawędzią skierowaną od źródła do ujścia o dodatniej przepustowości.
    \item Połączonymi pojedynczą krawędzią skierowaną od źródła do ujścia o dodatniej przepustowości z pętlą o dodatniej przepustowości w ujściu.
    \item Połączonymi pojedynczą krawędzią skierowaną od źródła do ujścia o dodatniej przepustowości z pętlą o ujemnej przepustowości w źródle.
    \item Połączonymi pojedynczą krawędzią skierowaną od źródła do ujścia o zerowej przepustowości.
    \item Połączonymi pojedynczą krawędzią skierowaną od źródła do ujścia o ujemnej przepustowości.
    \item Połączonymi pojedynczą krawędzią skierowaną od ujścia do źródła o dodatniej przepustowości.
    \item Połączonymi pojedynczą krawędzią skierowaną od ujścia do źródła o ujemnej przepustowości.
    \item Z wieloma krawędziami skierowanymi od żródła do ujścia.
    \item Z wieloma krawędziami skierowanymi od ujścia do źródła.
    \item Z wieloma krawędziami skierowanymi w różnych stronach.
\end{itemize}


\subsubsection{Sieć z 1 węzłem pośrednim}

\begin{itemize}[nosep]
    \item Połączone pojedynczymi krawędziami skierowanymi ze źródła do węzłą
    pośredniego i z węzła pośredniego do ujścia.
    % Z -> P -> U
    \item Z pojedynczymi krawędziami skierowanymi z ujścia do węzła pośredniego
    i z węzła pośredniego do źródła.
    % Z <- P <- U
    \item Z pojedynczymi krawędziami skierowanymi z węzła pośredniego do ujścia
    i z węzła pośredniego do źródła.
    % Z <- P -> U
    \item Z pojedynczymi krawędziami skierowanymi ze źródła do węzła pośredniego
    i z ujścia do węzła pośredniego.
    % Z -> P <- U
    \item Połączone pojedynczymi krawędziami skierowanymi ze źródła do węzła
    pośredniego i z węzła pośredniego do źródła z dodatkową krawędzią o
    dodatniej przepustowości ze źródła do ujścia.
    % Z -> P -> U
    % |_________^
    \item Połączone pojedynczymi krawędziami skierowanymi ze źródła do węzła
    pośredniego i z węzła pośredniego do źródła z pętlą o dodatniej
    przepustowości w węźle pośrednim.
    % Z -> P -> U
    %     / ^
    %     \_|
    \item Połączone zwielokrotnionymi krawędziami ze źródła do węzła pośredniego
    i z węzła pośredniego do źródła, z mieszanymi zwrotami.
    % Z -> P -> U
    %  ^__/ ^__/
    \item Połączone pojedynczą krawędzią skierowaną ze źródła do węzła
    pośredniego, bez krawędzi do ujścia.
    % Z -> P    U
    \item Połączone pojedynczą krawędzia skierowaną ze źródła do węzła
    pośredniego, oraz krawędzią ze źródłą do ujścia.
    % Z -> P    U
    % |_________^
    \item Połączone pojedynczą krawędzia skierowaną z węzła pośredniego do
    ujścia, bez połączenia ze źródłem.
    % Z    P -> U
\end{itemize}


\subsubsection{Sieć z 2 węzłami pośrednimi połączonymi równolegle}

\begin{itemize}[nosep]
    \item Z krawędziami o dodatniej przepustowości ze źródła do obu węzłów
    pośrednich i z obu węzłów pośrednich do ujścia.
    \emph{Zrealizowano. Algorytm znajduje maksymalny przepływ.}
    \item Z dodatkową krawędzią o dodatniej przepustowości z pierwszego węzła
    pośredniego do drugiego węzła pośredniego.
    \emph{Zrealizowano. Algorytm znajduje maksymalny przepływ.}
    \item Z dodatkową krawędzią o dodatniej przepustowości z drugiego węzła
    pośredniego do pierwszego węzła pośredniego.
    \emph{Zrealizowano. Algorytm znajduje maksymalny przepływ.}
    \item Z dodatkowymi krawędziami o dodatniej przepustowości z pierwszego
    węzła pośredniego do drugiego węzła pośredniego i z drugiego węzła
    pośredniego do pierwszego węzła pośredniego.\footnote{Potencjalne zagrożenie
    stworzeniem złożonej pętli.}
    \item Z dodatkową krawędzią o dodatniej przepustowości ze źródła do ujścia.
    \item Z dodatkową krawędzia o dodatniej przepustowości z ujścia do źródła.
    \item Z dodatkową krawędzią o dodatniej przepustowości z ujścia do
    pierwszego węzła pośredniego.
    \item Z dodatkową krawędzią o dodatniej przepustowości z pierwszego węzła
    pośredniego do źródłą.
    \item Z dodatkowymi krawędziami o dodatniej przepustowości z ujścia do
    pierwszego węzła pośredniego i z pierwszego węzłą pośredniego do źródła.
\end{itemize}


\subsubsection{Sieć z 2 węzłami pośrednimi połączonymi szeregowo}

\begin{itemize}[nosep]
    \item Z krawędziami o dodatniej przepustowości ze źródła do pierwszego węzła
    pośredniego, z pierwszego węzła pośredniego do drugiego węzła pośredniego i z
    drugiego węzła pośredniego do ujścia.
    \item Sieć z 2 węzłami pośrednimi połączonymi szeregowo z wieloma
    krawędziami pomiędzy węzłami pośrednimi.
    \item Sieć z 2 węzłami pośrednimi połączonymi szeregowo z krawędzią o
    zerowej przepustowości pomiędzy węzłami pośrednimi.
    \item Sieć z 2 węzłami pośrednimi połączonymi szeregowo z krawędzią o
    ujemnej przepustowości pomiędzy węzłami pośrednimi.
    \item Sieć z 2 węzłami pośrednimi połączonymi szeregowo z krawędzią z
    drugiego węzłą pośredniego do pierwszego węzła pośredniego.\footnote{Ujemna
    przepustowość pomiędzy pierwszym a drugim węzłem pośrednim nie jest tożsama
    z krawędzią o dodatniej przepustowości od drugiego do pierwszego węzła
    pośredniego. Ujemna przepustowość krawędzi jest niepoprawna z definicji,
    natomiast krawędź skierowana w przeciwną stronę jest jak najbardziej
    poprawna z punktu widzenia definicji, choć taka sieć wciąż nie miałaby
    żadnej przepustowości.}
    \item Sieć z 2 węzłami pośrednimi połączonymi szeregowo z krawędzią od
    pierwszego węzła pośredniego do ujścia.
    \item Sieć z 2 węzłami pośrednimi połączonymi szeregowo z krawędzią od
    źródła do drugiego węzła pośredniego.
\end{itemize}


\subsubsection{Sieć z nieistniejącymi węzłami}


\subsubsection{Sieć z ujemną liczbą węzłów}
\emph{Częściowo zrealizowano w klasie NegativeSizeNetworkTest}
Nie jest możliwe stworzenie sieci z ujemną ilością węzłów.

\subsubsection{Sieć z $1e9$ węzłów}
\emph{Ograniczono wielkość testowanej sieci do $1e3$ węzłów}
Ręczne stworzenie sieci przepływu tej wielkości jest nierealne.
Metoda generateEdges z klasy FlowNetworkGenerator ma złożoność rzędu O(n^3),
co znacząco ogranicza możliwości w tym zakresie.
%%n^3 poprawić notację

