\subsection{Testowanie czarno\dywiz skrzynkowe}

\subsubsection{Sieć bez węzłów}
Wszystkie testy na sieci bez węzłów zrealizowano w klasie
\texttt{rw.blackbox.EmptyNetworkTest}.

\begin{center}
\begin{tabular}{@{} >{\bfseries}p{0.2\textwidth} @{\hspace{0.02\textwidth}} p{0.6\textwidth} @{}}
    \toprule
    TestCase & \texttt{rw.blackbox.EmptyNetworkTest.testCase1()} \\
    \midrule
    ID & C.1.1.1 \\ % Popraw ID do swojej sytuacji.
    \midrule
    Dane wejściowe &
    \begin{minipage}[h]{0.6\textwidth}
    \begin{itemize}[leftmargin=*]
        \item liczba węzłów w sieci: \texttt{0},
        \item indeks źródła: \texttt{2},
        \item indeks ujścia: \texttt{5},
        \item krawędzie: \texttt{NULL};
    \end{itemize}
    \end{minipage} \\
    \midrule
    Oczekiwany wynik &
    \begin{minipage}[h]{0.6\textwidth}
    \texttt{0}
    \end{minipage} \\
    \midrule
    Wykonawca & RW \\
    \bottomrule
\end{tabular}
\end{center}

\begin{center}
\begin{tabular}{@{} >{\bfseries}p{0.2\textwidth} @{\hspace{0.02\textwidth}} p{0.6\textwidth} @{}}
    \toprule
    TestCase & \texttt{rw.blackbox.EmptyNetworkTest.testCase2()} \\
    \midrule
    ID & C.1.1.2 \\
    \midrule
    Dane wejściowe &
    \begin{minipage}[h]{0.6\textwidth}
    \begin{itemize}[leftmargin=*]
        \item liczba węzłów w sieci: \texttt{0},
        \item indeks źródła: \texttt{0},
        \item indeks ujścia: \texttt{0},
        \item krawędzie: \texttt{NULL};
    \end{itemize}
    \end{minipage} \\
    \midrule
    Oczekiwany wynik &
    \begin{minipage}[h]{0.6\textwidth}
    \texttt{Exception}
    \end{minipage} \\
    \midrule
    Wykonawca & RW \\
    \bottomrule
\end{tabular}
\end{center}

\subsubsection{Sieć z 1 węzłem}
Wszystkie testy zrealizowane w klasie \texttt{rw.blackbox.JustOneElementTest}.

\begin{center}
\begin{tabular}{@{} >{\bfseries}p{0.2\textwidth} @{\hspace{0.02\textwidth}} p{0.6\textwidth} @{}}
    \toprule
    TestCase & \texttt{rw.blackbox.JustOneElementTest.testCase1()} \\
    \midrule
    ID & C.1.2.1 \\
    \midrule
    Dane wejściowe &
    \begin{minipage}[h]{0.6\textwidth}
    \begin{itemize}[leftmargin=*]
        \item liczba węzłów w sieci: \texttt{1},
        \item indeks źródła: \texttt{0},
        \item indeks ujścia: \texttt{1},
        \item krawędzie: \texttt{NULL};
    \end{itemize}
    \end{minipage} \\
    \midrule
    Oczekiwany wynik &
    \begin{minipage}[h]{0.6\textwidth}
    \texttt{0}
    \end{minipage} \\
    \midrule
    Wykonawca & RW \\
    \bottomrule
\end{tabular}
\end{center}

\begin{center}
\begin{tabular}{@{} >{\bfseries}p{0.2\textwidth} @{\hspace{0.02\textwidth}} p{0.6\textwidth} @{}}
    \toprule
    TestCase & \texttt{rw.blackbox.JustOneElementTest.testCase2()} \\
    \midrule
    ID & C.1.2.2 \\
    \midrule
    Dane wejściowe &
    \begin{minipage}[h]{0.6\textwidth}
    \begin{itemize}[leftmargin=*]
        \item liczba węzłów w sieci: \texttt{1},
        \item indeks źródła: \texttt{2},
        \item indeks ujścia: \texttt{1},
        \item krawędzie: \texttt{NULL};
    \end{itemize}
    \end{minipage} \\
    \midrule
    Oczekiwany wynik &
    \begin{minipage}[h]{0.6\textwidth}
    \texttt{Exception}
    \end{minipage} \\
    \midrule
    Wykonawca & RW \\
    \bottomrule
\end{tabular}
\end{center}

\begin{center}
\begin{tabular}{@{} >{\bfseries}p{0.2\textwidth} @{\hspace{0.02\textwidth}} p{0.6\textwidth} @{}}
    \toprule
    TestCase & \texttt{rw.blackbox.JustOneElementTest.testCase3()} \\
    \midrule
    ID & C.1.2.3 \\
    \midrule
    Dane wejściowe &
    \begin{minipage}[h]{0.6\textwidth}
    \begin{itemize}[leftmargin=*]
        \item liczba węzłów w sieci: \texttt{1},
        \item indeks źródła: \texttt{1},
        \item indeks ujścia: \texttt{0},
        \item krawędzie: \texttt{NULL};
    \end{itemize}
    \end{minipage} \\
    \midrule
    Oczekiwany wynik &
    \begin{minipage}[h]{0.6\textwidth}
    \texttt{Exception}
    \end{minipage} \\
    \midrule
    Wykonawca & RW \\
    \bottomrule
\end{tabular}
\end{center}

\begin{center}
\begin{tabular}{@{} >{\bfseries}p{0.2\textwidth} @{\hspace{0.02\textwidth}} p{0.6\textwidth} @{}}
    \toprule
    TestCase & \texttt{rw.blackbox.JustOneElementTest.testCase4()} \\
    \midrule
    ID & C.1.2.4 \\
    \midrule
    Dane wejściowe &
    \begin{minipage}[h]{0.6\textwidth}
    \begin{itemize}[leftmargin=*]
        \item liczba węzłów w sieci: \texttt{1},
        \item indeks źródła: \texttt{0},
        \item indeks ujścia: \texttt{0},
        \item krawędzie: \texttt{NULL};
    \end{itemize}
    \end{minipage} \\
    \midrule
    Oczekiwany wynik &
    \begin{minipage}[h]{0.6\textwidth}
    \texttt{0}
    \end{minipage} \\
    \midrule
    Wykonawca & RW \\
    \bottomrule
\end{tabular}
\end{center}

\begin{center}
\begin{tabular}{@{} >{\bfseries}p{0.2\textwidth} @{\hspace{0.02\textwidth}} p{0.6\textwidth} @{}}
    \toprule
    TestCase & \texttt{rw.blackbox.JustOneElementTest.testCase5()} \\
    \midrule
    ID & C.1.2.5 \\
    \midrule
    Dane wejściowe &
    \begin{minipage}[h]{0.6\textwidth}
    \begin{itemize}[leftmargin=*]
        \item liczba węzłów w sieci: \texttt{1},
        \item indeks źródła: \texttt{0},
        \item indeks ujścia: \texttt{1},
        \item krawędzie: \texttt{\{(0, 0, 5)\}}
    \end{itemize}
    \end{minipage} \\
    \midrule
    Oczekiwany wynik &
    \begin{minipage}[h]{0.6\textwidth}
    \texttt{0}
    \end{minipage} \\
    \midrule
    Wykonawca & RW \\
    \bottomrule
\end{tabular}
\end{center}

\subsubsection{Sieć bez węzłów pośrednich}
Testy na sieciach bez węzłów pośrednich zrealizowano w klasie JustSourceAndSinkTest.

\begin{center}
\begin{tabular}{@{} >{\bfseries}p{0.2\textwidth} @{\hspace{0.02\textwidth}} p{0.6\textwidth} @{}}
    \toprule
    TestCase & \texttt{rw.blackbox.JustSourceAndSinkTest.testCase1()} \\
    \midrule
    ID & C.1.3.1 \\
    \midrule
    Dane wejściowe &
    \begin{minipage}[h]{0.6\textwidth}
    \begin{itemize}[leftmargin=*]
        \item liczba węzłów w sieci: \texttt{2},
        \item indeks źródła: \texttt{0},
        \item indeks ujścia: \texttt{1},
        \item krawędzie: \texttt{NULL}
    \end{itemize}
    \end{minipage} \\
    \midrule
    Oczekiwany wynik &
    \begin{minipage}[h]{0.6\textwidth}
    \texttt{0}
    \end{minipage} \\
    \midrule
    Wykonawca & RW \\
    \bottomrule
\end{tabular}
\end{center}

\begin{center}
\begin{tabular}{@{} >{\bfseries}p{0.2\textwidth} @{\hspace{0.02\textwidth}} p{0.6\textwidth} @{}}
    \toprule
    TestCase & \texttt{rw.blackbox.JustSourceAndSinkTest.testCase2()} \\
    \midrule
    ID & C.1.3.2 \\
    \midrule
    Dane wejściowe &
    \begin{minipage}[h]{0.6\textwidth}
    \begin{itemize}[leftmargin=*]
        \item liczba węzłów w sieci: \texttt{2},
        \item indeks źródła: \texttt{0},
        \item indeks ujścia: \texttt{1},
        \item krawędzie: \texttt{\{(0, 1, 9)\}}
    \end{itemize}
    \end{minipage} \\
    \midrule
    Oczekiwany wynik &
    \begin{minipage}[h]{0.6\textwidth}
    \texttt{9}
    \end{minipage} \\
    \midrule
    Wykonawca & RW \\
    \bottomrule
\end{tabular}
\end{center}

\begin{center}
\begin{tabular}{@{} >{\bfseries}p{0.2\textwidth} @{\hspace{0.02\textwidth}} p{0.6\textwidth} @{}}
    \toprule
    TestCase & \texttt{rw.blackbox.JustSourceAndSinkTest.testCase3a()} \\
    \midrule
    ID & C.1.3.3 \\
    \midrule
    Dane wejściowe &
    \begin{minipage}[h]{0.6\textwidth}
    \begin{itemize}[leftmargin=*]
        \item liczba węzłów w sieci: \texttt{2},
        \item indeks źródła: \texttt{0},
        \item indeks ujścia: \texttt{1},
        \item krawędzie: \texttt{\{(0, 1, 57956756);(1, 1, 6363)\}}
    \end{itemize}
    \end{minipage} \\
    \midrule
    Oczekiwany wynik &
    \begin{minipage}[h]{0.6\textwidth}
    \texttt{57956756}
    \end{minipage} \\
    \midrule
    Wykonawca & RW \\
    \bottomrule
\end{tabular}
\end{center}

\begin{center}
\begin{tabular}{@{} >{\bfseries}p{0.2\textwidth} @{\hspace{0.02\textwidth}} p{0.6\textwidth} @{}}
    \toprule
    TestCase & \texttt{rw.blackbox.JustSourceAndSinkTest.testCase3b()} \\
    \midrule
    ID & C.1.3.4 \\
    \midrule
    Dane wejściowe &
    \begin{minipage}[h]{0.6\textwidth}
    \begin{itemize}[leftmargin=*]
        \item liczba węzłów w sieci: \texttt{2},
        \item indeks źródła: \texttt{0},
        \item indeks ujścia: \texttt{1},
        \item krawędzie: \texttt{\{(0, 1, 12);(1, 1, 2)\}}
    \end{itemize}
    \end{minipage} \\
    \midrule
    Oczekiwany wynik &
    \begin{minipage}[h]{0.6\textwidth}
    \texttt{Exception}
    \end{minipage} \\
    \midrule
    Wykonawca & RW \\
    \bottomrule
\end{tabular}
\end{center}

\begin{center}
\begin{tabular}{@{} >{\bfseries}p{0.2\textwidth} @{\hspace{0.02\textwidth}} p{0.6\textwidth} @{}}
    \toprule
    TestCase & \texttt{rw.blackbox.JustSourceAndSinkTest.testCase4a()} \\
    \midrule
    ID & C.1.3.5 \\
    \midrule
    Dane wejściowe &
    \begin{minipage}[h]{0.6\textwidth}
    \begin{itemize}[leftmargin=*]
        \item liczba węzłów w sieci: \texttt{2},
        \item indeks źródła: \texttt{0},
        \item indeks ujścia: \texttt{1},
        \item krawędzie: \texttt{\{(0, 1, 346723);(0, 0, -1623474)\}}
    \end{itemize}
    \end{minipage} \\
    \midrule
    Oczekiwany wynik &
    \begin{minipage}[h]{0.6\textwidth}
    \texttt{346723}
    \end{minipage} \\
    \midrule
    Wykonawca & RW \\
    \bottomrule
\end{tabular}
\end{center}

\begin{center}
\begin{tabular}{@{} >{\bfseries}p{0.2\textwidth} @{\hspace{0.02\textwidth}} p{0.6\textwidth} @{}}
    \toprule
    TestCase & \texttt{rw.blackbox.JustSourceAndSinkTest.testCase4b()} \\
    \midrule
    ID & C.1.3.6 \\
    \midrule
    Dane wejściowe &
    \begin{minipage}[h]{0.6\textwidth}
    \begin{itemize}[leftmargin=*]
        \item liczba węzłów w sieci: \texttt{2},
        \item indeks źródła: \texttt{0},
        \item indeks ujścia: \texttt{1},
        \item krawędzie: \texttt{\{(0, 1, 4);(0, 0, -2)\}}
    \end{itemize}
    \end{minipage} \\
    \midrule
    Oczekiwany wynik &
    \begin{minipage}[h]{0.6\textwidth}
    \texttt{Exception}
    \end{minipage} \\
    \midrule
    Wykonawca & RW \\
    \bottomrule
\end{tabular}
\end{center}

\begin{center}
\begin{tabular}{@{} >{\bfseries}p{0.2\textwidth} @{\hspace{0.02\textwidth}} p{0.6\textwidth} @{}}
    \toprule
    TestCase & \texttt{rw.blackbox.JustSourceAndSinkTest.testCase5()} \\
    \midrule
    ID & C.1.3.7 \\
    \midrule
    Dane wejściowe &
    \begin{minipage}[h]{0.6\textwidth}
    \begin{itemize}[leftmargin=*]
        \item liczba węzłów w sieci: \texttt{2},
        \item indeks źródła: \texttt{0},
        \item indeks ujścia: \texttt{1},
        \item krawędzie: \texttt{\{(0, 1, 0)\}}
    \end{itemize}
    \end{minipage} \\
    \midrule
    Oczekiwany wynik &
    \begin{minipage}[h]{0.6\textwidth}
    \texttt{0}
    \end{minipage} \\
    \midrule
    Wykonawca & RW \\
    \bottomrule
\end{tabular}
\end{center}

\begin{center}
\begin{tabular}{@{} >{\bfseries}p{0.2\textwidth} @{\hspace{0.02\textwidth}} p{0.6\textwidth} @{}}
    \toprule
    TestCase & \texttt{rw.blackbox.JustSourceAndSinkTest.testCase6a()} \\
    \midrule
    ID & C.1.3.8 \\
    \midrule
    Dane wejściowe &
    \begin{minipage}[h]{0.6\textwidth}
    \begin{itemize}[leftmargin=*]
        \item liczba węzłów w sieci: \texttt{2},
        \item indeks źródła: \texttt{0},
        \item indeks ujścia: \texttt{1},
        \item krawędzie: \texttt{\{(0, 1, -5)\}}
    \end{itemize}
    \end{minipage} \\
    \midrule
    Oczekiwany wynik &
    \begin{minipage}[h]{0.6\textwidth}
    \texttt{0}
    \end{minipage} \\
    \midrule
    Wykonawca & RW \\
    \bottomrule
\end{tabular}
\end{center}


\begin{center}
\begin{tabular}{@{} >{\bfseries}p{0.2\textwidth} @{\hspace{0.02\textwidth}} p{0.6\textwidth} @{}}
    \toprule
    TestCase & \texttt{rw.blackbox.JustSourceAndSinkTest.testCase6b()} \\
    \midrule
    ID & C.1.3.9 \\
    \midrule
    Dane wejściowe &
    \begin{minipage}[h]{0.6\textwidth}
    \begin{itemize}[leftmargin=*]
        \item liczba węzłów w sieci: \texttt{2},
        \item indeks źródła: \texttt{0},
        \item indeks ujścia: \texttt{1},
        \item krawędzie: \texttt{\{(0, 1, -173848)\}}
    \end{itemize}
    \end{minipage} \\
    \midrule
    Oczekiwany wynik &
    \begin{minipage}[h]{0.6\textwidth}
    \texttt{0}
    \end{minipage} \\
    \midrule
    Wykonawca & RW \\
    \bottomrule
\end{tabular}
\end{center}

\begin{center}
\begin{tabular}{@{} >{\bfseries}p{0.2\textwidth} @{\hspace{0.02\textwidth}} p{0.6\textwidth} @{}}
    \toprule
    TestCase & \texttt{rw.blackbox.JustSourceAndSinkTest.testCase7()} \\
    \midrule
    ID & C.1.4.0 \\
    \midrule
    Dane wejściowe &
    \begin{minipage}[h]{0.6\textwidth}
    \begin{itemize}[leftmargin=*]
        \item liczba węzłów w sieci: \texttt{2},
        \item indeks źródła: \texttt{0},
        \item indeks ujścia: \texttt{1},
        \item krawędzie: \texttt{\{(1, 0, 5)\}}
    \end{itemize}
    \end{minipage} \\
    \midrule
    Oczekiwany wynik &
    \begin{minipage}[h]{0.6\textwidth}
    \texttt{0}
    \end{minipage} \\
    \midrule
    Wykonawca & RW \\
    \bottomrule
\end{tabular}
\end{center}

\begin{center}
\begin{tabular}{@{} >{\bfseries}p{0.2\textwidth} @{\hspace{0.02\textwidth}} p{0.6\textwidth} @{}}
    \toprule
    TestCase & \texttt{rw.blackbox.JustSourceAndSinkTest.testCase8a()} \\
    \midrule
    ID & C.1.4.1 \\
    \midrule
    Dane wejściowe &
    \begin{minipage}[h]{0.6\textwidth}
    \begin{itemize}[leftmargin=*]
        \item liczba węzłów w sieci: \texttt{2},
        \item indeks źródła: \texttt{0},
        \item indeks ujścia: \texttt{1},
        \item krawędzie: \texttt{\{(1, 0, -4)\}}
    \end{itemize}
    \end{minipage} \\
    \midrule
    Oczekiwany wynik &
    \begin{minipage}[h]{0.6\textwidth}
    \texttt{0}
    \end{minipage} \\
    \midrule
    Wykonawca & RW \\
    \bottomrule
\end{tabular}
\end{center}


\begin{center}
\begin{tabular}{@{} >{\bfseries}p{0.2\textwidth} @{\hspace{0.02\textwidth}} p{0.6\textwidth} @{}}
    \toprule
    TestCase & \texttt{rw.blackbox.JustSourceAndSinkTest.testCase8b()} \\
    \midrule
    ID & C.1.4.2 \\
    \midrule
    Dane wejściowe &
    \begin{minipage}[h]{0.6\textwidth}
    \begin{itemize}[leftmargin=*]
        \item liczba węzłów w sieci: \texttt{2},
        \item indeks źródła: \texttt{0},
        \item indeks ujścia: \texttt{1},
        \item krawędzie: \texttt{\{(1, 0, -1639273)\}}
    \end{itemize}
    \end{minipage} \\
    \midrule
    Oczekiwany wynik &
    \begin{minipage}[h]{0.6\textwidth}
    \texttt{Exception}
    \end{minipage} \\
    \midrule
    Wykonawca & RW \\
    \bottomrule
\end{tabular}
\end{center}

\begin{center}
\begin{tabular}{@{} >{\bfseries}p{0.2\textwidth} @{\hspace{0.02\textwidth}} p{0.6\textwidth} @{}}
    \toprule
    TestCase & \texttt{rw.blackbox.JustSourceAndSinkTest.testCase9a()} \\
    \midrule
    ID & C.1.4.3 \\
    \midrule
    Dane wejściowe &
    \begin{minipage}[h]{0.6\textwidth}
    \begin{itemize}[leftmargin=*]
        \item liczba węzłów w sieci: \texttt{2},
        \item indeks źródła: \texttt{0},
        \item indeks ujścia: \texttt{1},
        \item krawędzie: \texttt{\{(0,1,12);(0,1,2);(0,1,4)\}}
    \end{itemize}
    \end{minipage} \\
    \midrule
    Oczekiwany wynik &
    \begin{minipage}[h]{0.6\textwidth}
    \texttt{18}
    \end{minipage} \\
    \midrule
    Wykonawca & RW \\
    \bottomrule
\end{tabular}
\end{center}

\begin{center}
\begin{tabular}{@{} >{\bfseries}p{0.2\textwidth} @{\hspace{0.02\textwidth}} p{0.6\textwidth} @{}}
    \toprule
    TestCase & \texttt{rw.blackbox.JustSourceAndSinkTest.testCase9b()} \\
    \midrule
    ID & C.1.4.4 \\
    \midrule
    Dane wejściowe &
    \begin{minipage}[h]{0.6\textwidth}
    \begin{itemize}[leftmargin=*]
        \item liczba węzłów w sieci: \texttt{2},
        \item indeks źródła: \texttt{0},
        \item indeks ujścia: \texttt{1},
        \item krawędzie: \texttt{\{(1,0,9);(1,0,3);(1,0,15)\}}
    \end{itemize}
    \end{minipage} \\
    \midrule
    Oczekiwany wynik &
    \begin{minipage}[h]{0.6\textwidth}
    \texttt{Exception}
    \end{minipage} \\
    \midrule
    Wykonawca & RW \\
    \bottomrule
\end{tabular}
\end{center}

\begin{center}
\begin{tabular}{@{} >{\bfseries}p{0.2\textwidth} @{\hspace{0.02\textwidth}} p{0.6\textwidth} @{}}
    \toprule
    TestCase & \texttt{rw.blackbox.JustSourceAndSinkTest.testCase10a()} \\
    \midrule
    ID & C.1.4.5 \\
    \midrule
    Dane wejściowe &
    \begin{minipage}[h]{0.6\textwidth}
    \begin{itemize}[leftmargin=*]
        \item liczba węzłów w sieci: \texttt{2},
        \item indeks źródła: \texttt{0},
        \item indeks ujścia: \texttt{1},
        \item krawędzie: \texttt{\{(0,1,-16);(0,1,-5);(0,1,-3)\}}
    \end{itemize}
    \end{minipage} \\
    \midrule
    Oczekiwany wynik &
    \begin{minipage}[h]{0.6\textwidth}
    \texttt{0}
    \end{minipage} \\
    \midrule
    Wykonawca & RW \\
    \bottomrule
\end{tabular}
\end{center}

\begin{center}
\begin{tabular}{@{} >{\bfseries}p{0.2\textwidth} @{\hspace{0.02\textwidth}} p{0.6\textwidth} @{}}
    \toprule
    TestCase & \texttt{rw.blackbox.JustSourceAndSinkTest.testCase10b()} \\
    \midrule
    ID & C.1.4.5 \\
    \midrule
    Dane wejściowe &
    \begin{minipage}[h]{0.6\textwidth}
    \begin{itemize}[leftmargin=*]
        \item liczba węzłów w sieci: \texttt{2},
        \item indeks źródła: \texttt{0},
        \item indeks ujścia: \texttt{1},
        \item krawędzie: \texttt{\{(1,0,-7);(1,0,-1)\}}
    \end{itemize}
    \end{minipage} \\
    \midrule
    Oczekiwany wynik &
    \begin{minipage}[h]{0.6\textwidth}
    \texttt{Exception}
    \end{minipage} \\
    \midrule
    Wykonawca & RW \\
    \bottomrule
\end{tabular}
\end{center}

\begin{center}
\begin{tabular}{@{} >{\bfseries}p{0.2\textwidth} @{\hspace{0.02\textwidth}} p{0.6\textwidth} @{}}
    \toprule
    TestCase & \texttt{rw.blackbox.JustSourceAndSinkTest.testCase11a()} \\
    \midrule
    ID & C.1.4.6 \\
    \midrule
    Dane wejściowe &
    \begin{minipage}[h]{0.6\textwidth}
    \begin{itemize}[leftmargin=*]
        \item liczba węzłów w sieci: \texttt{2},
        \item indeks źródła: \texttt{0},
        \item indeks ujścia: \texttt{1},
        \item krawędzie: \texttt{\{(0,1,4);(1,0,3)\}}
    \end{itemize}
    \end{minipage} \\
    \midrule
    Oczekiwany wynik &
    \begin{minipage}[h]{0.6\textwidth}
    \texttt{4}
    \end{minipage} \\
    \midrule
    Wykonawca & RW \\
    \bottomrule
\end{tabular}
\end{center}

\begin{center}
\begin{tabular}{@{} >{\bfseries}p{0.2\textwidth} @{\hspace{0.02\textwidth}} p{0.6\textwidth} @{}}
    \toprule
    TestCase & \texttt{rw.blackbox.JustSourceAndSinkTest.testCase11b()} \\
    \midrule
    ID & C.1.4.7 \\
    \midrule
    Dane wejściowe &
    \begin{minipage}[h]{0.6\textwidth}
    \begin{itemize}[leftmargin=*]
        \item liczba węzłów w sieci: \texttt{2},
        \item indeks źródła: \texttt{0},
        \item indeks ujścia: \texttt{1},
        \item krawędzie: \texttt{\{(0,1,6);(1,0,11);(0,1,7);(1,0,2)\}}
    \end{itemize}
    \end{minipage} \\
    \midrule
    Oczekiwany wynik &
    \begin{minipage}[h]{0.6\textwidth}
    \texttt{13}
    \end{minipage} \\
    \midrule
    Wykonawca & RW \\
    \bottomrule
\end{tabular}
\end{center}

\begin{center}
\begin{tabular}{@{} >{\bfseries}p{0.2\textwidth} @{\hspace{0.02\textwidth}} p{0.6\textwidth} @{}}
    \toprule
    TestCase & \texttt{rw.blackbox.JustSourceAndSinkTest.testCase11c()} \\
    \midrule
    ID & C.1.4.8 \\
    \midrule
    Dane wejściowe &
    \begin{minipage}[h]{0.6\textwidth}
    \begin{itemize}[leftmargin=*]
        \item liczba węzłów w sieci: \texttt{2},
        \item indeks źródła: \texttt{0},
        \item indeks ujścia: \texttt{1},
        \item krawędzie: \texttt{\{(0,1,15);(1,0,9);(0,1,7);(1,0,4)\}}
    \end{itemize}
    \end{minipage} \\
    \midrule
    Oczekiwany wynik &
    \begin{minipage}[h]{0.6\textwidth}
    \texttt{Exception}
    \end{minipage} \\
    \midrule
    Wykonawca & RW \\
    \bottomrule
\end{tabular}
\end{center}


\subsubsection{Sieć z 1 węzłem pośrednim}
Testy na sieciach z jednym węzłem pośrednim zostały zgrupowane
w klasie SingleVertexTest.
\begin{itemize}[nosep]
    \item Połączone pojedynczymi krawędziami skierowanymi ze źródła do węzłą
    pośredniego i z węzła pośredniego do ujścia.
    \texttt{testCase1a}
    Liczba węzłów w sieci: 3
    Indeks źródła: 0
    Indeks ujścia: 2
    Krawędzie:
    \begin{enumerate}[nosep]
        \item krawędź:
        początek: 0
        koniec: 1
        przepustowość: 512
        \item krawędź:
        początek: 1
        koniec: 2
        przepustowość: 126
    \end{enumerate}
    Oczekiwany wynik: 126

    \texttt{testCase1b}
    Liczba węzłów w sieci: 3
    Indeks źródła: 0
    Indeks ujścia: 2
    Krawędzie:
    \begin{enumerate}[nosep]
        \item krawędź:
        początek: 0
        koniec: 1
        przepustowość: 104526
        \item krawędź:
        początek: 1
        koniec: 2
        przepustowość: 75269
        \item krawędź:
        początek: 1
        koniec: 0
        przepustowość: 1523
    \end{enumerate}
    Oczekiwany wynik: 75269

    \item Z pojedynczymi krawędziami skierowanymi z ujścia do węzła pośredniego
    i z węzła pośredniego do źródła.
    \texttt{testCase2}
    Liczba węzłów w sieci: 3
    Indeks źródła: 0
    Indeks ujścia: 2
    Krawędzie:
    \begin{enumerate}[nosep]
        \item krawędź:
        początek: 1
        koniec: 0
        przepustowość: 43
        \item krawędź:
        początek: 2
        koniec: 1
        przepustowość: 76
    \end{enumerate}
    Oczekiwany wynik: 0

    \item Z pojedynczymi krawędziami skierowanymi z węzła pośredniego do ujścia
    i z węzła pośredniego do źródła.
    \texttt{testCase3}
    Liczba węzłów w sieci: 3
    Indeks źródła: 0
    Indeks ujścia: 2
    Krawędzie:
    \begin{enumerate}[nosep]
        \item krawędź:
        początek: 1
        koniec: 0
        przepustowość: 325
        \item krawędź:
        początek: 1
        koniec: 2
        przepustowość: 12
    \end{enumerate}
    Oczekiwany wynik: 0

    \item Z pojedynczymi krawędziami skierowanymi ze źródła do węzła pośredniego
    i z ujścia do węzła pośredniego.
    \texttt{testCase4}
    Liczba węzłów w sieci: 3
    Indeks źródła: 0
    Indeks ujścia: 2
    Krawędzie:
    \begin{enumerate}[nosep]
        \item krawędź:
        początek: 0
        koniec: 1
        przepustowość: 9
        \item krawędź:
        początek: 2
        koniec: 1
        przepustowość: 13
    \end{enumerate}
    Oczekiwany wynik: 0

    \item Połączone pojedynczymi krawędziami skierowanymi ze źródła do węzła
    pośredniego i z węzła pośredniego do źródła z dodatkową krawędzią o
    dodatniej przepustowości ze źródła do ujścia.
    \texttt{testCase5}
    Liczba węzłów w sieci: 3
    Indeks źródła: 0
    Indeks ujścia: 2
    Krawędzie:
    \begin{enumerate}[nosep]
        \item krawędź:
        początek: 0
        koniec: 1
        przepustowość: 107209
        \item krawędź:
        początek: 1
        koniec: 2
        przepustowość: 75269
        \item krawędź:
        początek: 0
        koniec: 2
        przepustowość: 8301
    \end{enumerate}
    Oczekiwany wynik: 83570

    \item Połączone pojedynczymi krawędziami skierowanymi ze źródła do węzła
    pośredniego i z węzła pośredniego do źródła z pętlą o dodatniej
    przepustowości w węźle pośrednim.
    \texttt{testCase6a}
    Liczba węzłów w sieci: 3
    Indeks źródła: 0
    Indeks ujścia: 2
    Krawędzie:
    \begin{enumerate}[nosep]
        \item krawędź:
        początek: 0
        koniec: 1
        przepustowość: 73
        \item krawędź:
        początek: 1
        koniec: 2
        przepustowość: 17
        \item krawędź:
        początek: 1
        koniec: 1
        przepustowość: 345
    \end{enumerate}
    Oczekiwany wynik: 17

    \texttt{testCase6b}
    Liczba węzłów w sieci: 3
    Indeks źródła: 0
    Indeks ujścia: 2
    Krawędzie:
    \begin{enumerate}[nosep]
        \item krawędź:
        początek: 0
        koniec: 1
        przepustowość: 73
        \item krawędź:
        początek: 1
        koniec: 2
        przepustowość: 17
        \item krawędź:
        początek: 1
        koniec: 1
        przepustowość: 345
    \end{enumerate}
    Oczekiwany wynik: Exception

    \item Połączone zwielokrotnionymi krawędziami ze źródła do węzła pośredniego
    i z węzła pośredniego do źródła, z mieszanymi zwrotami.
    \texttt{testCase7a}
    Liczba węzłów w sieci: 3
    Indeks źródła: 0
    Indeks ujścia: 2
    Krawędzie:
    \begin{enumerate}[nosep]
        \item krawędź:
        początek: 0
        koniec: 1
        przepustowość: 5
        \item krawędź:
        początek: 1
        koniec: 2
        przepustowość: 7
        \item krawędź:
        początek: 1
        koniec: 0
        przepustowość: 12
        \item krawędź:
        początek: 2
        koniec: 1
        przepustowość: 3
    \end{enumerate}
    Oczekiwany wynik: 5

    \texttt{testCase7b}
    Liczba węzłów w sieci: 3
    Indeks źródła: 0
    Indeks ujścia: 2
    Krawędzie:
    \begin{enumerate}[nosep]
        \item krawędź:
        początek: 0
        koniec: 1
        przepustowość: 16
        \item krawędź:
        początek: 1
        koniec: 2
        przepustowość: 8
        \item krawędź:
        początek: 2
        koniec: 1
        przepustowość: 2
        \item krawędź:
        początek: 1
        koniec: 2
        przepustowość: 13
    \end{enumerate}
    Oczekiwany wynik: Exception

    \texttt{testCase7c}
    Liczba węzłów w sieci: 3
    Indeks źródła: 0
    Indeks ujścia: 2
    Krawędzie:
    \begin{enumerate}[nosep]
        \item krawędź:
        początek: 0
        koniec: 1
        przepustowość: 16
        \item krawędź:
        początek: 1
        koniec: 2
        przepustowość: 8
        \item krawędź:
        początek: 1
        koniec:0
        przepustowość: 5
        \item krawędź:
        początek: 2
        koniec: 1
        przepustowość: 2
        \item krawędź:
        początek: 1
        koniec: 2
        przepustowość: 13
    \end{enumerate}
    Oczekiwany wynik: 16

    \item Połączone pojedynczą krawędzią skierowaną ze źródła do węzła
    pośredniego, bez krawędzi do ujścia.

    \texttt{testCase8}
    Liczba węzłów w sieci: 3
    Indeks źródła: 0
    Indeks ujścia: 2
    Krawędzie:
    \begin{enumerate}[nosep]
        \item krawędź:
        początek: 0
        koniec: 1
        przepustowość: 3
    \end{enumerate}
    Oczekiwany wynik: 0

    \item Połączone pojedynczą krawędzia skierowaną ze źródła do węzła
    pośredniego, oraz krawędzią ze źródłą do ujścia.
    \texttt{testCase9}
    Liczba węzłów w sieci: 3
    Indeks źródła: 0
    Indeks ujścia: 2
    Krawędzie:
    \begin{enumerate}[nosep]
        \item krawędź:
        początek: 0
        koniec: 1
        przepustowość: 13
        \item krawędź:
        początek: 0
        koniec: 2
        przepustowość: 5
    \end{enumerate}
    Oczekiwany wynik: 5

    \item Połączone pojedynczą krawędzia skierowaną z węzła pośredniego do
    ujścia, bez połączenia ze źródłem.
    \texttt{testCase10}
    Liczba węzłów w sieci: 3
    Indeks źródła: 0
    Indeks ujścia: 2
    Krawędzie:
    \begin{enumerate}[nosep]
        \item krawędź:
        początek: 1
        koniec: 2
        przepustowość: 7
    \end{enumerate}
    Oczekiwany wynik: 0
\end{itemize}


\subsubsection{Sieć z 2 węzłami pośrednimi połączonymi równolegle}
Testy operujące na sieciach z 2 węzłami pośrednimi połączonymi równolegle zostały
zebrane w klasie TwoParallelVerticesTest.
\begin{itemize}[nosep]
    \item Z krawędziami o dodatniej przepustowości ze źródła do obu węzłów
    pośrednich i z obu węzłów pośrednich do ujścia.
    \texttt{testCase1}
    Liczba węzłów w sieci: 4
    Indeks źródła: 0
    Indeks ujścia: 3
    Krawędzie:
    \begin{enumerate}[nosep]
        \item krawędź:
        początek: 0
        koniec: 1
        przepustowość: 1
        \item krawędź:
        początek: 0
        koniec: 2
        przepustowość: 4
        \item krawędź:
        początek: 1
        koniec: 3
        przepustowość: 2
        \item krawędź:
        początek: 2
        koniec: 3
        przepustowość: 3
    \end{enumerate}
    Oczekiwany wynik: 4

    \item Z dodatkową krawędzią o dodatniej przepustowości z pierwszego węzła
    pośredniego do drugiego węzła pośredniego.
    \texttt{testCase2}
    Liczba węzłów w sieci: 4
    Indeks źródła: 0
    Indeks ujścia: 3
    Krawędzie:
    \begin{enumerate}[nosep]
        \item krawędź:
        początek: 0
        koniec: 1
        przepustowość: 7
        \item krawędź:
        początek: 0
        koniec: 2
        przepustowość: 1
        \item krawędź:
        początek: 1
        koniec: 3
        przepustowość: 5
        \item krawędź:
        początek: 2
        koniec: 3
        przepustowość: 3
        \item krawędź:
        początek: 1
        koniec: 2
        przepustowość: 9
    \end{enumerate}
    Oczekiwany wynik: 8

    \item Z dodatkową krawędzią o dodatniej przepustowości z drugiego węzła
    pośredniego do pierwszego węzła pośredniego.
    \texttt{testCase3}
    Liczba węzłów w sieci: 4
    Indeks źródła: 0
    Indeks ujścia: 3
    Krawędzie:
    \begin{enumerate}[nosep]
        \item krawędź:
        początek: 0
        koniec: 1
        przepustowość: 3
        \item krawędź:
        początek: 0
        koniec: 2
        przepustowość: 6
        \item krawędź:
        początek: 1
        koniec: 3
        przepustowość: 10
        \item krawędź:
        początek: 2
        koniec: 3
        przepustowość: 4
        \item krawędź:
        początek: 2
        koniec: 1
        przepustowość: 2
    \end{enumerate}
    Oczekiwany wynik: 9

    \item Z dodatkowymi krawędziami o dodatniej przepustowości z pierwszego
    węzła pośredniego do drugiego węzła pośredniego i z drugiego węzła
    pośredniego do pierwszego węzła pośredniego.
    \texttt{testCase4}
    Liczba węzłów w sieci: 4
    Indeks źródła: 0
    Indeks ujścia: 3
    Krawędzie:
    \begin{enumerate}[nosep]
        \item krawędź:
        początek: 0
        koniec: 1
        przepustowość: 17
        \item krawędź:
        początek: 0
        koniec: 2
        przepustowość: 11
        \item krawędź:
        początek: 1
        koniec: 3
        przepustowość: 5
        \item krawędź:
        początek: 2
        koniec: 3
        przepustowość: 3
        \item krawędź:
        początek: 2
        koniec: 1
        przepustowość: 9
        \item krawędź:
        początek: 1
        koniec: 2
        przepustowość: 6
    \end{enumerate}
    Oczekiwany wynik: 8

    \item Z dodatkową krawędzią o dodatniej przepustowości ze źródła do ujścia.
    \texttt{testCase5}
    Liczba węzłów w sieci: 4
    Indeks źródła: 0
    Indeks ujścia: 3
    Krawędzie:
    \begin{enumerate}[nosep]
        \item krawędź:
        początek: 0
        koniec: 1
        przepustowość: 5
        \item krawędź:
        początek: 0
        koniec: 2
        przepustowość: 3
        \item krawędź:
        początek: 1
        koniec: 3
        przepustowość: 4
        \item krawędź:
        początek: 2
        koniec: 3
        przepustowość: 9
        \item krawędź:
        początek: 0
        koniec: 3
        przepustowość: 3
    \end{enumerate}
    Oczekiwany wynik: 10

    \item Z dodatkową krawędzia o dodatniej przepustowości z ujścia do źródła.
    \texttt{testCase6}
    Liczba węzłów w sieci: 4
    Indeks źródła: 0
    Indeks ujścia: 3
    Krawędzie:
    \begin{enumerate}[nosep]
        \item krawędź:
        początek: 0
        koniec: 1
        przepustowość: 2
        \item krawędź:
        początek: 0
        koniec: 2
        przepustowość: 5
        \item krawędź:
        początek: 1
        koniec: 3
        przepustowość: 8
        \item krawędź:
        początek: 2
        koniec: 3
        przepustowość: 1
        \item krawędź:
        początek: 3
        koniec: 0
        przepustowość: 4
    \end{enumerate}
    Oczekiwany wynik: 3

    \item Z dodatkową krawędzią o dodatniej przepustowości z ujścia do
    pierwszego węzła pośredniego.
    \texttt{testCase7}
    Liczba węzłów w sieci: 4
    Indeks źródła: 0
    Indeks ujścia: 3
    Krawędzie:
    \begin{enumerate}[nosep]
        \item krawędź:
        początek: 0
        koniec: 1
        przepustowość: 1
        \item krawędź:
        początek: 0
        koniec: 2
        przepustowość: 9
        \item krawędź:
        początek: 1
        koniec: 3
        przepustowość: 12
        \item krawędź:
        początek: 2
        koniec: 3
        przepustowość: 4
        \item krawędź:
        początek: 3
        koniec: 1
        przepustowość: 3
    \end{enumerate}
    Oczekiwany wynik: 5

    \item Z dodatkową krawędzią o dodatniej przepustowości z pierwszego węzła
    pośredniego do źródłą.
    \texttt{testCase8}
    Liczba węzłów w sieci: 4
    Indeks źródła: 0
    Indeks ujścia: 3
    Krawędzie:
    \begin{enumerate}[nosep]
        \item krawędź:
        początek: 0
        koniec: 1
        przepustowość: 8
        \item krawędź:
        początek: 0
        koniec: 2
        przepustowość: 3
        \item krawędź:
        początek: 1
        koniec: 3
        przepustowość: 5
        \item krawędź:
        początek: 2
        koniec: 3
        przepustowość: 7
        \item krawędź:
        początek: 2
        koniec: 1
        przepustowość: 6
    \end{enumerate}
    Oczekiwany wynik: 8

    \item Z dodatkowymi krawędziami o dodatniej przepustowości z ujścia do
    pierwszego węzła pośredniego i z pierwszego węzłą pośredniego do źródła.
    \texttt{testCase9}
    Liczba węzłów w sieci: 4
    Indeks źródła: 0
    Indeks ujścia: 3
    Krawędzie:
    \begin{enumerate}[nosep]
        \item krawędź:
        początek: 0
        koniec: 1
        przepustowość: 7
        \item krawędź:
        początek: 0
        koniec: 2
        przepustowość: 8
        \item krawędź:
        początek: 1
        koniec: 3
        przepustowość: 2
        \item krawędź:
        początek: 2
        koniec: 3
        przepustowość: 3
        \item krawędź:
        początek: 3
        koniec: 1
        przepustowość: 11
        \item krawędź:
        początek: 1
        koniec: 0
        przepustowość: 8
    \end{enumerate}
    Oczekiwany wynik: 5
\end{itemize}


\subsubsection{Sieć z 2 węzłami pośrednimi połączonymi szeregowo}
Testy wykorzystujące sieci z 2 węzłami pośrednimi znajdują się z kolei w klasie
TwoSerialVerticesTest.
\begin{itemize}[nosep]
    \item Z krawędziami o dodatniej przepustowości ze źródła do pierwszego węzła
    pośredniego, z pierwszego węzła pośredniego do drugiego węzła pośredniego i z
    drugiego węzła pośredniego do ujścia.
    \texttt{testCase1}
    Liczba węzłów w sieci: 4
    Indeks źródła: 0
    Indeks ujścia: 3
    Krawędzie:
    \begin{enumerate}[nosep]
        \item krawędź:
        początek: 0
        koniec: 1
        przepustowość: 5
        \item krawędź:
        początek: 1
        koniec: 2
        przepustowość: 3
        \item krawędź:
        początek: 2
        koniec: 3
        przepustowość: 7
    \end{enumerate}
    Oczekiwany wynik: 15

    \item Sieć z 2 węzłami pośrednimi połączonymi szeregowo z wieloma
    krawędziami pomiędzy węzłami pośrednimi.
    \texttt{testCase2a}
    Liczba węzłów w sieci: 4
    Indeks źródła: 0
    Indeks ujścia: 3
    Krawędzie:
    \begin{enumerate}[nosep]
        \item krawędź:
        początek: 0
        koniec: 1
        przepustowość: 15
        \item krawędź:
        początek: 1
        koniec: 2
        przepustowość: 1
        \item krawędź:
        początek: 2
        koniec: 3
        przepustowość: 21
        \item krawędź:
        początek: 1
        koniec: 2
        przepustowość: 5
        \item krawędź:
        początek: 1
        koniec: 2
        przepustowość: 7
        \item krawędź:
        początek: 1
        koniec: 2
        przepustowość: 3
    \end{enumerate}
    Oczekiwany wynik: 15

    \texttt{testCase2b}
    Liczba węzłów w sieci: 4
    Indeks źródła: 0
    Indeks ujścia: 3
    Krawędzie:
    \begin{enumerate}[nosep]
        \item krawędź:
        początek: 0
        koniec: 1
        przepustowość: 15
        \item krawędź:
        początek: 1
        koniec: 2
        przepustowość: 1
        \item krawędź:
        początek: 2
        koniec: 3
        przepustowość: 21
        \item krawędź:
        początek: 1
        koniec: 2
        przepustowość: 5
        \item krawędź:
        początek: 1
        koniec: 2
        przepustowość: 7
        \item krawędź:
        początek: 1
        koniec: 2
        przepustowość: 3
    \end{enumerate}
    Oczekiwany wynik: Exception

    \item Sieć z 2 węzłami pośrednimi połączonymi szeregowo z krawędzią o
    zerowej przepustowości pomiędzy węzłami pośrednimi.
    \texttt{testCase3}
    Liczba węzłów w sieci: 4
    Indeks źródła: 0
    Indeks ujścia: 3
    Krawędzie:
    \begin{enumerate}[nosep]
        \item krawędź:
        początek: 0
        koniec: 1
        przepustowość: 3
        \item krawędź:
        początek: 1
        koniec: 2
        przepustowość: 0
        \item krawędź:
        początek: 2
        koniec: 3
        przepustowość: 4
    \end{enumerate}
    Oczekiwany wynik: 0

    \item Sieć z 2 węzłami pośrednimi połączonymi szeregowo z krawędzią o
    ujemnej przepustowości pomiędzy węzłami pośrednimi.
    \texttt{testCase4}
    Liczba węzłów w sieci: 4
    Indeks źródła: 0
    Indeks ujścia: 3
    Krawędzie:
    \begin{enumerate}[nosep]
        \item krawędź:
        początek: 0
        koniec: 1
        przepustowość: 7
        \item krawędź:
        początek: 1
        koniec: 2
        przepustowość: -1
        \item krawędź:
        początek: 2
        koniec: 3
        przepustowość: 8
    \end{enumerate}
    Oczekiwany wynik: 0

    \item Sieć z 2 węzłami pośrednimi połączonymi szeregowo z krawędzią z
    drugiego węzłą pośredniego do pierwszego węzła pośredniego.
    \texttt{testCase5}
    Liczba węzłów w sieci: 4
    Indeks źródła: 0
    Indeks ujścia: 3
    Krawędzie:
    \begin{enumerate}[nosep]
        \item krawędź:
        początek: 0
        koniec: 1
        przepustowość: 12
        \item krawędź:
        początek: 1
        koniec: 2
        przepustowość: 4
        \item krawędź:
        początek: 2
        koniec: 3
        przepustowość: 15
        \item krawędź:
        początek: 2
        koniec: 1
        przepustowość: 6
    \end{enumerate}
    Oczekiwany wynik: 4

    \item Sieć z 2 węzłami pośrednimi połączonymi szeregowo z krawędzią od
    pierwszego węzła pośredniego do ujścia.
    \texttt{testCase6}
    Liczba węzłów w sieci: 4
    Indeks źródła: 0
    Indeks ujścia: 3
    Krawędzie:
    \begin{enumerate}[nosep]
        \item krawędź:
        początek: 0
        koniec: 1
        przepustowość: 9
        \item krawędź:
        początek: 1
        koniec: 2
        przepustowość: 7
        \item krawędź:
        początek: 2
        koniec: 3
        przepustowość: 2
        \item krawędź:
        początek: 2
        koniec: 1
        przepustowość: 5
    \end{enumerate}
    Oczekiwany wynik: 7

    \item Sieć z 2 węzłami pośrednimi połączonymi szeregowo z krawędzią od
    źródła do drugiego węzła pośredniego.
    \texttt{testCase7}
    Liczba węzłów w sieci: 4
    Indeks źródła: 0
    Indeks ujścia: 3
    Krawędzie:
    \begin{enumerate}[nosep]
        \item krawędź:
        początek: 0
        koniec: 1
        przepustowość: 11
        \item krawędź:
        początek: 1
        koniec: 2
        przepustowość: 3
        \item krawędź:
        początek: 2
        koniec: 3
        przepustowość: 10
        \item krawędź:
        początek: 0
        koniec: 2
        przepustowość: 6
    \end{enumerate}
    Oczekiwany wynik: 9

\end{itemize}


\subsubsection{Sieć z nieistniejącymi węzłami}


\subsubsection{Sieć z ujemną liczbą węzłów}
Częściowo zrealizowano w klasie NegativeSizeNetworkTest Nie jest możliwe
stworzenie sieci z ujemną ilością węzłów.

\subsubsection{Sieć z $1e9$ węzłów}
Ograniczono wielkość testowanej sieci do $1e3$ węzłów, zrealizowano w klasie
HugeFlowNetworkTest Ręczne stworzenie sieci przepływu tej wielkości jest
nierealne. Metoda generateEdges z klasy FlowNetworkGenerator ma złożoność
rzędu $O(n^3)$, co znacząco ogranicza możliwości w tym zakresie.
%%n^3 poprawić notację
