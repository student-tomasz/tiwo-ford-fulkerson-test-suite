\subsection{Testy czarno\dywiz skrzynkowe}



\subsubsection{Sieć bez węzłów}
Wszystkie testy na sieci bez węzłów zrealizowano w klasie EmptyNetworkTest.

\emph{testCase1}
Liczba węzłów w sieci:0
Indeks źródła: 2
Indeks ujścia: 5
Krawędzie: NULL
Oczekiwany wynik: 0

\emph{testCase2}
Liczba węzłów w sieci: 0
Indeks źródła: 0
Indeks ujścia: 0
Krawędzie: NULL
Oczekiwany wynik: Exception

\emph{Algorytm poprawnie nie znajduje niezerowego przepływu maksymalnego w sieci
z samym źródłem.}

\subsubsection{Sieć z 1 węzłem}
Wszystkie testy zrealizowane w klasie JustOneElementTest.
\begin{itemize}[nosep]
    \item Samego źródła.
    \emph{testCase1}
    Liczba węzłów w sieci: 1
    Indeks źródła: 0
    Indeks ujścia: 1
    Krawędzie: NULL
    Oczekiwany wynik: 0

    \emph{Algorytm poprawnie nie znajduje niezerowego przepływu maksymalnego w sieci
    z samym źródłem.}

    \item Węzła pośredniego.
    \emph{testCase2}
    Liczba węzłów w sieci: 1
    Indeks źródła: 2
    Indeks ujścia: 1
    Krawędzie: NULL
    Oczekiwany wynik: Exception

    \emph{Algorytm poprawnie zwraca wyjątek przy próbie odwołania się
    do nieistniejącego węzła.}

    \item Ujścia.
    \emph{testCase3}
    Liczba węzłów w sieci: 1
    Indeks źródła: 1
    Indeks ujścia: 0
    Krawędzie: NULL
    Oczekiwany wynik: Exception

    \emph{Algorytm poprawnie zwraca wyjątek przy próbie odwołania się
    do nieistniejącego węzła.}

    \item Wspólnego źródła i ujścia.
    \emph{testCase4}
    Liczba węzłów w sieci: 1
    Indeks źródła: 0
    Indeks ujścia: 0
    Krawędzie: NULL
    Oczekiwany wynik: 0

    \emph{Algorytm poprawnie nie znajduje niezerowego
    przepływu maksymalnego w sieci ze źródłem i ujściem w tym samym punkcie.}

    \item Wspólnego źródła i ujścia z pętlą.
    \emph{testCase5}
    Liczba węzłów w sieci: 1
    Indeks źródła: 0
    Indeks ujścia: 1
    Krawędzie:
    \begin{enumerate}[nosep]
        \item krawędź:
        początek: 0
        koniec: 0
        przepustowość: 5
    \end{enumerate}
    Oczekiwany wynik: 0

    \emph{Algorytm poprawnie nie znajduje niezerowego
    przepływu maksymalnego w sieci ze źródłem i ujściem w tym samym punkcie z pętlą.}
\end{itemize}


\subsubsection{Sieć bez węzłów pośrednich}
Testy na sieciach bez węzłów pośrednich zrealizowano w klasie JustSourceAndSinkTest.
\begin{itemize}[nosep]
    \item Nie połączonymi żadną krawędzią.
    \emph{testCase1}
    \emph{Algorytm poprawnie nie znajduje niezerowego
    przepływu maksymalnego w sieci bez ścieżki od źródła do ujścia.}

    \item Połączonymi pojedynczą krawędzią skierowaną od źródła do ujścia o dodatniej przepustowości.
    \emph{testCase2}
    \emph{Algorytm poprawnie znajduje maksymalny przepływ w sieci.}

    \item Połączonymi pojedynczą krawędzią skierowaną od źródła do ujścia o dodatniej przepustowości z pętlą o dodatniej przepustowości w ujściu.
    \emph{testCase3a i testCase3b}
    \emph{Algorytm poprawnie znajduje maksymalny przepływ
    w sieci. Algorytm nie informuje o istnieniu pętli w sieci.}

    \item Połączonymi pojedynczą krawędzią skierowaną od źródła do ujścia o dodatniej przepustowości z pętlą o ujemnej przepustowości w źródle.
    \emph{testCase4a i testCase4b}
    \emph{Algorytm poprawnie znajduje maksymalny przepływ
    w sieci. Algorytm nie podnosi wyjątku przy pętli o ujemnej przepustowości.}

    \item Połączonymi pojedynczą krawędzią skierowaną od źródła do ujścia o zerowej przepustowości.
    \emph{testCase5}
    \emph{Algorytm poprawnie nie znajduje niezerowego przepływu
    maksymalnego w sieci bez ścieżki od źródła do ujścia.}

    \item Połączonymi pojedynczą krawędzią skierowaną od źródła do ujścia o ujemnej przepustowości.
    \emph{testCase6a i testCase6b}
    \emph{Algorytm poprawnie nie znajduje
    dodatniego przepływu maksymalnego w sieci bez ścieżki od źródła do ujścia. Poprawnie
    podnoszony jest wyjątek dla krawędzi o ujemnej przepustowości.}

    \item Połączonymi pojedynczą krawędzią skierowaną od ujścia do źródła o dodatniej przepustowości.
    \emph{testCase7}
    \emph{Algorytm poprawnie nie znajduje niezerowego
    przepływu maksymalnego w sieci bez ścieżki od źródła do ujścia.}

    \item Połączonymi pojedynczą krawędzią skierowaną od ujścia do źródła o ujemnej przepustowości.
    \emph{testCase8a i testCase8b}
    \emph{Algorytm poprawnie nie znajduje
    niezerowego przepływu maksymalnego w sieci bez ścieżki o dodatniej
    przepustowości od źródła do ujścia. Nie jest podnoszony wyjątek
    w związku z występowaniem krawędzi o ujemnej przepustowości.}

    \item Z wieloma krawędziami skierowanymi od żródła do ujścia.
    \emph{testCase9a i testCase9b}
    \emph{Algorytm nie odczytuje
    poprawnie przepustowości ze zwielokrotnionych krawędzi. Nie jest
    podnoszony wyjątek w związku z występowaniem zwielokrotnionych krawędzi.}

    \item Z wieloma krawędziami skierowanymi od ujścia do źródła.
    \emph{testCase10a i testCase10b}
    \emph{Algorytm poprawnie
    nie znajduje niezerowego przepływu maksymalnego w sieci bez ścieżki
    o dodatniej przepustowości od źródła do ujścia. Nie jest podnoszony
    wyjątek mimo występowania krawędzi o ujemnej przepustowości i krawędzi
    zwielokrotnionych.}

    \item Z wieloma krawędziami skierowanymi w różnych stronach.
    \emph{testCase11a, testCase11b i testCase11c}
    \emph{Algorytm poprawnie znajduje przepływ maksymalny w sieci z dodatkową krawędzią
    o tym samym kierunku, a przeciwnym zwrocie. W przypadku zwielokrotnienia,
    którejkolwiek z tych krawędzi, algorytm zwraca nieprawidłową wartość
    przepływu, nie podnosząc wyjątku w związku z występowaniem krawędzi
    o ujemnej przepustowości.}

\end{itemize}


\subsubsection{Sieć z 1 węzłem pośrednim}
Testy na sieciach z jednym węzłem pośrednim zostały zgrupowane
w klasie SingleVertexTest.
\begin{itemize}[nosep]
    \item Połączone pojedynczymi krawędziami skierowanymi ze źródła do węzłą
    pośredniego i z węzła pośredniego do ujścia.
    \emph{testCase1a i testCase1b}
    \emph{Algorytm poprawnie
    znajduje maksymalny przepływ w sieci.}
    % Z -> P -> U

    \item Z pojedynczymi krawędziami skierowanymi z ujścia do węzła pośredniego
    i z węzła pośredniego do źródła.
    \emph{testCase2}
    \emph{Algorytm poprawnie nie znajduje niezerowego
    przepływu maksymalnego w sieci bez ścieżki od źródła do ujścia.}
    % Z <- P <- U

    \item Z pojedynczymi krawędziami skierowanymi z węzła pośredniego do ujścia
    i z węzła pośredniego do źródła.
    \emph{testCase3}
    \emph{Algorytm poprawnie nie znajduje niezerowego
    przepływu maksymalnego w sieci bez ścieżki od źródła do ujścia.}
    % Z <- P -> U

    \item Z pojedynczymi krawędziami skierowanymi ze źródła do węzła pośredniego
    i z ujścia do węzła pośredniego.
    \emph{testCase4}
    \emph{Algorytm poprawnie nie znajduje niezerowego
    przepływu maksymalnego w sieci bez ścieżki od źródła do ujścia.}
    % Z -> P <- U

    \item Połączone pojedynczymi krawędziami skierowanymi ze źródła do węzła
    pośredniego i z węzła pośredniego do źródła z dodatkową krawędzią o
    dodatniej przepustowości ze źródła do ujścia.
    \emph{testCase5}
    \emph{Algorytm poprawnie znajduje maksymalny
    przepływ w sieci.}
    % Z -> P -> U
    % |_________^

    \item Połączone pojedynczymi krawędziami skierowanymi ze źródła do węzła
    pośredniego i z węzła pośredniego do źródła z pętlą o dodatniej
    przepustowości w węźle pośrednim.
    \emph{testCase6a i testCase6b}
    \emph{Algorytm poprawnie znajduje maksymalny
    przepływ w sieci. Algorytm nie informuje o istnieniu pętli w sieci.}
    % Z -> P -> U
    %     / ^
    %     \_|

    \item Połączone zwielokrotnionymi krawędziami ze źródła do węzła pośredniego
    i z węzła pośredniego do źródła, z mieszanymi zwrotami.
    \emph{testCase7a, testCase7b i testCase7c}
    \emph{Algorytm poprawnie
    znajduje maksymalny przepływ w sieci o pojedynczych krawędziach. Konstruktor
    ani funkcja validate() z klasy FlowNetworkArray nie podnoszą wyjątków.
    W przypadku zwielokrotnienia krawędzi algorytm nie jest w stanie wyznaczyć
    prawidłowego maksymalnego przepływu w sieci.}
    % Z -> P -> U
    %  ^__/ ^__/

    \item Połączone pojedynczą krawędzią skierowaną ze źródła do węzła
    pośredniego, bez krawędzi do ujścia.
    \emph{testCase8}
    \emph{Algorytm poprawnie nie znajduje niezerowego
    przepływu maksymalnego w sieci bez ścieżki od źródła do ujścia.}
    % Z -> P    U

    \item Połączone pojedynczą krawędzia skierowaną ze źródła do węzła
    pośredniego, oraz krawędzią ze źródłą do ujścia.
    \emph{testCase9}
    \emph{Algorytm poprawnie znajduje maksymalny
    przepływ w sieci.}
    % Z -> P    U
    % |_________^

    \item Połączone pojedynczą krawędzia skierowaną z węzła pośredniego do
    ujścia, bez połączenia ze źródłem.
    \emph{testCase10}
    \emph{Algorytm poprawnie nie znajduje niezerowego
    przepływu maksymalnego w sieci bez ścieżki od źródła do ujścia.}
    % Z    P -> U
\end{itemize}


\subsubsection{Sieć z 2 węzłami pośrednimi połączonymi równolegle}
Testy operujące na sieciach z 2 węzłami pośrednimi połączonymi równolegle zostały
zebrane w klasie TwoParallelVerticesTest.
\begin{itemize}[nosep]
    \item Z krawędziami o dodatniej przepustowości ze źródła do obu węzłów
    pośrednich i z obu węzłów pośrednich do ujścia.
    \emph{testCase1}
    \emph{Algorytm poprawnie znajduje maksymalny przepływ.}

    \item Z dodatkową krawędzią o dodatniej przepustowości z pierwszego węzła
    pośredniego do drugiego węzła pośredniego.
    \emph{testCase2}
    \emph{Algorytm poprawnie znajduje maksymalny przepływ.}

    \item Z dodatkową krawędzią o dodatniej przepustowości z drugiego węzła
    pośredniego do pierwszego węzła pośredniego.
    \emph{testCase3}
    \emph{Algorytm poprawnie znajduje maksymalny przepływ.}

    \item Z dodatkowymi krawędziami o dodatniej przepustowości z pierwszego
    węzła pośredniego do drugiego węzła pośredniego i z drugiego węzła
    pośredniego do pierwszego węzła pośredniego.
    \emph{testCase4}
    \emph{Algorytm poprawnie znajduje maksymalny przepływ.}

    \item Z dodatkową krawędzią o dodatniej przepustowości ze źródła do ujścia.
    \emph{testCase5}
    \emph{Algorytm
    poprawnie znajduje maksymalny przepływ.}

    \item Z dodatkową krawędzia o dodatniej przepustowości z ujścia do źródła.
    \emph{testCase6}
    \emph{Algorytm poprawnie znajduje maksymalny przepływ.}

    \item Z dodatkową krawędzią o dodatniej przepustowości z ujścia do
    pierwszego węzła pośredniego.
    \emph{testCase7}
    \emph{Algorytm poprawnie znajduje maksymalny przepływ.}

    \item Z dodatkową krawędzią o dodatniej przepustowości z pierwszego węzła
    pośredniego do źródłą.
    \emph{testCase8}
    \emph{Algorytm
    poprawnie znajduje maksymalny przepływ.}

    \item Z dodatkowymi krawędziami o dodatniej przepustowości z ujścia do
    pierwszego węzła pośredniego i z pierwszego węzłą pośredniego do źródła.
    \emph{testCase9}
    \emph{Algorytm poprawnie znajduje maksymalny przepływ.}
\end{itemize}


\subsubsection{Sieć z 2 węzłami pośrednimi połączonymi szeregowo}
Testy wykorzystujące sieci z 2 węzłami pośrednimi znajdują się z kolei w klasie
TwoSerialVerticesTest.
\begin{itemize}[nosep]
    \item Z krawędziami o dodatniej przepustowości ze źródła do pierwszego węzła
    pośredniego, z pierwszego węzła pośredniego do drugiego węzła pośredniego i z
    drugiego węzła pośredniego do ujścia.
    \emph{testCase1}
    \emph{Algorytm
    poprawnie znajduje maksymalny przepływ w sieci.}

    \item Sieć z 2 węzłami pośrednimi połączonymi szeregowo z wieloma
    krawędziami pomiędzy węzłami pośrednimi.
    \emph{testCase2a i testCase2b}
    \emph{Zwraca niepoprawną wartość. Konstruktor ani funkcja validate()
    z klasy FlowNetworkArray nie podnoszą wyjątków.}

    \item Sieć z 2 węzłami pośrednimi połączonymi szeregowo z krawędzią o
    zerowej przepustowości pomiędzy węzłami pośrednimi.
    \emph{testCase3}
    \emph{Poprawnie
    nie znajduje niezerowego maksymalnego przepływu
    w sieci o zerowym przepływie maksymalnym.}

    \item Sieć z 2 węzłami pośrednimi połączonymi szeregowo z krawędzią o
    ujemnej przepustowości pomiędzy węzłami pośrednimi.
    \emph{testCase4}
    \emph{Poprawnie
    nie znajduje niezerowego maksymalnego przepływu
    w sieci z ujemną przepustowością krawędzi w wąskim gardle.}

    \item Sieć z 2 węzłami pośrednimi połączonymi szeregowo z krawędzią z
    drugiego węzłą pośredniego do pierwszego węzła pośredniego.
    \emph{testCase5}
    \emph{Algorytm
    poprawnie znajduje maksymalny przepływ w sieci.}

    \item Sieć z 2 węzłami pośrednimi połączonymi szeregowo z krawędzią od
    pierwszego węzła pośredniego do ujścia.
    \emph{testCase6}
    \emph{Algorytm
    poprawnie znajduje maksymalny przepływ w sieci.}

    \item Sieć z 2 węzłami pośrednimi połączonymi szeregowo z krawędzią od
    źródła do drugiego węzła pośredniego.
    \emph{testCase7}
    \emph{Algorytm
    poprawnie znajduje maksymalny przepływ w sieci.}

\end{itemize}


\subsubsection{Sieć z nieistniejącymi węzłami}


\subsubsection{Sieć z ujemną liczbą węzłów}
\emph{Częściowo zrealizowano w klasie NegativeSizeNetworkTest}
Nie jest możliwe stworzenie sieci z ujemną ilością węzłów.

\subsubsection{Sieć z $1e9$ węzłów}
\emph{Ograniczono wielkość testowanej sieci do $1e3$ węzłów, zrealizowano w klasie HugeFlowNetworkTest}
Ręczne stworzenie sieci przepływu tej wielkości jest nierealne.
Metoda generateEdges z klasy FlowNetworkGenerator ma złożoność rzędu O(n^3),
co znacząco ogranicza możliwości w tym zakresie.
%%n^3 poprawić notację

