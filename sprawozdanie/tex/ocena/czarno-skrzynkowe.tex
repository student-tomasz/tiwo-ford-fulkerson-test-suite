\subsection{Testy czarno\dywiz skrzynkowe}

Testy czarnoskrzynkowe można uruchomić z poziomu testSuite'a BlackBoxText.xml
w pakiecie rw.blackbox.

\subsubsection{Sieć bez węzłów}
Wszystkie testy na sieci bez węzłów zrealizowano w klasie EmptyNetworkTest.

\texttt{testCase2}
Oczekiwany wynik: Exception
Uzyskany wynik: Exception
\textsc{Test zaliczony}

\emph{Algorytm poprawnie nie znajduje niezerowego przepływu maksymalnego w sieci
z samym źródłem.}


\subsubsection{Sieć bez węzłów pośrednich}
Testy na sieciach bez węzłów pośrednich zrealizowano w klasie JustSourceAndSinkTest.
\begin{itemize}[nosep]

    \texttt{testCase3b}
    Oczekiwany wynik: Exception
    Uzyskany wynik: brak wyjątku
    \textsc{Test nie zaliczony}

    \emph{Algorytm poprawnie znajduje maksymalny przepływ w sieci. Algorytm
    nie informuje o istnieniu pętli w sieci.}

    \texttt{testCase8b}
    Oczekiwany wynik: Exception
    Uzyskany wynik: brak wyjątku
    \textsc{Test nie zaliczony}

    \emph{Algorytm poprawnie nie znajduje niezerowego przepływu maksymalnego w sieci
    bez ścieżki o dodatniej przepustowości od źródła do ujścia. Nie jest podnoszony 
    wyjątek w związku z występowaniem krawędzi o ujemnej przepustowości.}

    \item Z wieloma krawędziami skierowanymi od żródła do ujścia.
    \texttt{testCase9a}
    Oczekiwany wynik: 18
    Uzyskany wynik: 4
    \textsc{Test nie zaliczony}

    \texttt{testCase9b}
    Oczekiwany wynik: Exception
    Uzyskany wynik: brak wyjątku
    \textsc{Test nie zaliczony}

    \emph{Algorytm nie odczytuje poprawnie przepustowości ze zwielokrotnionych
    krawędzi. Nie jest podnoszony wyjątek w związku z występowaniem
    zwielokrotnionych krawędzi.}

    \texttt{testCase10b}
    Oczekiwany wynik: Exception
    Uzyskany wynik: brak wyjątku

    \emph{Algorytm poprawnie nie znajduje niezerowego przepływu maksymalnego
    w sieci bez ścieżki o dodatniej przepustowości od źródła do ujścia.
    Nie jest podnoszony wyjątek mimo występowania krawędzi o ujemnej przepustowości
    i krawędzi zwielokrotnionych.}

    \texttt{testCase11b}
    Oczekiwany wynik: 13
    Uzyskany wynik: 7
    \textsc{Test nie zaliczony}

    \texttt{testCase11c}
    Oczekiwany wynik: Exception
    Uzyskany wynik: brak wyjątku
    \textsc{Test nie zaliczony}

    \emph{Algorytm poprawnie znajduje przepływ maksymalny w sieci z dodatkową
    krawędzią o tym samym kierunku, a przeciwnym zwrocie. W przypadku
    zwielokrotnienia, którejkolwiek z tych krawędzi, algorytm zwraca nieprawidłową
    wartość przepływu, nie podnosząc wyjątku w związku z występowaniem krawędzi
    o ujemnej przepustowości.}

\end{itemize}


\subsubsection{Sieć z 1 węzłem pośrednim}
Testy na sieciach z jednym węzłem pośrednim zostały zgrupowane
w klasie SingleVertexTest.
\begin{itemize}[nosep]
}


    \texttt{testCase6b}
    Oczekiwany wynik: Exception
    Uzyskany wynik: brak wyjątku

    \emph{Algorytm poprawnie znajduje maksymalny przepływ w sieci. Algorytm
    nie informuje o istnieniu pętli w sieci.}
    % Z -> P -> U
    %     / ^
    %     \_|

    \item Połączone zwielokrotnionymi krawędziami ze źródła do węzła pośredniego
    i z węzła pośredniego do źródła, z mieszanymi zwrotami.

    \texttt{testCase7b}
    Oczekiwany wynik: Exception
    Uzyskany wynik: brak wyjątku
    \textsc{Test nie zaliczony}

    \texttt{testCase7c}
    Oczekiwany wynik: 16
    Uzyskany wynik: 13
    \textsc{Test nie zaliczony}

\end{itemize}

\subsubsection{Sieć z 2 węzłami pośrednimi połączonymi szeregowo}
Testy wykorzystujące sieci z 2 węzłami pośrednimi znajdują się z kolei w klasie
TwoSerialVerticesTest.
\begin{itemize}[nosep]
    \item Sieć z 2 węzłami pośrednimi połączonymi szeregowo z wieloma
    krawędziami pomiędzy węzłami pośrednimi.
    \texttt{testCase2a}
    Oczekiwany wynik: 15
    Uzyskany wynik: 3
    \textsc{Test nie zaliczony}

    \texttt{testCase2b}
    Oczekiwany wynik: Exception
    Uzyskany wynik: brak wyjątku
    \textsc{Test nie zaliczony}

    \emph{Zwraca niepoprawną wartość. Konstruktor ani funkcja validate() z klasy
    FlowNetworkArray nie podnoszą wyjątków.}

\end{itemize}


\subsubsection{Sieć z nieistniejącymi węzłami}


\subsubsection{Sieć z ujemną liczbą węzłów}
Częściowo zrealizowano w klasie NegativeSizeNetworkTest Nie jest możliwe
stworzenie sieci z ujemną ilością węzłów.

\subsubsection{Sieć z $1e9$ węzłów}
Ograniczono wielkość testowanej sieci do $1e3$ węzłów, zrealizowano w klasie
HugeFlowNetworkTest Ręczne stworzenie sieci przepływu tej wielkości jest
nierealne. Metoda generateEdges z klasy FlowNetworkGenerator ma złożoność
rzędu O(n^3), co znacząco ogranicza możliwości w tym zakresie.
%%n^3 poprawić notację

