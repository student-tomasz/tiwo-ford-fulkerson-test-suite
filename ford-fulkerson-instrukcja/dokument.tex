\documentclass[10pt]{dokument-tiwo}


\begin{document}


\Tytul{Instrukcja obsługi algorytmu Forda-Fulkersona}
\Data{2012-12-18}
\Wersja{1.0}
\Autorzy{MO, RW, TC, MM}
\MakeDokumentMeta


\section{Wstęp merytoryczny}

Opracowywany w ramach projektu algorytm Forda-Fulkersona pozwala na wyznaczenie
sieci przepływowej o maksymalnym przepływie. Algorytm operuje na grafach
spójnych, skierowanych, których krawędzie posiadają przypisane dwie wartości
\emph{przepływ} i \emph{przepustowość}. Dodatkowo, algorytm jest kompletny, gdy
graf, na którym operuje, spełnia trzy warunki:

\begin{enumerate}
    \item \emph{Ograniczenie przepustowości} --- Dla każdej krawędzi
    \emph{przepływ} ma wartość nieujemną i nie przekracza wartości
    \emph{przepustowości} tej krawędzi.
    \item \emph{Zachowanie przepływu} --- Dla każdego wierzchołka, suma
    przepływów wszystkich krawędzi wchodzących jest równa sumie przepływów
    wszystkich krawędzi wychodzących.
    \item \emph{Symetria przepływu krawędzi} --- Dla każdej krawędzi wartość
    przepływu krawędzi jest równa wartości ujemnej dla tej krawędzi w
    przeciwnym zwrocie.
\end{enumerate}

\noindent
Definicje pojęć wykorzystanych w opisie znajdują się paragrafie
\ref{sec:slownik} na stronie \pageref{sec:slownik}.

\noindent
Szczegółowy opis działania algorytmu znajduje się w paragrafie \ref{sec:analiza}
na stronie \pageref{sec:analiza}.


\section{Instrukcja obsługi}

\subsection{Dane wejściowe}
Algorytm Forda-Fulkersona opisany powyżej przyjmuje dwa argumenty:
\begin{itemize}
  \item \texttt{FlowNetwork} -- graf przepływu sieci,
  \item \texttt{Search} -- metoda wykorzystywana do wyszukiwania ścieżek
  powiększających.
\end{itemize}

\subsubsection{Graf przepływu \texttt{FlowNetworkAdjacencyArray}}
Graf przepływu realizowany jest przy pomocy algorytmu z użyciem list
powiązanych. Do każdego wierzchołka grafu przypisane są dwie listy krawędzi,
przednich oraz tylnych. Rozwiązanie charakteryzuje się wysoką przejrzystością. Z
powodu redundacji w postaci duplikowaniu listy krawędzie wykazuje niską
wydajność pamięciową. Niezalecane jest stosowanie reprezentacji w przypadku
dużych, gęstych grafów.\\

\noindent
Argumenty wejściowe konstruktora \texttt{FlowNetworkAdjacencyArray}:
\begin{itemize}
  \item \texttt{int} -- ilość wszystkich węzłów w grafie,
  \item \texttt{int} -- indeks węzła początkowego,
  \item \texttt{int} -- indeks węzła docelowego,
  \item \texttt{Iterator<EdgesInfo>} -- lista krawędzi wraz z ich przepustowością.
\end{itemize}

\subsubsection{Krawędzie}
Informacje o krawędziach grafu przechowywane jako kolekcji instancji klasy
\texttt{EgdeInfo}. Zawierają informacje o wierzchołkach, pomiędzy którymi
dana krawędź się znajduje oraz wartość jej przepustowości.

\noindent
Argumenty konstruktora \texttt{EdgeInfo}:
\begin{itemize}
  \item \texttt{int} -- indeks węzła startowego
  \item \texttt{int} -- indeks węzła końcowego
  \item \texttt{int} -- wartość przepustowa krawędzi
\end{itemize}

\subsection{Dane wyjściowe}
Wyniki wyznaczania maksymalnego przepływu są zapisywane wewnątrz struktury
reprezentującej graf. Instancja \texttt{FlowNetwork} posiada metodę
\texttt{toString()}, która wyświetla aktualny stan grafu.

\noindent
Wynik wywołania \texttt{FlowNetwork.toString()} przed i po
wyznaczeniu maksymalnego przepływu jest zapisywany w formacie:

\begin{lstlisting}
[indeks wezla startowego] -> [indeks wezla koncowego] przeplyw/przepustowosc
\end{lstlisting}

\noindent
Przykładowy wydruk wywołania:

\begin{lstlisting}
Przed:
[0] -> [1] 0/3 @ 0
[0] -> [2] 0/2 @ 0
[1] -> [3] 0/2 @ 0
[1] -> [4] 0/2 @ 0
[2] -> [4] 0/3 @ 0
[2] -> [3] 0/2 @ 0
[3] -> [5] 0/3 @ 0
[4] -> [5] 0/2 @ 0

Po:
[0] -> [1] 3/3 @ 0
[0] -> [2] 2/2 @ 0
[1] -> [3] 2/2 @ 0
[1] -> [4] 1/2 @ 0
[2] -> [4] 1/3 @ 0
[2] -> [3] 1/2 @ 0
[3] -> [5] 3/3 @ 0
[4] -> [5] 2/2 @ 0
\end{lstlisting}


\section{Przykład użycia}
W pierwszej kolejności ustalane są główne parametry grafu:
\begin{itemize}
\item liczba wezłów grafu (poza źródłem i ujściem),
\item indeks węzła źródłowego,
\item indeks węzła docelowego.
\end{itemize}
\begin{lstlisting}
int numVertices = 6;
int srcIndex = 0;
int sinkIndex = 5;
\end{lstlisting}

Następnie definiowane są poszczególne krawędzie grafu wraz z ich
przepustowością:
\begin{itemize}
\item indeks węzła początkowego krawędzi,
\item indeks węzła końcowego krawędzi,
\item maksymalna przepustowość krawędzi.
\end{itemize}
\begin{lstlisting}
ArrayList preIterator = new ArrayList();
EdgeInfo edge1 = new EdgeInfo(0, 1, 3);
EdgeInfo edge2 = new EdgeInfo(1, 3, 2);
EdgeInfo edge3 = new EdgeInfo(3, 5, 3);
EdgeInfo edge4 = new EdgeInfo(1, 4, 2);
EdgeInfo edge5 = new EdgeInfo(0, 2, 2);
EdgeInfo edge6 = new EdgeInfo(2, 4, 3);
EdgeInfo edge7 = new EdgeInfo(4, 5, 2);
EdgeInfo edge8 = new EdgeInfo(2, 3, 2);
preIterator.add(edge1);
preIterator.add(edge2);
preIterator.add(edge3);
preIterator.add(edge4);
preIterator.add(edge5);
preIterator.add(edge6);
preIterator.add(edge7);
preIterator.add(edge8);

Iterator<EdgeInfo> edges = preIterator.iterator();
\end{lstlisting}

Tworzony jest obiekt reprezentujący cały graf przepływu.
\begin{lstlisting}
FlowNetworkAdjacencyList network = new FlowNetworkAdjacencyList(numVertices, srcIndex, sinkIndex, edges);

System.out.println(network.toString());
\end{lstlisting}

Wybierana jest funkcja wyszukująca i wykonywany jest algorytm Forda-Fulkersona.
\begin{lstlisting}
//algorytm przeszukiwania wszerz
BFS_SearchList search = new BFS_SearchList(network);

//algorytm FORD-FULKERSON
FordFulkerson fordFulkerson = new FordFulkerson(network, search);

//wykonaj algorytm
fordFulkerson.compute();
\end{lstlisting}

Wyświetlenie wyznaczonego maksymalnego przepływu.
\begin{lstlisting}
System.out.println("Wynik:");
System.out.println(network.toString());
\end{lstlisting}


\section{Słownik pojęć}
\label{sec:slownik}

\begin{longtable}{@{} >{\bfseries}p{0.19\textwidth} @{\hspace{0.02\textwidth}} p{0.79\textwidth} @{}}
\toprule
Pojęcie & \bfseries{Opis} \\
\toprule
Sieć przepływowa  &
Sieć reprezentująca przepływy pomiędzy węzłem źródłowym i docelowym, z
uwzględnieniem węzłów pośrednich.\\
\midrule
Graf przepływu &
Graf skierowany stanowiący abstrakcję sieci przepływowej, wierzchołki w grafie,
odpowiadają węzłom sieci.\\
\midrule
Wierzchołek grafu przepływu &
Wierzchołek w grafie przepływu reprezentuje węzeł w sieci przepływowej.
Wierzchołki w grafie przepływu, z wyjątkiem wierzchołka źródłowego i docelowego,
muszą spełniać warunek, że suma \emph{f(u,v)} wszystkich krawędzi \emph{(u,v)}
we wpływie do wierzchołka \emph{u}, musi być równa sumie \emph{f(u,w)}
wszystkich krawędzi w wypływie z wierzchołka \emph{u}. Co oznacza, iż żaden
wierzchołek, poza wierzchołkami źródłowym i docelowym, nie mogą produkować ani
konsumować przepływu.\\
\midrule
Krawędź grafu przepływu &
Krawędź w grafie przepływu łączy dwa wierzchołki. Wszystkie krawędzie w grafie
przepływu są krawedziami skierowanymi.\\
\midrule
Przepustowość krawędzi &
Przepustowość krawędzi wyraża ograniczenie, co do maksymalnej liczby jednostek,
które mogą tą krawędzią przepłynąć.\\
\midrule
Przepływ krawędzi &
Przepływ krawędzi definiuje liczbę jednostek przepływających z \emph{u} do
\emph{v} (z punktu początkowego krawędzi do jej punktu końcowego).\\
\midrule
Wierzchołek źródłowy &
Wierzchołek źródłowy (źródło, ang. source) to wierzchołek wytwarzający jednostki
towarów, które przepływają przez krawędzi grafu do wierzchołka docelowego.
Przyjmuje się, że wierzchołek źródłowy jest w stanie wytworzyć dowolną wymaganą
liczbę jednostek towarów, które zostaną z niego odebrane.\\
\midrule
Wierzchołek docelowy &
Wierzchołek docelowy(ujście, stacja końcowa, ang. target, terminus) to
wierzchołek, który konsumuje otrzymane jednostki towarów dostarczone z
wierzchołka źródłowego za pośrednictwem krawędzi grafu. Przyjmuje się, że
wierzchołek źródłowy jest w stanie skonsumować dowolną liczbę jednostek towarów
jakie zostaną do niego dostarczone.\\
\midrule
Ścieżka &
Ścieżka oznacza niecykliczną ścieżkę w grafie z niepowtarzalnymi wierzchołkami,
prowadzącą z wierzchołka źródłowego do wierzchołka docelowego.\\
\midrule
Ścieżka powiększająca &
Ścieżka powiększająca (ang. augmenting path) to taka ścieżka, do której można
dodać więcej przepływu. Co oznacza, że dla każdej krawędzi w ścieżce, przepływ
jest mniejszy od jej przepustowości.\\
\bottomrule
\end{longtable}


\section{Analiza algorytmu}
\label{sec:analiza}

\subsection{Opis algorytmu}
Algorytm Forda-Fulkersona analizuje znalezioną ścieżkę powiększającą, obliczając
maksymalny przepływ dostępny na tej ścieżce. Znajduje najmniejszy z maksymalnych
dostępnych przepływów spośród krawędzi na ścieżce.

Dla krawędzi przednich dostępny przepływ obliczamy jako różnicę przepustowości i
wykorzystanego przepływu \texttt{(u.v).przepustowość-(u.v).przepływ}. Dla
krawędzi tylnych dostępny przepływ jest równy wykorzystanemu przepływowi (do
przodu) na tej krawędzi \texttt{(u,v).przepływ}. Przepływ tylny krawędzi jest
pojęciem abstrakcyjnym na potrzeby prowadzonych obliczeń, a biorąc pod uwagę, że
przepływ musi być nieujemny, dostępny przepływ tylny dla krawędzi jest równy
wykorzystanemu przepływowi do przodu dla tej krawędzi.

Po przeanalizowaniu maksymalnej ścieżki powiększającej i obliczeniu maksymalnego
przepływu dostępnego na tej ścieżce, aktualizujemy przepływy krawędzi. Dla
krawędzi przednich, powiększamy przepływ krawędzi o obliczoną wartość
\texttt{(u,v).przepływ+=delta}. Dla krawędzi tylnych, pomniejszamy przepływ
krawędzi o obliczoną wartość \texttt{(u,v).przepływ-=delta}. Dopóki możliwe jest
znalezienie ścieżki powiększającej, kontynuujemy obliczanie maksymalnego
przepływu dostępnego na kolejnych ścieżkach i aktualizujemy przepływy na
poszczególnych krawędziach tych ścieżek.

Algorytm zakończy swoją pracę w momencie, kiedy nie istnieją już żadne ścieżki
powiększające, tj zostanie wykorzystana maksymalna przepustowość sieci ze źródła
do ujścia.


\section{Historia dokumentu}
\begin{versions}
    \version*{0.1}{2012-12-12}{MO}%
        Utworzono wstępną wersję instrukcji.
    \version{0.2}{2012-12-14}{MM}%
        Dodano rozdział \emph{Przykład użycia}.
    \version{0.3}{2012-12-16}{RW}%
        Dodano rozdział \emph{Słownik pojęć}.
    \version{0.4}{2012-12-16}{RW}%
        Dodano rozdział \emph{Kroki algorytmu}.
    \version{0.5}{2012-12-17}{RW}%
        Dodano skrócony opis algorytmu. Dodano opis algorytmy w języku
        naturalnym.
    \version{0.5.1}{2012-12-17}{MO}%
        Poprawiono literówki i formatowanie.
    \version{1.0}{2012-12-18}{TC}%
        Zatwierdzono.
    \version{1.1}{2012-12-21}{TC}%
        Usunięto kroki algorytmu.
\end{versions}


\end{document}
